\documentclass[
    a4paper,
    12pt,
    hyphens,
    chapterprefix=true,
    headheight=33pt,
    footheight=29pt,
    headings=optiontohead, % optionales Argument wird standardmäßig in Kopfzeile verwendet
]{scrartcl}

%\usepackage[latin1]{inputenc}
\usepackage[utf8]{inputenc}
\usepackage{ngerman}
\usepackage{color}
%\usepackage[]{footmisc}
\usepackage{graphicx}
\usepackage{wrapfig}
\definecolor{darkblue}{rgb}{0,0,.5}
\usepackage{hyperref}
%\def\UrlBreaks{\do\/\do-\do_\do.\do8\do\a\do\b\do\c\do\d\do\e\do\f\do\g\do\h\do\i\do\j\do\k\do\l\do\m\do\n\do\o\do\p\do\q\do\r\do\s\do\t\do\u\do\v\do\w\do\x\do\y\do\z}{}

%\def\UrlBreaks{\do\/\do-\do_\do\_\do=\do\=\do&}

\hypersetup{pdftex=true, colorlinks=true, breaklinks=true, linkcolor=darkblue, menucolor=darkblue, urlcolor=darkblue}
%\usepackage{url}
\usepackage[hyphens]{url}
% http://tex.stackexchange.com/questions/53962/why-are-urls-typeset-with-monospace-fonts-by-default
%\urlstyle{rm}
%\renewcommand\UrlFont{\color{darkblue}\rmfamily\itshape}
\addtolength{\footnotesep}{1mm} % change to 1mm
%https://en.wikibooks.org/wiki/LaTeX/Footnotes_and_Margin_Notes
\interfootnotelinepenalty=1000
%%%%\addtolength{\skip\footins}{1cm} % Länge zwischen Fußnotenbereich und Text 

\usepackage{etoolbox}
\usepackage{geometry}
\geometry{a4paper,left=29mm, right=29mm, top=28mm, bottom=28mm}

\usepackage[
  headsepline,
  footsepline,plainfootsepline,
  automark,
  ]{scrlayer-scrpage}

%\automark{chapter} %scrbook
\automark*{section}
\clearpairofpagestyles
\ohead{\headmark}
\ofoot{\pagemark}

%\KOMAoptions{onpsinit=\linespread{1}\selectfont}% Kopf- und Fußzeilen einzeilig
%\ohead{\headmark}
%\ofoot*{Seite \thepage\\}

% http://mirror.easyname.at/ctan/macros/latex/contrib/fancyhdr/fancyhdr.pdf
%Class scrartcl Warning: Usage of package `fancyhdr' together
%(scrartcl)              with a KOMA-Script class is not recommended.
%(scrartcl)              I'd suggest to use
%(scrartcl)              package `scrlayer-scrpage'.
%(scrartcl)              Nevertheless, using requested
%(scrartcl)              package `fancyhdr' on input line 23.
%\usepackage{fancyhdr}
%\pagestyle{fancy}
%\lhead{\small{Katharina Süß, 0721216}}
%\lhead{}
%\renewcommand{\chaptermark}[1]{\markboth{\thechapter.\ #1}{}}
%\rhead{\small{SE Die Macht der Bilder, SS 2013}}
%\rhead{}
%\chead{}
%\lfoot{}
%\cfoot{}
%\rfoot{\small{\thepage}}
%renewcommand{\headrulewidth}{0.4pt}
%renewcommand{\footrulewidth}{0.4pt}

\setlength{\skip\footins}{1.2cm}
\pagenumbering{arabic}
%\pagestyle{plain}
\setlength{\parindent}{0cm}
\linespread{1.5}


%%%%%%%%%%%%%%%%%%%%%%%%%%%%%%%%%%%%%%%%%%%%%%%%%%%%%%%%%%%%%%%%%%%%%%%%%%%%%%%

%titlepage


%%%%%%%%%%%%%%%%%%%%%%%%%%%%%%%%%%%%%%%%%%%%%%%%%%%%%%%%%%%%%%%%%%%%%%%%%%%%%%%




\begin{document}
%\maketitle
\thispagestyle{empty}
\begin{center}

\textbf{
{\Large
    Impfen: Ja oder Nein? \\
    Eine historische Betrachtung der Impfdebatte des 19. Jahrhunderts im deutschsprachigen Raum im Vergleich zur Gegenwart
    }
}
\\*[1cm]\Large { \textbf{Masterarbeit}
\\*[1cm]Zur Erlangung des Master of Arts
\\*[5mm]an der Kultur- und Gesellschaftswissenschaftlichen Fakultät
\\der Paris-Lodron-Universität Salzburg
\\*[1cm]Eingereicht von:
\\Katharina Süß, Matrikel-Nummer 0721216
\\*[1cm]Gutachter: Ass.-Prof. Mag. Dr. Alfred Stefan Weiss
\\Fachbereich: Geschichte
\vfill
Salzburg, September 2017
}

\end{center}


\newpage
\tableofcontents
\newpage

%%%%%%%%%%%%%%%%%%%%%%%%%%%%%%%%%%%%%%%%%%%%%%%%%%%%%%%%%%%%%%%


\section{Gendererklärung}
Aus Gründen der leichteren Lesbarkeit wird in der vorliegenden Masterarbeit die gewohnte männliche Sprachform bei personenbezogenen Substantiven und Pronomen verwendet. Dies impliziert jedoch keine Benachteiligung des weiblichen Geschlechts, sondern soll im Sinne der sprachlichen Vereinfachung als geschlechtsneutral zu verstehen sein.
\newpage


\section{Einleitung}

\textit{"`Gesundheit und froher Mut, das ist des Menschen höchstes Gut}\footnote{Deutsches Sprichwort: \url{http://www.aphorismen.de/suche?f_thema=Gesundheit&seite=2} 7.4.2016.}."'\\ "`Gesundheit"': Ein auf den ersten Blick unscheinbares und einfach zu definierendes Wort.
Omnipräsent im Österreich des 21. Jahrhunderts, egal ob in den Medien, der Lebensmittel- oder Pharmaindustrie. Eine "`Google-Schlagwort-Suche"' bringt ein Ergebnis von rund 180 Millionen Treffern. Welch hohen Wert die Gesundheit in Österreich hat, bezeugen Institutionen wie das Sozialversicherungssystem und die verpflichtende
Krankenversicherung für jedermann ebenso wie die \textit{World Health Organisation} (WHO) auf internationaler Ebene.\\
Fest steht, dass sich das Verständnis und die Definition von Gesundheit (wie auch von Krankheit) im Laufe der Jahrhunderte enorm verändert hat. Der christliche Kulturkreis zum Beispiel betrachtete (und betrachtet) Gesundheit und Krankheit als etwas Passives, abhängig vom Gehorsam gegenüber Gott. Daher findet sich in der hebräischen Bibel kein eigenes Wort für Gesundheit. Sie erscheint entweder als Gegenstück von Krankheit oder in Verbindung mit "`gesund werden"' und "`gesund bleiben"'.\footnote{Beispiel Moses (Exodus) 23,25.26: \textit{"`Und ihr sollt dem HERRN, eurem Gott, dienen, so wird er dein Brot und dein Wasser segnen; und ich will die Krankheit aus deiner Mitte tun."'}} \\
Als Basisdefinition für Gesundheit im 20. Jahrhundert gilt jene der WHO, welche sie als einen \textit{"`Zustand des vollständigen körperlichen, geistigen und sozialen Wohlergehens"'}\footnote{Verfassung der Weltgesundheitsorganisation vom 22.7.1946, Stand 8. Mai 2014,
S. 1. in: \url{https://www.admin.ch/opc/de/classified-compilation/19460131/201405080000/0.810.1.pdf} 6.4.2016.} betrachtet. 1986 wurde diese Definition im Rahmen der WHO Ottawa-Charta erweitert:\\
\textit{"`Gesundheit wird von Menschen in ihrer alltäglichen Umwelt geschaffen und gelebt: Dort wo sie spielen, lernen, arbeiten und lieben. Gesundheit entsteht dadurch, dass man in die Lage versetzt ist, selber Entscheidungen zu fällen und eine Kontrolle über die eigenen Lebensumstände auszuüben, sowie dadurch, dass die Gesellschaft, in
der man lebt, Bedingungen herstellt, die all ihren Bürgern Gesundheit ermöglicht."'}
\footnote{WHO-Ottawa-Charta: \url{http://www.euro.who.int/__data/assets/pdf_file/0006/129534/Ottawa_Charter_G.pdf} 7.4.2016.}\\
Die WHO definiert Gesundheit somit als etwas, für das Gesellschaft UND Individuum gleichermaßen verantwortlich sind. So verwundert es auch nicht, dass die WHO 1950 einen Weltgesundheitstag einführte, welcher alljährlich am 7. April begangen wird und auf Themen von globaler Relevanz aufmerksam machen soll.\\
Heute, im 21. Jahrhundert, hat sich das Gesundheitsverständnis der Menschen in Österreich dahingehend verändert, als dass "`Gesundheit"' und der Zugang zu den damit verbundenen Einrichtungen gemeinhin als selbstverständlich, ja sogar als grundsätzliches Menschenrecht angesehen werden. Die Vereinten Nationen (UN) haben dieses "`Recht auf Gesundheit"' in den Menschenrechtskonventionen festgehalten und unterscheiden darüber hinaus noch zwischen Erwachsenen und Kindern. Somit findet sich auch in der UN-Kinderrechtskonvention ein eigener Artikel, welcher das Recht des Kindes auf "`\textit{das erreichbare Höchstmaß an Gesundheit}"' festhält.\footnote{UN-Kinderrechtskonvention: \url{https://www.kinderrechtskonvention.info/gesundheitssorge-3601/} 18.6.2017.}\\
\\
Wenn Gesundheit also etwas ist, dass sowohl Individuum als auch Gesellschaft beeinflussen können, verwundert es kaum, dass dem Staat die allgemeine Gesundheit, die \textit{Public Health}, besonders am Herzen liegt; nicht zu Letzt deshalb, da die Versorgung vieler Kranker den Staatshaushalt enorm belastet. Eine Möglichkeit der Staaten, diese "`allgemeine Gesundheit"' zu erreichen, bieten geförderte und frei zugängliche Präventionsmaßnahmen. Das Wort "`Prävention"' entstammt dem lateinischen Verb \textit{praevenire} und bedeutet soviel wie "`verhindern"'\footnote{Gesundheitsprävention: \url{http://www.aamp.at/unsere-themen/praevention/definition-praevention/} 15.4.2016.}: \textit{"`Prävention zielt im Sinne von Krankheitsverhütung
-- anders als die Gesundheitsförderung -- darauf, eine bestimmte gesundheitliche Schädigung oder Erkrankung durch gezielte Aktivitäten zu verhindern, weniger wahrscheinlich zu machen oder zu verzögern."'}\footnote{BMG Prävention: \url{http://www.bmg.gv.at/home/Schwerpunkte/Gesundheitsfoerderung\_Praevention/} 15.4.2016.}\\
Das österreichische Gesundheitssystem stellt seinen Bürgern dafür eine Vielzahl von Maßnahmen zur Verfügung: Mammographie, Ernährungsberatung, den Mutter-Kind-Pass oder die 1974 in Österreich eingeführte allgemeine Vorsorgeuntersuchung.\\
Als eine der ersten und damit ältesten Präventionsmaßnahmen kann wohl die Schutzimpfung vor Infektionskrankheiten genannt werden. Das Bundesministerium für Gesundheit (BMG) sieht in den Impfungen die heute "`\textit{wirksamste Prophylaxe}"'\footnote{BMG Impfen: \url{http://bmg.gv.at/home/Schwerpunkte/Gesundheitsfoerderung\_Praevention/Impfen/} 15.4.2016.} vor den so genannten "`Kinderkrankheiten"'. Dem folgend gibt es seit 1997 in Österreich ein öffentliches Impfkonzept, welches es ermöglicht, alle in Österreich lebenden Kinder bis zum 15. Lebensjahr mit den (gemäß dem BMG) für die öffentliche Gesundheit wichtigen Impfungen zu versehen. Aktuell bietet das Gesundheitsministerium zwölf kostenlose Impfungen an, die im Rahmen des Mutter-Kind-Passes oder bei
Schul-\\impfaktionen angeboten werden. Nach diesem österreichischen Impfplan, erstellt durch das nationale Impfgremium des BMG, sollen bereits bei Säuglingen im Alter von drei Monaten die ersten Schutzimpfungen vorgenommen werden. Eine Impfpflicht gibt es in Österreich jedoch nicht (mehr). Es handelt sich lediglich um eine Empfehlung der zuständigen Behörde.\footnote{Als Ausnahme können hier zum Beispiel die USA angeführt werden. Dort gibt es gegenwärtig einige Bundesstaaten, in welchen eine Impfpflicht zum Zeitpunkt der Einschulung besteht. Vgl.: Matthias Dahl, Impfung in der Pädiatrie und der "`informed consent"' -- Balanceakt zwischen Sozialpaternalismus und Autonomie, in: Ethik in der Medizin, Band 14, Heft 3, Stuttgart, 2002, S. 204.} Entsprechend obliegt es jedem Einzelnen sich \textit{für} oder \textit{gegen} eine oder mehrere Impfungen, beziehungsweise die Impfung seiner Kinder, zu entscheiden. Diese auf den ersten Blick so unscheinbare Frage "`Impfen: ja oder nein?"' beinhaltet auf den zweiten Blick ungeahnte Tiefen, ja sie spaltet die Gesellschaft geradezu in zwei "`verfeindete"' Lager, die, wie es scheint, kaum miteinander zu versöhnen sind: nämlich jene der Impfgegner und jene der Befürworter. Ein Fakt, welcher dabei den wenigsten bekannt sein dürfte, ist, dass diese Diskussion des \textit{Für} und \textit{Wider} von Schutzimpfungen keine allzu neue ist, sondern mindestens seit der Entwicklung und Institutionalisierung der Kuhpockenimpfung existiert. Genaugenommen besteht diese Diskussion (zumindest in Europa) schon seit dem Bekanntwerden der Variolation im 18. Jahrhundert. Die absichtliche Einimpfung der echten Pocken zum Zwecke der Immunisierung stellte den ersten Versuch dar, der verheerenden Seuche mithilfe einer Präventivmaßnahme etwas entgegenzustellen. Den wie Robert Walker im 18. Jahrhundert schrieb: \\
\begin{center}
\textit{"`Unter allen Krankheiten, denen das menschliche Geschlecht unterworfen ist, giebt es keine, welche eine ernstlichere und genauere Untersuchung verdient, als die Pocken, man mag entweder die beständige jährliche Sterblichkeit, so diese Seuche begleitet oder die unangenehmen Folgen bedenken, die sie bey denen hervorbringt, welche bey ihren Verheerungen leben bleiben. Die Einimpfung ist das einzige Mittel, das man in Großbritannien zur Verhinderung der grossen Verheerung in Vorschlag gebracht hat, welche die natürlichen Pocken anrichten; und man muß gestehen, daß die weit geringere Anzahl von Todten, die es bei dieser Methode gab, verglichen mit der ausserordentlich grossen Menge von Menschen, welche durch die natürlichen Pocken weggeraft wurden, große Hoffnung machte, daß sie diesem Endzwcke entsprechen würde. Aber eine Erfahrung von beynahe siebenzig Jahren hat das Irrige dieser Vermuthung dargethan, indem man unläugbar gesehen hat, daß die Sterblichkeit in diesem erwähnten Zeitraume beständig eben so groß gewesen ist, und daß die Krankheit sich jetzt noch eben so tödlich, oder wohl gar noch tödlicher zeigt, als vor der Einpfropfung der Pockeneinimpfung."'}\footnote{Robert Walker, Untersuchung der Pocken in medicinischer und politischer Rücksicht, nebst einer glücklichen Methode, diese Krankheit zu heilen, einer Erklärung der Ursache der Pockengruben, einem Mittel dieselben abzuwenden und einem Anhange über den gegenwärtigen Zustand der Pocken, Leipzig, 1791, S. 1.}
\end{center}


\section{Forschungsgebiet}

In der vorliegenden Master Arbeit soll genau diese Frage des "`Impfen: ja oder nein?"' im Zentrum stehen. Das Kerninteresse dabei liegt nicht im medizinischen Bereich, sondern auf der im deutschsprachigen Raum geführten Impfdebatte im Zeitraum vom 19. bis zum 21. Jahrhundert, wobei die Pro- und Contraargumente im historischen Verlauf in den Fokus rücken.

\subsection{Forschungsfragen}
Um ein derart umfassendes Thema gewinnbringend bearbeiten zu können, werden folgende Forschungsfragen an das Material gestellt:

\begin{itemize}
\item{Wer führte und führt die Impfdebatte beziehungsweise wer beteiligt/e sich daran? Sind es Ärzte, ist es der Staat, die Industrie, die Eltern?}
\item{In welchen Medien wurde und wird die Debatte geführt? Zum Beispiel in Predigten, Zeitungen und Zeitschriften im 19. Jahrhundert und Film, Fernsehen und Internet in der Gegenwart. Als jüngstes Beispiel wäre hier der im April 2016 erschienene Film von Andrew Wakefield\footnote{Ein britischer Arzt veröffentlichte 1998 einen Artikel in der renommierten Zeitschrift \textit{The Lancet}, in dem er Autismus als Nebenwirkung der MMR-Impfung postulierte. Wie sich später herausstellte, hatte Wakefield von einer Anwaltskanzlei ein Honorar dafür erhalten, wirkungsvolle Argumente für eine Schadensersatzklage für eine Gruppe Eltern autistischer Kinder zu sammeln. Der Artikel wurde von der Zeitschrift zurückgezogen und Wakefield mit einem Berufsverbot belegt. Diese Theorie verfolgt er auch in seinem jüngst veröffentlichten Film "`\textit{Vaxxed}"'. Dennoch konnte diese These bis heute nicht wissenschaftlich bewiesen werden. Vgl.: The Lancet, 1998 Feb 28, Heft 351 (9103), S. 637-41, online: \url{http://www.thelancet.com/journals/lancet/article/PIIS0140-6736(97)11096-0/abstract} 25.7.2017 und Martina Lenzen-Schulte, \\Impfungen. 99 verblüffende Tatsachen, Wackernheim, 2008, S. 88.} "`\textit{Vaxxed: From Cover-Up to Catastrophe}"'} zu nennen.
\item{Wie wurde und wird die Impfdebatte im deutschsprachigen Raum geführt? }
\item{Hat sich die Debatte im Laufe der Zeit verändert, wenn ja, inwiefern? Hier scheinen die Meinungen auseinander zu gehen, denn während das österreichische "`Forum Impfschutz"' in erster Linie von (Contra-)Argumenten ausgeht, die nicht neu sind,\footnote{Michael Kunze, Forum Impfschutz: Das österreichische Impfsystem und seine Finanzierung. Lösungsvorschläge für eine alternative Finanzierungsform, Wien, 2010, S. 6.} sieht der Medizinhistoriker Eberhard Wolff immense Unterschiede in der Impfdiskussion von damals und heute, vor allem wegen der strukturellen Verschiedenheit.\footnote{Eberhard Wolff, Medizinkritik der Impfgegner im Spannungsfeld zwischen Lebenswelt- und Wissenschaftsorientierung, S. 83, in: Martin Dinges (Hg.), Medizinkritische Bewegungen im Deutschen Reich (ca. 1870 -- ca. 1933), Stuttgart, 1996, S. 79--108.} }
\end{itemize}

Der zu untersuchende zeitliche Rahmen ergibt sich aus dem Forschungsgegenstand selbst, da sich diese Arbeit die Aufgabe stellt, den gesamten Zeitraum der Impfdebatte zu betrachten, um ein möglichst vollständiges Bild davon zu erhalten. Damit steht am Beginn die Einführung der Kuhpockenimpfungen in die medizinische Praxis Ende des 18. Jahrhunderts und geht über die Einführung der Impfpflichten bis in die Gegenwart. Die genau Geschichte der Schutzimpfung wird im Kapitel "`Historischer Kontext"' dargestellt.\\
Hinsichtlich des geographischen Raumes steht grundsätzlich der deutsche Sprachraum im Fokus, hier vor allem Österreich und Deutschland mit den gegenwärtigen Staatsgrenzen. Allerdings müssen die entsprechenden Veränderungen der Hoheitsgebiete und Herrschaftsverhältnisse im Lauf des untersuchten Zeitraumes mitgedacht werden, befasst sich diese Arbeit doch mit einem Zeitalter enormer geographischer, politischer und sozialer Veränderungen: die Auflösung des Heiligen Römischen Reiches (1806), die Gründung des österreichischen Kaisertums (1804), die Napoleonischen Kriege (1792--1815), die Neuordnung Europas durch den Wiener Kongress (1814--1815), die Gründung des deutschen Kaiserreiches (1861) bis hin zu den beiden Weltkriegen des 20. Jahrhunderts (1914--1918 und 1939--1945) und die daraus resultierende abermalige Neuordnung der Landkarte, um nur die bekanntesten Großereignisse zu erwähnen.

\subsection{Abgrenzung des Forschungsgebietes}
Auf Grund des Umfanges und der anhaltenden Aktualität des gewählten Themas, müssen entsprechende Abgrenzungen vorgenommen werden. Etwa bezüglich der Vielzahl der vorhandenen Impfmöglichkeiten. Hier  werden ausschließlich jene gegen die "`klassischen Kinderkrankheiten"' wie Pocken, Mumps-Masern-Rötteln, Keuchhusten, Kinderlähmung, Diphtherie etc. betrachtet. Der Begriff der "`Kinderkrankheiten"' ist jedoch irreführend, da es sich um Infektionskrankheiten handelt, die grundsätzlich jeden, gleich welchen Alters, befallen können. Etwaige Reiseimpfungen wie Japan-Enzephalitis oder Gelbfieber werden ausgeklammert.\\
Festgehalten werden muss außerdem, dass diese Masterarbeit keine Auflistung aller Pro- und Contraargumente im Sinne eines Impfratgebers darstellt. Es geht keinesfalls darum, ob Impfungen sinnvoll, wirksam oder gefährlich sind. Entsprechend soll keine "`Missionierung"' für oder gegen Impfung stattfinden. Die medizinwissenschaftliche Entwicklung der Impfstoffe wird im historischen Kontext am Rande thematisiert, steht aber ebenso wenig im Blickpunkt wie deren Inhaltsstoffe oder die Frage nach der Verträglichkeit der einzelnen Bestandteile. Ebenso wird keine Bewertung der Argumente oder der Versuch der Bestätigung beziehungsweise Entkräftung derselben seitens der Autorin vorgenommen.

\section{Forschungsstand und Quellenbasis}
Grundsätzlich wurde das Thema Impfen seit seiner Einführung aus unterschiedlichsten Perspektiven und Blickwinkeln bearbeitet. Begonnen beispielsweise bei Edward Jenner (1749--1823), der die Vaccination mit Kuhpocken in die medizinische Praxis eingeführt hat. Die erste Biographie über ihn erschien bereits 1838.\footnote{John Baron, The Life of Edward Jenner, London, 1838.} Über die Impfung selbst wurde ebenfalls bereits früh geschrieben. Man kann hier zum Beispiel die Werke des oberösterreichischen Pfarrers Johann Evangelist Kumpfhofer\footnote{Johann Kumpfhofer, Predigt von der Pflicht der Eltern ihren Kindern die Kuhpocken einimpfen zu lassen, Linz, 1808.} und des österreichischen Arztes Johann de Carro\footnote{Johann de Carro, Beobachtungen und Erfahrungen über die Impfung der Kuhpocken, Wien, 1802.} nennen oder 100 Jahre später jenes von Gustav Adolf Schlechtendahl\footnote{Gustav Adolf Schlechtendahl, Wahn oder Wirklichkeit? Vorurteil oder Wahrheit? Gedanken und Aktenstücke zur Frage der Schutzpocken-Impfung, Berlin, 1908.}. Eine andere, gegenwärtige Publikation ist jene von Marcus Sonntag\footnote{Marcus Sonntag, Pockenimpfung und Aufklärung. Popularisierung der Inokulation und Vakzination. Impfkampagne im 18. und frühen 19. Jahrhundert, Bremen, 2014.}, der sich in seinem Werk mit dem Zusammenhang von Pockenimpfung und Aufklärung befasst. Bis hin zur Aufarbeitung der Geschichte der Pockenimpfungen in geographisch begrenzten Gebieten und aus unterschiedlichen Perspektiven, wie etwa bei Eberhard Wolff\footnote{Eberhard Wolff, Einschneidende Maßnahmen. Pockenschutzimpfung und traditionale Gesellschaft im Württemberg des frühen 19. Jahrhunderts, Stuttgart, 1998.}, welcher die Pockenimpfung in Württemberg aus Patientenperspektive betrachtete. Mit der Geschichte des Impfwesens in Österreich im Speziellen befasst sich zum Beispiel Johannes Wimmer in seinem Werk "`Gesundheit Krankheit und Tod im Zeitalter der Aufklärung"'\footnote{Johannes Wimmer, Gesundheit, Krankheit und Tod im Zeitalter der Aufklärung. Fallstudien aus den habsburgischen Erbländern, Wien, 1991.}. Wenngleich Wimmer hier die Zeit vor der Einführung der Kuhpockenimpfung in Österreich behandelt, gibt er doch einen umfassenden Einblick in das Medizinalwesen der Habsburgermonarchie in der zweiten Hälfte des 18. Jahrhunderts, den Problemen und den Intentionen einer landesfürstlichen Krankheitsprävention sowie den ersten Versuchen einer Einführung der Inokulation in der Steiermark. Für die Auseinandersetzung mit der Impfgeschichte Österreichs ebenfalls zu nennen sind die Beiträge von Michael Prammer\footnote{Michael Pammer, Vom Beichtzettel zum Impfzeugnis. Beamte, Ärzte, Priester und die Einführung der Vaccination, in: Institut für Österreichkunde (Hrg.), Österreich in Geschichte und Literatur, 39. Jahrgang, Heft 1, Jänner 1995, S. 11--29.}, Sabine Falk und Alfred Stefan Weiss\footnote{Sabine Falk und Alfred Stefan Weiss, "`Hier sind die Blattern."' Der Kampf von Staat und Kirche für die Durchsetzung der (Kinder-)Schutzpockenimpfung in Stadt und Land Salzburg (Ende des 18. Jahrhunderts bis ca. 1820), in Mitteilungen der Gesellschaft für Salzburger Landeskunde 131 (131), S. 163--186.}, welche sich mit der Problematik der Einführung der Kuhpockenimpfung in Österreich aus unterschiedlichen Blickwinkeln befassen und unter anderem die Verbindung von Staat und Kirche im Rahmen der "`Blatternpredigten"' thematisieren, auf welche in einem späteren Abschnitt noch näher eingegangen wird. Weiters sind die Beiträge über die österreichische Impfgeschichte zu nennen, welche im Rahmen der Wiener Medizinischen Wochenschrift publiziert wurden, wie etwa jene von Heinz Flamm\footnote{Heinz Flamm, Pasteurs Wut-Schutzimpfung - vor 130 Jahren in Wien mit Erfolg begonnen und doch offiziell abgelehnt, in: Wiener Medizinische Wochenschrift, Heft 165, Wien, 2015, S. 322--339.}, Christian Vutuc\footnote{Heinz Flamm, Christian Vutuc, Geschichte der Pocken-Bekämpfung in Österreich, in: Wiener klinische Wochenschrift, Heft 122, Wien, 2010, S. 265--275.}, Ingomar Mutz und Diether Spork\footnote{Ingomar Mutz, Diether Spork, Geschichte der Impfempfehlungen in Österreich, in: Wiener Medizinische Wochenschrift, Heft 157/5, Wien, 2007, S. 94--97.}, oder Markus W. Moser und Beatrix Patzak\footnote{Markus W. Moser und Beatrix Patzak, Variola: zur Geschichte einer museal präsenten Seuche, in: Wiener klinische Wochenschrift, Heft 120, Wien, 2008, S. 3--10.}. Eine andere Möglichkeit der Betrachtung der Impfgeschichte zeigen Michael Memmer\footnote{Michael Memmer, Die Geschichte der Schutzimpfung in Österreich. Eine rechts-historische Analyse, in: Gerhard Aigner (Hrg.), Markus Grimm (Hrg.) u. a., Schutzimpfungen -- Rechtliche, ethische und medizinische Aspekte. Schriftenreihe Ethik und Recht in der Medizin, Band 11, Wien, 2016, S. 7--36.} in seiner rechts-historischen Analyse über die Schutzimpfung in Österreich und Alexander Langbauer\footnote{Alexander Langbauer, Das Österreichische Impfwesen unter besonderer Berücksichtigung der Schutzimpfung, Linz, 2010.}, der sich in seiner Dissertation mit der Entwicklung der dazugehörigen Gesetze befasst.\\
Für die Auseinandersetzung mit Anti-Impfbewegungen im deutschsprachigen Raum gibt es nur wenige Beispiele zu nennen, wie die Dissertation von Caroline Marie Humm\footnote{Caroline Marie Humm, Die Geschichte der Pockenimpfung im Spiegel der Impfgegner, München, 1986.} oder das Buch von Martin Dinges\footnote{Martin Dinges (Hg.), Medizinkritische Bewegungen im Deutschen Reich, 1870 bis 1933, Stuttgart, 1996.} über die medizinkritischen Bewegungen im Deutschen Reich. Als jüngste Neuerscheinung zum Thema Impfgeschichte kann das Werk von Wolfgang Eckart\footnote{Wolfgang Uwe Eckart (Hg.), Jenner. Untersuchungen über die Ursachen und Wirkungen der Kuhpocken, Berlin, 2016.} genannt werden. Als jüngste Jenner-Biographie gilt das Werk von Herve Bazin\footnote{Herve Bazine, The Eradiction of Smallpox. Edward Jenner and the First and Only Eradication of a Human Infectious Disease, San Diego, 2000.}.\\
\\
Bei den genannten Autoren und Werken handelt es sich um einen Überblick der bestehenden Literatur, denn auf Grund der langen Geschichte der Impfung ist die Zahl der darüber verfassten Werke entsprechend groß. Man muss hier jedoch große Vorsicht walten lassen, denn wegen der kontrovers geführten Diskussion und den uneingeschränkten Publizierungsmöglichkeiten mittels neuer und sozialer Medien entspricht lange nicht jedes Werk wissenschaftlichen Standards. Beispielhaft genannt werden können hierfür die Werke von Johann Loibner, Gerald Buchwald oder Stefan Lanka und Karl Krafeld\footnote{Karl Krafeld, Stefan Lanka, Impfen -- Völkermord im dritten Jahrtausend? Stuttgart, 2003. \textit{Anmerkung der Autorin: Dieses Werk ist als wissenschaftlich fragwürdig einzustufen.}}. Diese Autoren zählen zu den extremen Impfgegnern, welche in erster Linie verschwörungstheoretisch argumentieren und mitunter grundsätzliche, medizinische Tatsachen in Abrede stellen, wie etwa jene, dass es Viren gibt, welche Krankheiten auslösen\footnote{Preisausschreiben zum Beweis des Masernvirus: \url{https://web.archive.org/web/20120329214816\\/http://www.klein-klein-verlag.de/Viren-|-Erschienen-in-2011/24112011-das-masern-virus-100000-euro-belohnung.html} 29.1.2016.}. Der studierte Biologe Lanka erhielt für seine Verleugnung der Existenz von Viren 2015 den wissenschaftlichen Schmähpreis "`Goldenes Brett vorm Kopf"'.\footnote{Goldenes Brett vorm Kopf: \url{http://wien.orf.at/news/stories/2738165/} 29.1.2016.}\\
Um diesem Problem Rechnung zu tragen, werden jene Literaturnachweise in den Fußnoten durch eine Anmerkung der Autorin ausdrücklich gekennzeichnet, welche als problematisch einzustufen sind, jedoch zur Veranschaulichung der Bandbreite der Impfdiskussion dienen. Des Weiteren wird die verwendete Sekundärliteratur im Rahmen der Bibliographie in Quellen und Literatur unterschieden, wobei das Jahr 1945 als Zäsur gesehen wird. Werke, die davor erschienen, gelten als Quelle, da sie nicht den gegenwärtigen wissenschaftlichen Standards entsprechen. Jene nach 1945 werden als wissenschaftliche Literatur eingestuft.\\
Ganz allgemein muss dieser Arbeit daher ein sehr ausführliches und exaktes Literaturstudium vorangehen. Dabei lassen sich auch etwaige Lücken feststellen. Während es für das 19. Jahrhundert ausführliche und detaillierte Schilderungen zur Entdeckung und Entwicklung der Impfungen gibt, ist es auffällig schwierig, ähnliche Werke für das 20. Jahrhundert zu finden, welche die dort gemachten Entdeckungen (MMR, Polio, ...) schildern. Ähnlich problematisch ist es, verlässliche Quellen über Probleme oder Unfälle bei der Impfstoffherstellung, Angaben über geschädigte Personen oder Rückrufe von Impfstoffen zu erhalten. Die entsprechenden Informationen finden sich, wenn überhaupt, nur vereinzelt in Beiträgen von Ärztezeitschriften und anderen Zeitungen. Ebenfalls mangelt es an Literatur zur Geschichte der Impfgegnerbewegung in Österreich. Beispielhaft kann hier der Aufsatz von Dorothy und Roy Porter\footnote{Dorothy Porter, Roy Porter, The politics of Prevention: Anti-Vaccinationism and Public Health in nineteenth-century England, in: Medical History, Heft 32, 1988, S. 231--252.} herausgegriffen werden, welcher die Institutionen der Impfgegner im England des 19. Jahrhunderts beschreibt.\\
\\
Als Quellenbasis für die Auswertung der Argumente, welche zum Vergleich herangezogen werden sollen, dienen die medizinhistorischen Werke des 19. Jahrhunderts zu den Themen Impfen sowie Kinder- und Infektionskrankheiten, welche sich in den Archiven und Bibliotheken der Medizinischen Gesellschaft OÖ, der Universitätsbibliothek Salzburg, der Landesbibliothek Oberösterreich und der Österreichischen Nationalbibliothek befinden. Dazu kommen aus dem genannten Themenkreis die Werke des 20. und 21. Jahrhunderts.\\
Bei den herangezogenen Quellen handelt es sich ausschließlich um Bücher, die entweder in Druckform oder als Digitalversion verfügbar sind. Vor allem die älteren Werke des frühen 19. Jahrhunderts sind über die Österreichische Nationalbibliothek mittlerweile vermehrt digital und online zugänglich. Da es sich hierbei um eine Untersuchung der Impfdebatte im deutschsprachigen Raum handelt, wurden entsprechende deutschsprachige Quellen herangezogen.\\
In Anbetracht der langen Dauer des untersuchten Zeitrahmens und den unterschiedlichen Themenkreisen sind verschiedene Textarten zu erwarten: Zunächst sind die medizinhistorischen Texte des 19. Jahrhunderts zu erwähnen, welche sich in erster Linie mit dem Für und Wider der Pockenimpfung, deren Wirken, Folgen und Problemen oder den damit einhergehenden gesetzlichen Zwangsmaßnahmen dieser Zeit befassen. Der Fokus dieser frühen Quellen auf die Kuhpockenimpfung liegt darin begründet, dass diese die Erste und lange Zeit die einzige Impfung war, wie im Kapitel "`Historischer Kontext"' erläutert wird. Die medizinhistorischen Werke des späten 19. und frühen 20. Jahrhunderts beziehen sich zwar ebenfalls in erster Linie auf die Pockenimpfung, befassen sich dabei aber auch ganz allgemein mit Fragen und Überlegungen zur Gesundheitspflege des Kindes oder der Verbreitung,respektive Verhütung, von Seuchen. In der zweiten Hälfte des 20. und bis zur Gegenwart sind außerdem medizinische Fachliteratur und populärwissenschaftliche Literatur zu erwarten. Diese Werke thematisieren das Pro und Contra von Schutzimpfungen aus den unterschiedlichsten Perspektiven sowie allgemeine medizinische Fragen rund um die genannten Themenkreise Impfen, Kinder- und Infektionskrankheiten. Für diesen Zeitraum kommen zudem noch Elternratgeber dazu, die seit Ende des 20. Jahrhunderts mehr und mehr an Popularität gewannen. Diese befassen sich in der Regel mit allen Bereichen, die das Heranwachsen von Kindern betreffen, so eben auch mit Fragen zur Gesundheit, dem Behandeln und Verhüten von Krankheiten und eben Meinungen über das Für und Wider von Impfungen. Damit werden Elternratgeber ebenfalls als Quellen für die Argumentenliste herangezogen.\\
Die Entstehungszusammenhänge dieser Werke sind zwar grundsätzlich interessant, finden hier jedoch keine nähere Betrachtung, da der Inhalt der Texte im Vordergrund steht. Viele der als Quellen zum Erstellen der Argumentenliste herangezogenen Werke, vor allem aus dem populärwissenschaftlichen Bereich und den Elternratgebern des 20. und 21. Jahrhunderts, entsprechen, ähnlich wie bei der ausgewählten Literatur, nicht den heutigen wissenschaftlichen Standards. Somit muss auch an dieser Stelle betont werden, dass das Heranziehen dieser Werke ausschließlich dazu dient, die Bandbreite der Impfdebatte darzustellen.\\
Jene Quellen, aus denen die Argumente entnommen wurden, finden sich in einem eigenen Literaturverzeichnis im Anhang. Die Liste mit den ausgewerteten Argumenten befindet sich ebenfalls im Anhang. Eine detaillierte Beschreibung über die Auswahl der Argumente, deren Kategorisierung und Inhalt, findet sich in den Kapiteln "`Kategorisierung und Auswertung"', dem "`Vergleich"' sowie dem "`Fazit"'.

\section{Der theoretische Rahmen: Die Modernisierungstheorie}
Um sich der gestellten Forschungsfragen wissenschaftlich fundiert annähern zu können, sollen im folgenden Überlegungen zum theoretischen Rahmen angestellt werden. Bei genauerer Betrachtung und Analyse dieses sozialgeschichtlichen Themas erschien das Heranziehen der Modernisierungs- respektive der Medikalisierungstheorie als sinnvoll.\\
\\
Die Modernisierungstheorie wurde in den 1950er Jahren von politisch motivierten Sozialwissenschaftlern formuliert und befasste sich in ihrer ursprünglichen Form mit der Entwicklungslogik neuzeitlicher Gesellschaften.\footnote{Thomas Mergel, Modernisierung, Pkt. 1, in: Europäische Geschichte Online (EGO), hg. vom Institut für Europäische Geschichte (IEG), Mainz 2011-04-27. \url{http://www.ieg-ego.eu/mergelt-2011-de} 24.1.2016.} Sie ist zurückzuführen auf das Bedürfnis dieser Wissenschaftler nach einem allgemeineren, alternativen Sammelbegriff für die negativ konnotierten Konzepte jener Zeit, wie "`Europäisierung, Verwestlichung oder Zivilisierung"'.\footnote{Hans-Ulrich Wehler, Modernisierungstheorie und Geschichte, Göttingen, 1975, S. 11.} "`Modernisierung"' erschien dafür als die attraktivste Alternative, da es sich um einen sehr vieldeutigen, in erster Linie positiv assoziierten Begriff handelte, welcher seither in den theoretischen und historischen Sozialwissenschaften intensiv verhandelt wurde und wird. Unter dem Dach der "`Modernisierungstheorie"' versteht man jedoch keine einheitlich formulierte Theorie. Vielmehr sammelt sich unter diesem Stichwort ein Konglomerat an Überlegungen zu langwierigen wirtschaftlichen Entwicklungen, sozialen Erhebungen bis hin zu empirischen Untersuchungen der politischen Kultur.\footnote{Mergel, Modernisierung, Pkt. 2.} In der Nachkriegszeit war Modernisierung lange an eine bestimmte innere und äußere Situation des Staates als Zielvorstellung gebunden und galt im Grunde als Synonym für "`Amerikanisierung"'. Man dachte "`Modernisierung"' dabei als einen Prozess, bei welchem sich der Zustand einer Gesellschaft von Traditionalität befreien, Züge der Moderne annehmen und sich dafür bestimmter progressiver, unausweichlicher Prozesse wie Industrialisierung, Demokratisierung, Bürokratisierung und Säkularisierung bedienen würde. Markant war dabei das dichotome Gegenüberstellen von kategorisierter Modernität und traditionalem Gegensatz, wie etwa hohe Lebenserwartung in moderner versus geringer Lebenserwartung in traditionellen Gesellschaften.\footnote{Wehler, Modernisierungstheorie, S. 14--17.}\\
Die Modernisierungstheorie verfügte über große Anziehungskraft, nicht zuletzt deswegen, weil sie eine Art "`Entwicklungsschablone"' für politische Handlungsrichtungen beinhaltete. Ebenso schnell geriet sie gerade dafür in Kritik. Bis heute wird die Theorie diskutiert und verhandelt, wobei man sich im Verlauf der Forschung von Zielvorstellungen und davon abhängigen Entwicklungsprozessen gelöst hat hin zu einem neuen Konzept, worin Moderne sich selbst wahrnimmt, historisiert und reflektiert.\footnote{Mergel, Modernisierung, Pkt. 17.}\\
\\
Befasst man sich nun mit Themen aus der sozialhistorischen Sparte der Medizin, wird für die Modernisierungstheorie gerne auch der Begriff der "`Medikalisierungstheorie"' verwendet. Nur, was bedeutet \textbf{Medikalisierung}? Darunter versteht man jenen Prozess, bei welchem die menschliche Lebenswelt mehr und mehr in den Fokus der medizinischen Wissenschaft und des Staates gerät. Der Beginn dafür wird im 18. Jahrhundert gesehen, als der aufgeklärt-absolutistische Staat die Gesundheit, respektive Krankheit seiner Bürger als gesellschaftspolitisches Problem erkannte und sich darum annahm. Dies führte zu einer staatlich unterstützten und geförderten Professionalisierung und Monopolisierung des Ärzteberufes, was das Gesundheits- und Krankheitsverhalten der Bevölkerung maßgeblich beeinflusste und veränderte.\footnote{Francisca Loetz, Vom Kranken zum Patienten. "`Medikalisierung"' und medizinische Vergesellschaftung am Beispiel Badens 1750--1850, Stuttgart, 1993, S. 14--15.} Anfangs betrachtete die Wissenschaft diesen Vorgang sehr einseitig, nämlich ausschließlich "`von oben nach unten"' und sah damit in der Medikalisierung einen Prozess, den Staat und Ärzte der Gesellschaft gegen deren Willen "`überstülpten"'. Dieser Aspekt der Theorie wurde bald einiger Kritik unterzogen und nicht zuletzt mit der vermehrt praktizierten Patientengeschichte setzte ein Perspektivenwechsel ein. Man kam zu dem Schluss, dass es sich bei der Medikalisierung um keinen einseitigen Prozess, sondern vielmehr um ein \textit{top down and bottom up} handelte. Demnach haben nicht Ärzte und Staat allein die Medikalisierung getragen, sondern auch die Bevölkerung war an diesem Prozess beteiligt und zwar in dem Sinne als die Menschen unter den, strukturbedingt zur Wahl stehenden, Heilverfahren jenes in Anspruch nahmen, von welchem sie sich am ehesten Erfolg versprachen und das den ökonomischen Rahmenbedingungen entsprach.\footnote{Wolfgang Uwe Eckart, Robert Jütte, Medizingeschichte. Eine Einführung, Köln, 2007, S. 175.} Nachfrage und Angebot bedingen sich also vielmehr, als dass sie sich wie in der traditionellen Modernisierungstheorie ausschließen, beziehungsweise dichotom gegenüberstehen.\footnote{Marina Hilber, Institutionalisierte Geburt. Eine Mikrogeschichte des Gebärhauses, Bielefeld, 2012, S. 26.} \\
Die Sozialhistorikerin Francisca Loetz plädiert daher dafür, eher von einer medizinischen Vergesellschaftung als von Medikalisierung als reinem Disziplinierungsprozess zu sprechen. Sie beruft sich dabei auf Georg Simmels \textit{Soziologie} und dessen Überlegung, \textit{"`dass Gesellschaft konstituiert wird durch die Handlungszusammenhänge, die zwischen ihren Mitgliedern entstehen"'}.\footnote{Loetz, Vom Kranken zum Patienten, S. 15.}\\
Neben der bereits zitierten Studie von Francisca Loetz kann jene des Medizinhistorikers Eberhard Wolff über die Annahme und Ablehnung der Pockenimpfung aus Patientenperspektive als Beispiel für die Medikalisierungstheorie im Sinne der Vergesellschaftungsidee genannt werden. Wolff betrachtet jedoch die umgekehrte Perspektive. Er nahm nicht nur die Sicht des Patienten ein, sondern steckte die Rahmenbedingungen für das Konzept der Traditionalität neu ab und orientierte sich an kategorisierten Idealtypen, um Tendenzen aufzuzeigen. Damit wollte er dem Begriff der Traditionalität den ausschließlich negativen Bezugsrahmen zur Modernität nehmen.\footnote{Wolff, Einschneidende Maßnahmen, S. 92.}\\
\\
Wie kann die Medikalisierungstheorie, als Aspekt der Modernisierungstheorie, nun in dieser Arbeit Anwendung finden? Als markante Stichwörter des bisher Erläuterten können "`Prozess"', beziehungsweise "`Entwicklung"' und "`Kategorien"' hervorgehoben werden. Betrachtet man die Geschichte der Schutzimpfung, so kann man sagen, dass die Einführung der Pockenimpfung und die damit einhergehenden gesetzlichen Regelungen über allgemeine Impfpflichten zu jenen Maßnahmen zählen, welche zu einer maßgeblichen Medikalisierung der Bevölkerung beigetragen haben:
\begin{itemize}
\item{Die impfbedingten Reglementierungen erweiterten nicht nur den Funktionsbereich der Ärzteschaft, sondern bezogen auch die ländliche Bevölkerung mit ein, welche davor tendenziell arztfern gelebt hatte.}
\item{Die Impfung war das erste, wirksame und vor allen Dingen präventive Mittel gegen eine weit verbreitetet Infektionskrankheit mit hoher Todesrate.}
\item{Die Impfung entspricht der zeitgenössischen, von der Aufklärung geprägten Idee, aus der Unmündigkeit herauszutreten und aktiv sein weltliches Schicksal in die Hand zunehmen.}
\end{itemize}
Auf dieser Basis lässt sich folgende These als Ausgangspunkt dieser Arbeit formulieren: Die Entdeckung, Einführung und Verbreitung der Pockenimpfung zog eine Welle von medizinwissenschaftlichen sowie medizin-hygienischen Erfindungen und Medikalisierungsmaßnahmen nach sich. All diese Entwicklungen bedingten einen großen gesellschaftlichen Wandel, der unter anderem das Gesundheits- und Krankheitsverhalten der Menschen maßgeblich veränderte. Bedenkt man die lange Dauer der Impfdebatte, die sich vom 19. Jahrhundert ununterbrochen bis zur Gegenwart erstreckt, ist es naheliegend, dass dieser gesellschaftliche Wandel auch anhand der Argumente der Impfdebatte nachvollziehbar sein muss.


\section{Methoden}
Bevor man sich nun dem inhaltlichen Teil der Arbeit zuwenden kann, bedarf es nach den theoretischen noch methodischer Überlegungen. Da die Argumente der Impfdebatte im Fokus liegen, bedarf es einer Methode, welche in der Lage ist, die Inhalte der bearbeiteten Quellen herauszufiltern, mit deren Hilfe die eingangs gestellten Forschungsfragen beantwortet werden können. Dem entsprechend fiel die Entscheidung auf eine inhaltsanalytische Methode aus den Sozialwissenschaften, welche eingeschränkt zur Anwendung kommt. Zum anderen handelt es sich, wie bereits im Titel angekündigt, um eine vergleichende Arbeit. Entsprechend kommt die Methode des historischen Vergleiches zur Anwendung. Beide Methoden werden im folgenden erläutert.

\subsection{Qualitative Inhaltsanalyse}
Inhaltsanalytische Verfahren sind heute aus der Sozialforschung nicht mehr weg zu denken und werden mittlerweile von vielen unterschiedlichen Wissenschaftsdisziplinen verwendet. Sie dienen der Analyse von Interviews, Texten, Nachrichten oder Zeitungsbeiträgen. Der Begriff der Inhaltsanalyse ist eine Übersetzung des englischen "`\textit{content analysis}"' und bezeichnet eben jenes Verfahren, welches der Analyse von Kommunikationsinhalten, primär von Texten dient.\footnote{Peter Atteslander, Methoden der empirischen Sozialforschung, Berlin, 2010, S. 195.} Innerhalb der Inhaltsanalyse unterscheidet man zwischen quantitativen und qualitativen Verfahren, wobei sich die qualitativen Ansätze in erster Linie aus der Kritik an den Quantitativen heraus entwickelt haben. Diese wurden von einigen Forscher zwar als exakte Verfahren, aber "`inhaltsleer"' eingestuft. Im Laufe der Zeit haben sich auch hier, im Rahmen der qualitativen Inhaltsanalyse, unterschiedliche Verfahren entwickelt, die in ihrer Gesamtheit in diesem Rahmen unmöglich darzustellen sind.\footnote{Atteslander, Methoden der Sozialforschung, S. 198 u. 212.} Für diese Arbeit soll das vom Sozialwissenschaftler Philipp Mayring, auf Grundlage der quantitativen Inhaltsanalyse der Kommunikationswissenschaften, entwickelte Verfahren der qualitativen Inhaltsanalyse angewendet werden. Die Methode basiert auf der Tradition der Hermeneutik und zielt darauf ab, mittels betrachteter Aussagen Rückschlüsse auf das zu analysierende Material zu gewinnen und dadurch zum Beispiel Absichten des Absenders oder Wirkungen beim Empfänger aufzuzeigen.\footnote{Philip Mayring, Qualitative Inhaltsanalyse. Grundlagen und Techniken, 12. Auflage, Basel, 2015, S. 12--13.} Es geht bei dieser schlussfolgernden Analysetechnik also um eine Erfassung des Untersuchungsgegenstandes unter Berücksichtigung der Vielfalt und Komplexität der menschlichen Wirklichkeit.\footnote{Mayring, Inhaltsanalyse, S. 19--20.}\\
Ausgangspunkt dafür muss eine theoretische Fragestellung sein, vor deren Hintergrund der Forschungsgegenstand ausgewertet wird. Herangezogen werden kann dafür jede Art von fixierter Kommunikation wie Interviews, Dokumente, aber auch Bilder oder Musik. Das jeweils ausgewählte Material wird einerseits nach Themen und Gedankengängen als primären Inhalt durchforstet und andererseits nach latenten Inhalten, welche durch Textinterpretation und Kontext erschlossen wird, untersucht.\footnote{Philipp Mayring, Qualitative Inhaltsanalyse, in: Forum: Qualitative Sozialforschung, Volume 1, No. 2, Art. 20, Juni 2000, o.A., \url{http://www.qualitative-research.net/index.php/fqs/article/view/1089/2383} 25.7.2017.} Der bereits zitierte Eberhard Wolff verwendete diese Methode in seiner Studie über die Pockenimpfung aus Patientenperspektive und brachte für diese Suche nach primären und latenten Inhalten ein plakatives Beispiel: Hinter dem früher häufig angeführten Argument, dass Kinder nicht geimpft werden sollten, da die Eltern nicht in das Schicksal eingreifen wollten, scheint auf den ersten Blick religiöser Prädestinationsglaube zu stehen. Beim zweiten Blick auf den Kontext meint Wolff, unter Anwendung der qualitativen Inhaltsanalyse, die Scheu der Eltern zu erkennen, die nicht die Verantwortung für eine als riskant wahrgenommen Maßnahme übernehmen wollten.\footnote{Wolff, Einschneidende Maßnahmen, S. 50.}\\
\\
Ein besonderes Augenmerk legt die qualitative Inhaltsanalyse auf Sprache und Wortwahl. Ziel von Mayrings Analyse ist eine Reduktion des Materials. Die ausgewählten Textbausteine werden dafür vom Betrachter durch Abstraktion des Inhaltes so zusammengefasst, dass am Ende ein überschaubarer Korpus geschaffen wurde, welcher aber immer noch ein Abbild des Grundmaterials darstellt.\footnote{Atteslander, Methoden der Sozialforschung, S. 213.} Hier kommt es zu einer Einschränkung der Methode, da die einzelnen Argumente möglichst wortgetreu übernommen und analysiert werden sollen. Dennoch geht es trotz der Verwendung der Vollzitate auch hier um die Zusammenfassung der Textausschnitte im Rahmen eines Kategoriensystemes, wie es die Methode der Inhaltsanalyse vorsieht. Diese Kategorien können entweder induktiv oder deduktiv entwickelt werden. Wie gestaltet sich nun die praktische Anwendung der qualitativen Inhaltsanalyse für diese Arbeit?

\subsubsection{Ablaufmodell}
Um die Intersubjektivität der Methode zu gewährleisten, schlägt Mayring die Erstellung eines Ablaufmodells mit folgenden Arbeitsschritten vor, an denen sich diese Arbeit orientiert:\\
\\
\textbf{Festlegung des Materials}: Hier geht es um die Auswahl der Kommunikationsart, welche sich auf die Fragestellung bezieht und im Rahmen des Verfahrens analysiert werden sollen. Für die Betrachtung der historischen Impfdebatte werden gedruckte Quellen herangezogen, aus welchen die Argumente herausgenommen werden. Es handelt sich dabei, wie erläutert, zum einen um medizinhistorische Werk des 19. Jahrhunderts, zum anderen um Quellen aus dem 20. und 21. Jahrhundert. Der Themenkreis der Werke, aus welchen die Argumente entnommen werden, umfasst die Themen Impfen, Kinder- und Infektionskrankheiten sowie Elternratgeber, welche sich mit Impfen auseinandersetzen. Nach der Bandbreite des Untersuchungszeitraumes und -materials ist zu erwarten, dass man während der Analyse auf verschiedene Textarten trifft, wie medizinhistorische Texte, gegenwärtige medizinische Fachliteratur, populärwissenschaftliche oder allgemeine Literatur.\\

\textbf{Analyse der Entstehungssituation}: "`Entstehungssituation"' bezieht sich hier einerseits auf jene des Materials und andererseits auf jene des Untersuchungsgegenstandes, nämlich der Impfdebatte.  Hier kommt es dahingehend zu einer Einschränkung, als dass die Entstehungssituationen der verwendeten Literatur und Quellen nicht behandelt werden, da diese zwar durchaus interessant ist, jedoch den Rahmen der Arbeit bei weitem sprengen würde. Der zweite Aspekt wird in erster Linie im Kapitel "`Historischer Kontext"' behandelt.\\
\\
\textbf{Formale Charakteristika des Materials}: Dieser Punkt betrifft die Quellen, welche zur Analyse herangezogen werden. Wie im Kapitel "`Forschungsstand und Quellen"' erläutert wurde, handelt es sich um Quellen, die entweder in Papierform in Archiven und Bibliotheken vorliegen oder als Digitalversion in einer Onlinebibliothek zur Verfügung stehen. In jedem Fall handelt es sich um Originale in Textform.\\
\\
\textbf{Richtung der Analyse}: Betrachtet werden sollen die Pro- und Contraargumente der Impfdebatte und zwar dahingehend, dass mit Hilfe des Kategoriensystems und des Vergleiches die eingangs gestellten Forschungsfragen beantwortet werden können.\\
\\
\textbf{Theoriegeleitete Hypothese}: Basierend auf den Erläuterungen des fünften Kapitels wurde die Arbeitsthese formuliert, dass die Entdeckung, Einführung und Verbreitung der Pockenimpfung eine Welle von medizinwissenschaftlichen sowie medizin-hygienischen Erfindungen und Medikalisierungsmaßnahmen nach sich zog und dass all diese Entwicklungen einen großen gesellschaftlichen Wandel bedingten, der unter anderem das Gesundheits- und Krankheitsverhalten der Menschen maßgeblich veränderte. Bedenkt man die lange Dauer der Impfdebatte, die sich vom 19. Jahrhundert ununterbrochen bis zur Gegenwart erstreckt ist es naheliegend, dass dieser gesellschaftliche Wandel auch anhand der Argumente der Impfdebatte nachvollziehbar sein muss.\\
\\
\textbf{Analysetechniken}: Wie bereits mehrfach erwähnt, sieht die Methode die Einteilung in induktive oder deduktive Kategorien vor. Für die Einteilung der Autoren im Sinne eines Personenkreises und der Gruppen, bezogen auf Gegner oder Befürworter, scheint eine deduktive Kategorisierung sinnvoll. Sprich, die Kategorien werden auf Basis des Forschungsgegenstandes an das Material herangetragen. Eine entsprechend Einteilung wird im folgenden Unterkapitel "`Kategorien"' vorgenommen und beschrieben. Für die Einordnung der Argumente wiederum erscheint eine induktive Form der Kategorienbildung aus der Textanalyse heraus, sinnvoll. Das genaue Vorgehen wird im Kapitel "`Kategorisierung und Auswertung"' detailliert erläutert. \\
\\
\textbf{Festlegung der Analysetechnik}: Basierend auf den Grundformen der Interpretation empfiehlt Mayring die Verwendung der Zusammenfassung, der Explikation und/oder Strukturierung. Für diese Arbeit empfiehlt sich die Zusammenfassung, da diese Technik am ehesten einem induktiven Ansatz gerecht wird. In diesem Fall werden die inhaltstragenden Stellen aus dem Text herausgenommen und in einem mehrstufigen Prozess kategorisiert. Die Kategorienbezeichnung ist dabei ein sinnbezogener Begriff oder ein Satz aus dem Text. Textstellen mit ähnlicher Bedeutung werden unter den Kategorien subsumiert. Werden neue Inhalte ausfindig gemacht, werden sie entsprechend neuen Kategorien zugeordnet, wobei die Kategorien im Laufe des Prozesses immer wieder einer Überprüfung unterzogen werden.\\
\\
\textbf{Analyse des Materials}: An dieser Stelle werden die Quellen gemäß dem beschriebenen Verfahren analysiert und bearbeitet. Die ausgewerteten Textstellen finden sich in einer eigenen Liste im Anhang dieser Arbeit.\\
\\
\textbf{Interpretation in Richtung der Fragestellung}: Die Interpretation stellt das Ergebnis der Untersuchung dar, sprich die Beantwortung der eingangs gestellten Forschungsfragen sowie die Überprüfung der aufgestellten Theorie.

\subsubsection{Kategorien}
Aufgrund der bisherigen Betrachtung erscheinen folgende deduktive Kategorien als sinnvoll:\\
\\
\textbf{Personenkreis}:
\begin{enumerate}
    \item{akademische Medizin: jene Personen, welche ein Medizinstudium abgeschlossen haben}
  \item{medizinisches Personal: jene, welche eine (heute) staatlich anerkannte medizinische Ausbildung durchlaufen haben (Pflegepersonal, Heilpraktiker, etc.)}
  \item{akademische Naturwissenschaften: Personen mit einem naturwissenschaftlichen Studium wie Biologie, Pharmazie, etc.}
  \item{medizinische Laien: Menschen, welche keine medizinische Ausbildung durchlaufen haben.}
\end{enumerate}

\pagebreak

\textbf{Gruppen}:
 \begin{enumerate}
   \item{\textbf{Gegner} lehnen Impfungen von Grund auf ab und argumentieren mit Alternativmedizin, anthroposophisch, esoterisch, pseudowissenschaftlich bis hin zu dogmatisch-verschwörungstheoretisch.}
  \item{\textbf{Skeptiker} lehnen Impfungen nicht grundsätzlich ab, sondern meist nur Einzelaspekte, wie den vorgegebenen Zeitpunkt.}
  \item{\textbf{Befürworter} erachten Schutzimpfungen grundsätzlich als sinnvoll.}
 \end{enumerate}

\subsection{Auswahlkriterium der Argumente}
Zur Einteilung der Argumente stellt sich zunächst die sehr schwere Frage, was ist eigentlich ein Argument? Das Wort selbst stammt vom lateinischen \textit{argumentum} ab und bedeutet etwas beweisen, erhellen.\footnote{Argument: \url{http://www.duden.de/rechtschreibung/Argument} 26.4.2016.} Allgemein gesprochen ist ein Argument eine Summe von mehreren Aussagen, welche eine oder mehrere Annahmen enthält, aus welcher mindestens eine Schlussfolgerung gezogen wird. Sinn oder Ziel einer Argumentation ist es, den Adressaten von der Wahrheit oder Falschheit einer Aussage, eines Gegenstandes zu überzeugen. Im Idealfall wird die Argumentation beim Gegenüber zu einer permanenten Einstellungsänderung führen.\footnote{Argument: \url{http://www.ejka.org/de/content/wie-ist-eine-gute-argumentation-aufgebaut} 26.4.2016.} \\
Sprachwissenschaftlich gesprochen besteht ein gutes Argument immer aus drei Teilen:

\textit{Behauptung}: Beinhaltet eine Aussage und besteht meist aus einem Satz. Beispiel: Impfungen sind grundsätzlich schlecht/gut.
\\
\textit{Begründung}: Kann mehrere Sätze lang sein und soll durch Ausführungen die Behauptung belegen. Sie beinhaltet meist Wörter wie "`weil, da, zumal, ..."'. Beispiel: Weil einige Bestandteile des Impfstoffes schädlich für den menschlichen Körper sind. Weil die Krankheit harmlos ist und es damit nicht nötig ist, das Risiko von Impfschäden in Kauf zu nehmen. Eine durchgemachte Krankheit stärkt zudem das Immunsystem auf natürlichem Weg.
\\
\textit{Beispiel}: Dabei sollten persönliche Erlebnisse, wissenschaftliche Studien oder sonstiges die vorangegangene Argumentationskette verständlich machen. Beispiel: Ich habe diese Krankheit selbst erlebt und gut überstanden, auch ohne Impfung. Eine Studie belegt, dass das Risiko von Impfschäden höher als der Nutzen derselben ist. Es gibt keinen Beweis für die Wirkung einer Impfung. Studien belegen, dass die SSPE-Rate seit der Einführung der Impfung gesunken ist. Die Tetanusimpfung hat unzähligen Soldaten des Ersten Weltkrieges das Leben gerettet.
\\
In der Realität bestehen die meisten Argumente jedoch nur aus ein oder zwei Teilen dieser "`3B-Regel"' oder überhaupt nur aus Annahme und Schlussfolgerung.\\
Eine etwas ältere Definition für ein Argument findet sich in Johann Heinrich Zedlers \textit{"`Grosses vollständiges Universal-Lexicon aller Wissenschafften und Künste"'}  (1731--1754), welches auch als digitale Onlineversion vorliegt. Darin findet sich folgende Erklärung: \textit{"`Argumentum heist überhaupt ein Beweiß-Grund, in der Rede-Kunst: aber eine wahre oder doch wahrscheinliche Findung, einen von etwas zu überreden."'}\footnote{Zedlers Universallexikon, Band 2: \url{https://www.zedler-lexikon.de/index.html?c=blaettern&seitenzahl=710&bandnummer=02&view=150&l=de} 27.12.2016.}\\
Zedler nennt zudem verschiedene Arten von Argumenten wie die \textit{Artificialia}, \textit{Commoventia}, oder \textit{Explicantia}. Für den Zweck dieser Arbeit treffen am ehesten Zedlers Definition einer \textit{Personalia} und einer \textit{Persuadentia} zu. Erstere "`\textit{gehen den Redner oder den Zuhörer insonderheit an}"'\footnote{Ebd.}, zweitere "`\textit{sind diejenigen, welche einen Zuhörer vornehmlich zu etwas überreden}"'\footnote{Ebd.} sollen.\\
\\
Betrachtet man diese verschiedenen Definitionen und bedenkt man dazu noch die unterschiedlichen Textarten, welche zur Auswertung herangezogen werden, wird eine einheitliche Definition sichtlich schwierig. Demnach erscheint es sinnvoll, sich an Auswahlkriterien für das herangezogene Material zu orientieren. Diese basieren zunächst darauf, dass ein Argument mindestens aus einer Aussage und einer Schlussfolgerung besteht und den Leser von einer fremden Meinung überzeugen soll. Da es sich beim behandelten Thema um eine kontrovers geführte Diskussion handelt, liegt die Motivation der einzelnen Autoren in der Regel darin, den Adressaten davon zu überzeugen, dass die Impfung gut oder schlecht ist. Da es sich bei den Texten um ganze Monographien handelt, ist es sehr unwahrscheinlich, viele Argumente zu finden, welche im Sinne der 3B-Regel \textit{Beweisführung}, \textit{Begründung} oder \textit{Beispiel} als Gesamtargument anführen oder jeweils mit einer Aussage, wie "`die Impfung ist gut oder schlecht"' beginnen. Vielmehr ist es so, dass Autoren im Sinne eines Fließtextes eine ganze Reihe von Aussagen und Schlussfolgerungen in den Gesamttext einflechten. Die Kernaussage, welche demnach durch den Text an die Leser herangetragen wird, ist: "`Die Impfung (unabhängig welche) ist gut, weil"' respektive "`die Impfung ist schlecht, weil"'. Betrachtet man diese Kernaussage des gesamten Werkes bereits als Teil des Argumentes, kann man die ausgewählten Textstellen als "`Schlussfolgerung"' im Sinne eines Argumentes und eine Antwort auf die Kernaussage betrachten. Im Idealfall enthalten die gewählten Textstellen zudem noch zwei oder alle drei Teile der 3B-Regel. Liest man also die einzelnen Argumente in der angehängten Liste, muss jedes dieser Argumente als Schlussfolgerung betrachtet und gedanklich die genannte Kernaussage vorangestellt werden.

 \subsection{Der historische Vergleich}
Unter "`historischem Vergleich"' versteht man die Gegenüberstellung von zwei, drei oder mehreren Forschungseinheiten, um sie auf deren Gemeinsamkeiten oder Unterschiede hin zu untersuchen. So ein Vergleich kann synchron oder diachron sein, sprich es werden die ausgewählten Sachverhalte entweder aus einer Epoche oder aber über unterschiedliche Zeiträume hinweg betrachtet. Ebenso kann er symmetrisch oder asymmetrisch sein, also die historischen Einheiten mit gleicher Intensität betrachten, oder aber einen Fall ins Zentrum stellen und auf den Anderen nur einen kurzen Blick werfen.\footnote{Hartmut Kaelble, Historischer Vergleich, Version: 1.0, in: Docupedia-Zeitgeschichte, 14.8.2012, \url{http://docupedia.de/zg/Historischer_Vergleich?oldid=125457} 16.7.2015.}
Die Forschungspraxis des historischen Vergleiches, noch im 19. Jahrhundert von Historikern skeptisch betrachtet, erfreute sich in den letzten Jahren einer immer stärkeren Beliebtheit. Dies verwundert kaum, wenn man bedenkt, dass seine besondere Stärke gerade darin liegt, den Forscher dazu zu zwingen, die eigene Position und Fragestellung selbstreflexiv zu betrachten und zu relativieren. Er zeigt uns zudem mögliche Alternativen zu bislang als selbstverständlich oder herausragend betrachteten Entwicklungswegen auf.\footnote{Jakob Hort, Vergleichen, Verflechten, Verwirren. Vom Nutzen und Nachteil der Methodendiskussion in der wissenschaftlichen Praxis: ein Erfahrungsbericht, in: Agnes Arndt u. a. (Hg.), Vergleichen verflechten, verwirren? Europäische  Geschichtsschreibung zwischen Theorie und Praxis, Göttingen, 2011, S. 319--341, hier S. 324.}  Als Besonderheit dieser Methode muss beachtet werden, dass bei einem Vergleich kein großes Phänomen in seiner ganzen, komplexen Totalität Beachtung finden kann, sondern dass der Untersuchungsgegenstand einer gewissen Selektion unterzogen werden muss.\footnote{Heinz-Gerhard Haupt, Jürgen Kocka, Historischer Vergleich: Methoden, Aufgaben, Probleme. Eine Einleitung, in: Heinz-Gerhard Haupt, Jürgen Kocka (Hrg.), Geschichte und Vergleich. Ansätze und Ergebnisse international vergleichender Geschichtsschreibung, Frankfurt/Main, 1996. S. 9--46, hier S. 23.}\\
So soll konkreter Gegenstand dieser Arbeit die Impfdebatte sein. Es geht damit, wie bereits dargelegt, nicht um die Frage nach Wirksamkeit oder Sinnhaftigkeit von Impfung, sondern tatsächlich um die Argumente der Diskussion. Da, wie erwähnt ein relativ breiter Zeitraum zur Bearbeitung herangezogen wird, nämlich das 19. Jahrhundert mit dem 20. und 21. Jahrhundert, handelt es sich um einen diachronen Vergleich. Außerdem soll versucht werden, beiden Teilen der Debatte - Pro und Contra - gleichwertig, also symmetrisch zu begegnen. Da es auf Grund der zu erwartenden Menge an Argumenten nicht möglich ist, diese direkt einander gegenüberzustellen, dienen die mit Hilfe der Qualitativen Inhaltsanalyse gewonnen Kategorien als Vergleichswert.


\section{Historischer Kontext}
Befasst man sich mit einem Thema wie der Impfdebatte, muss zunächst der historische Kontext abgesteckt und erläutert werden. Natürlich kann an dieser Stelle keine vollständige Medizingeschichte vom 18. Jahrhundert bis zur Gegenwart, inklusive der problematischen Rolle der Medizin im Nationalsozialismus, beschrieben werden. Dies würde den Rahmen dieser Arbeit bei weitem sprengen. Entsprechend werden nur jene medizinhistorischen Abschnitte in Auszügen beschrieben, welche für die Geschichte des Impfens wesentlich sind, wie etwa die Entwicklung der Bakteriologie oder der Serologie. Einige Aspekte, wie die Veränderungen der Bedeutung von Krankheiten oder die Entwicklung der Hygiene, sind für die Impfgeschichte grundsätzlich relevant, können jedoch nur am Rande erwähnt werden. Daneben muss festgehalten werden, dass bei den hier herausgegriffenen Aspekten kein Anspruch auf Vollständigkeit gestellt wird.

\subsection{Ein englischer Landarzt}
Ein Name, welcher in Verbindung mit dem Impfthema sowohl im Negativen wie auch im Positiven auftaucht und den vermutlich jeder und jede kennt, der sich mit dem Thema beschäftigt, ist \textit{Edward Jenner}.
Geboren wurde er am 17. Mai 1749 in Berkely, Gloucestershire, als achtes von neun Kindern.\footnote{Edward Jenner:  Enxyclopaedia Britannica: \url{http://www.britannica.com/biography/Edward-Jenner} 8.3.2016.} Nach dem Schulbesuch ging er mit 13 Jahren bei einem Chirurgen in die Lehre. Mit 21, nach Beendigung dieser Ausbildung, zog er nach London, wo er am St. Georg's Hospital Schüler von John Hunter\footnote{Englischer Arzt und Chirurg, gilt als Begründer der Pathologie und der experimentellen Chirurgie sowie als Vorreiter auf dem Gebiet der vergleichenden Anatomie. Vgl.: Calixte Hudemann-Simon, Die Eroberung der Gesundheit 1750--1900, Frankfurt, 2000, S. 14.} (1728--1793) wurde.\footnote{Edward Jenner, Enxyclopaedia Britannica: \url{http://www.britannica.com/biography/Edward-Jenner} 8.3.2016.}\\
1773, mit 23 Jahren, kehrte er in seine Heimatstadt zurück, wo er eine Praxis eröffnete und sich als Landarzt etablierte. Später kamen noch zwei weitere Praxen in London und Cheltenham dazu.\footnote{Edward Jenner: \url{http://www.bbc.co.uk/history/historic_figures/jenner_edward.shtml} 11.9.2017.}\\
1788 ehelichte er Catherine Kingscote. Aus dieser Verbindung gingen drei Kinder hervor: 1789 der älteste Sohn Edward, den er als eines der ersten Kleinkinder mit Kuhpocken impfte. 1810 starb dieser möglicherweise an Tuberkulose.\footnote{An vielen Stellen wird behauptet, dass Edward an einem Hirnschaden oder "`impfbedingten Schwachsinn"', ausgelöst durch die Pockenimpfung starb. Diese Autoren sehen in Jenners Sohn den ersten Impfschadensfall der Geschichte. Vgl.: Friedrich Graf, Die Impfentscheidung. Ansichten, Überlegungen und Informationen -- vor jeglicher Ausführung!, 5. Auflage, Ascheberg, 2013, S. 23 od.: Geschichte der Pockenimpfung von 1713 bis 1977: \url{http://www.impf-alternative.de/2011/01/350/} 14.3.2017.} 1794 wurde die Tochter Catherine und 1797 der zweite Sohn Robert Fitzhardinge geboren. Jenners Frau Catherine starb 1815, ebenfalls an Tuberkulose. Edward Jenner verschied am 26. Jänner 1823 vermutlich an einem Schlaganfall, im Alter von 73 Jahren.\footnote{Edward Jenner: \url{http://www.bbc.co.uk/history/historic\_figures/jenner\_edward.shtml} 11.9.2017.}\\
Jenner war in jedem Fall "`ein Kind seiner Zeit"' und entsprechend vielseitig begabt. Er musizierte, schrieb Verse, befasste sich mit Naturwissenschaften, stellte Beobachtungen an und sammelte Proben für John Hunter, mit dem er bis zu dessen Tod 1793 in Kontakt blieb.\footnote{Edward Jenner, Enxyclopaedia Britannica: \url{http://www.britannica.com/biography/Edward-Jenner} 8.3.2016.}\\

\subsection{Von der Kuhpockenvaccination zur allgemeinen Impfpflicht}
Im Jahr 1798 publizierte Edward Jenner die Ergebnisse seiner vorangegangenen Experimente über die Wirkung der Kuhpocken auf den Menschen: \textit{An Inquiry into the Causes and Effects of Variolae Vaccinae, a Disease Discovered in some of the Western Counties of England, Particularly Gloucestershire, and Known by the Name of The Cow Pox}.\\
\\
Die Pocken oder Blattern, eine schwere und hoch ansteckende Infektionskrankheit, begleiteten die Menschen seit jeher. Im 18. Jahrhundert gehörte diese Krankheit, obzwar gefürchtet, zum alltäglichen Leben dazu: "`\textit{Wir hören von hunderten, von tausenden, die in verschiedenen Städten an dieser Krankheit liegen, und denken darüber weg, als über ein Unglück, das sich nicht ändern läßt [...]}\footnote{Walker, Untersuchung der Pocken, S. IV.}."'
 Heute wird angenommen, dass sich durchschnittlich acht von zehn Personen ansteckten und einer unter sieben starb. Unter Kleinkindern zählten die Pocken zu den Haupttodesursachen.\footnote{Hudemann-Simon,
Eroberung der Gesundheit, S. 199.} All jene, welche die Pocken überlebten, mussten mit schweren Folgeerkrankungen, von Narben bis hin zu körperlichen oder geistigen Behinderungen, rechnen. Vielen unbekannt ist vermutlich, dass der Landarzt Edward Jenner seine Patienten bereits mit Hilfe der \textit{Variolation} vor diesem Schicksal zu schützen suchte. Bei der Variolation injizierte man einem gesunden Menschen das Pockensekret eines Erkrankten mit leichterem Verlauf. Der Betroffene erlitt im Idealfall einen milden Krankheitsausbruch und war dadurch für kommende Pockeninfektionen immunisiert. Wo der Ursprung der Methode liegt, kann nicht genau bestimmt werden, sicher ist jedoch, dass Lady Mary Wortley Montagu (1689--1762), die Frau des britischen Botschafters im Osmanischen Reich, die Variolation in England populär machte.\footnote{Eckart, Jenner, S. 1--2.} Obwohl in England ebenso wie in Europa praktiziert, konnte sich die Variolation nie ganz durchsetzten, blieb sie doch stets riskant: Es bestand immer die Gefahr eines schwereren Verlaufes und oder gar einer Epidemie.\footnote{Eckart, Jenner, S. 3.}\\
In einigen ländlichen Regionen Europas haben die Menschen jedoch beobachtet, dass Personen, welche einmal die Kuhpocken überstanden hatten, nicht mehr oder nur leicht an den Menschenpocken erkrankten. Obwohl gerne behauptet, war Edward Jenner damit keineswegs der erste, der um die Schutzwirkung der leichteren Kuhpocken beim Menschen wusste, dies gehörte viel eher zum "`bäuerlichen Erfahrungswissen"'. Für den deutschsprachigen Raum etwa ist für die 1760er Jahre die bewusste Übertragung der Kuhpocken auf den Menschen zum Schutz vor drohenden Pockenepidemien nachweisbar.\footnote{Eckart, Jenner, S. 4.} Vergleichbare Handhabungen sind auch in England belegt, etwa durch den Wundarzt Nosh\footnote{Der vollständige Name sowie Lebensdaten sind unbekannt. Es wird jedoch erwähnt, dass er ein Buch über seine Erfahrungen verfasste, welches auf Grund seines Todes 1786 nicht veröffentlicht wurde. Vgl.: A. K. Hesselbach (Hrg.), Bibliothek der deutschen Medicin und Chirugie, Würzburg, 1820, S. 2.}, der 1781 gezielt damit begann, Kinder mit den Kuhpocken zu infizieren, um sie für die Blattern zu immunisieren.\footnote{Eckart, Jenner, S. 4; Vgl.: Georg Friedrich Krauss, Die Schutzpockenimpfung in ihrer endlichen Entscheidung, als Angelegenheit des Staats, der Familien und des Einzelnen, Nürnberg, 1820, S. 223.} \\
Man kann an dieser Stelle also festhalten: Das Revolutionäre an Jenners Leistung war weniger die Erkenntnis um die Wirkung der Kuhpocken auf den Menschen, als vielmehr sein Vorgehen, mittels einer Versuchsreihe und Experimenten diese These unter Beweis zu stellen und seine Ergebnisse zu veröffentlichen. Die wichtigste Erkenntnis dabei war, dass die Krankheit nicht nur von Tier auf Mensch, sondern auch von Mensch zu Mensch übertragbar war und zwar ausschließlich durch eine absichtliche Ansteckung.\footnote{Details über den Verlauf der Experimente nachzulesen bei Eckart, Jenner.} Bei seinen Versuchen direkt am Menschen infizierte er die Probanden zunächst mit Kuhpocken und nach einigen Wochen mit echten Menschenpocken, um die Wirksamkeit der Methode auszutesten. Gemäß Wolfgang Eckart wurde dieses Vorgehen von den Zeitgenossen bereits als ethisch problematisch angesehen.\footnote{Eckart, Jenner, S. 4. Eckart führt hier jedoch nicht näher aus, was unter Jenners Zeitgenossen als ethisch problematisch verstanden wurde, möglicherweise die Versuche direkt am Menschen. Denn die Einpfropfung der Menschenpocken, also die Variolation, war ein gängiges Verfahren zur Pockenprävention. Auch bei späteren Versuchen in Österreich wurden etwa den Findelkindern zuerst die Kuhpocken und nach einiger Zeit die Menschenpocken eingeimpft, um die Wirksamkeit zu bestätigen.} Nichtsdestotrotz verbreitete sich die Kuhpockenimpfung verhältnismäßig rasch nach der Publikation, ermöglichte diese neue Methode immerhin, zumindest bei korrekter Ausführung, eine relativ sichere, aktive und vor allem präventive Maßnahme gegen eine Krankheit, der man bis dahin mehr oder minder hilflos ausgeliefert war. Aber nicht nur Ärzte nahmen diese neue, vielversprechende Methode auf, sondern auch Staatsoberhäupter und Behörden. So traf die Vaccination auch im Habsburgerreich auf große Begeisterung und 1799 wurde in Niederösterreich die erste Kuhpockenimpfung außerhalb Englands durch den Protomedikus Pascal Joseph Ferro (1753--1809) vorgenommen. Ferro impfte dazu seine eigenen drei Kinder im Alter von 19 Monaten bis vier Jahren.\footnote{Heinz Flamm, Christian Vutuc, Geschichte der Pocken-Bekämpfung in Österreich, in: Wiener klinische Wochenschrift, Heft 122, Wien, 2010, S. 265--275, hier S. 266.} Er reichte den von seinen Kindern gewonnenen Impfstoff an den Mediziner Johann de Carro (1770--1857) weiter, welcher damit erfolgreich seine Kinder vaccinierte. Wenige Monate später vollzogen beide die Gegenprobe, indem sie den Kindern echte Menschenpocken inoculierten, ohne dass diese zum Ausbruch kamen. De Carro wurde einer der ersten und eifrigsten Verfechter und Verbreiter der Kuhpockenimpfung innerhalb des österreichischen Kaiserreiches.\footnote{Gustav Paul, Die Entwicklung der Schutzpockenimpfung in Österreich, Wien, 1901, S. 1.} Die erste erfolgreiche Schutzpockenimfpung im Erzstift Salzburg wurde nach eigenen Angaben im Mai 1801 von Dr. Joseph d'Outrepont (1776--1845) durchgeführt.\footnote{Joseph d'Outrepont, Belehrung des Landvolkes über die Schutzblattern. Nebst einem kurzen Unterrichte über die Impfung derselben für die Wundärzte, Salzburg, 1803, S. 25. D'Outrepont lernte in Wien bei de Carro die Methode der Kuhpockenimpfung und brachte diese nach Salzburg, wo er am 19. Mai 1801 die erste Schutzimpfung durchführte. Vgl.: Alfred Stefan Weiss, Joseph Servatius von d'Outrepont (1776--1845). Zum 150. Todestag eines bedeutenden Salzburger und Würzburger Arztes, in: Salzburg Archiv 20 (1995), S. 169--184, hier S. 173.}\\
Wie rasch sich die Kuhpockenimpfung in der Welt verbreitete, belegen die allerorts zeitnah eingeführten Impfgesetze, welche bei Strafe eine verpflichtende Impfung der jeweiligen Bevölkerung vorsahen. Als eine der ersten Regierungen in Europa erließ das Königreich Bayern ein solches Gesetz 1807. Russland folgte 1812, Großbritannien 1853 und das Deutsche Reich 1874, unmittelbar nach dessen Gründung.\footnote{Eckart, Jenner, S. 12. Nittinger berichtet über etwas andere Jahreszahlen: 1800 in Preußen und Frankreich, 1801 englische Marine, 1806 in Bayern, 1809 in Baden, 1809 englisches National Vaccine Stablissement, 1810 in Dänemark, 1811 in Holstein, Schweden und Norwegen, 1818 in Würtenberg, 1821 in Hannover. Vgl.: Carl Georg Gottlob Nittinger, Gott und Abgott oder die Impfhexe, Stuttgart, 1863, S. 5.}\\
Innerhalb der österreichischen Monarchie war es die niederösterreichische Landesregierung, welche 1802 per Dekret die Kuhpockenimpfung als sicheres und unschädliches Schutzmittel einstufte. Von da aus verbreitete sie sich rasch innerhalb der übrigen Kronländer.\footnote{Paul, Schutzpockenimpfung in Österreich, S. 7.} Da die Impfung nicht in allen Teilen der Bevölkerung Anklang fand, kam es bald zu einem indirekten Impfzwang seitens der Regierung, indem man etwa die Aufnahme in öffentliche Lehranstalten, ebenso wie in Waisenhäuser oder die Gewährung von Stipendien von einem Impfnachweis abhängig machte. Dies ging sogar soweit, dass mit einem Hofkanzleidekret vom Februar 1811 bestimmt wurde, dass an Blattern verstorbene Kinder zwar von einem Priester eingesegnet werden durften, jedoch ohne jegliche Begleitung durch Eltern oder Familie bestattet werden mussten. Wie strikt diese Regelung tatsächlich umgesetzt wurde, bleibt jedoch fraglich. Eine andere Verordnung sah vor, dass Eltern, deren ungeimpfte Kinder erkrankten oder in Folge der Pocken verstarben, in der lokalen Zeitung namentlich erwähnt wurden.\footnote{Paul, Schutzpockenimpfung in Österreich, S. 4.}\\

Dies erschien den zuständigen Behörden offenbar, in Anbetracht des Widerstandes der Bevölkerung, als angebracht. Denn vor allem im ländlichen Raum stand man der neuen Maßnahme durchaus skeptisch gegenüber. Dass die Vaccination überwiegend von universitär gebildeten Ärzten durchgeführt wurde, denen man ohnehin mit Misstrauen begegnete, tat ihr übriges. 1836 wurde schließlich eine umfassende "`Vorschrift über die Kuhpocken-Impfung in den kaiserl. königl. Staaten"' erlassen. Ergänzt wurde dieser Erlass 1840 um eine Notimpfung aller Ungeimpften beziehungsweise einer Impfauffrischung im Falle einer Epidemie.\footnote{Flamm, Vutuc, Geschichte der Pocken-Bekämpfung in Österreich, S. 269.} In dieser ersten "`Impfempfehlung"' war die allgemeine Vaccination von Kindern ab der 8. Lebenswoche, in jedem Fall binnen des ersten Lebensjahres, vorgesehen.\footnote{Ingomar Mutz, Diether Spork, Geschichte der Impfempfehlungen in Österreich, S. 94, in: Wiener Medizinische Wochenschrift, Heft 157/5, Wien, 2007, S. 94--97.} Betont werden muss hier, dass es sich bei diesem "`Impfregulative"' um \textit{keine} Impfpflicht handelte, wie sie in anderen Ländern eingeführt wurde. Dieses Hofkanzleidekret von 1836 sah, neben dem mittelbaren Zwang, lediglich bestimmte Maßnahmen zur Belehrung der Bevölkerung vor. So sollten etwa Hebammen Mütter nach der Geburt über die Pockenimpfung aufklären oder bei der Taufe entsprechende Informationsblätter an die Eltern verteilt, respektive verlesen, werden.\footnote{Marius Kaiser, Pocken und Pockenschutzimpfung. Ein Leitfaden für Amtsärzte, Impfärzte und Studierende der Medizin, Wien, 1949, S. 198.} Einzig für die Aufnahme in die k.k. Armee bestand seit 1886 eine Impfpflicht.\footnote{Flamm, Vutuc, Geschichte der Pocken-Bekämpfung in Österreich, S. 270.} Erst mit 14. Juli 1939 kam es zur Verordnung der "`Einführung reichsrechtlicher Vorschriften zur Bekämpfung übertragbarer Krankheiten in der Ostmark"', was bedeutete, dass das deutsche Reichsimpfgesetz von 1874 nun auch auf Österreich ausgedehnt und somit eine allgemeine Pocken-Impfpflicht eingeführt wurde.\footnote{Kaiser, Pocken und Pockenschutzimpfung, S. 199.}\\
\\
An dieser Stelle lohnt es sich, einen genaueren Blick auf diese Gesetzesentwürfe zu werfen, können hier doch einige historisch bedeutende Entwicklungen ausgemacht werden. Damit wurden Prozesse bedingt, welche auf den ersten Blick nicht von Bedeutung erscheinen, tatsächlich aber Auswirkungen bis in unsere Zeit haben und zwar im Sinne der Medizin als \textit{Public Health}. Denn die Pflichtimpfung gilt als eine der ersten Maßnahmen, die in einem bisher unbekannten Ausmaß die gesamte Bevölkerung mit den universitären Ärzten in Berührung brachte. Weder davor, noch danach gab es ein vergleichbares Gesetz, mit welchem der Staat seiner Bevölkerung eine medizinische Maßnahme verpflichtend aufoktroyierte.\footnote{Eckart, Jenner, S. 19.} Diese staatliche "`Zwangsbeglückung"' kann als weiterer großer Schritt dahingehend gesehen werden, dass der Staat die Medizin "`in seinen Dienst nahm."' Dies fußte auf der Idee des Aufgeklärten Absolutismus des 18. Jahrhunderts, in welchem der Fürst, als Diener des Staates, durch sein Handeln die Wohlfahrt, Macht und Stärke seines Herrschaftsgebietes vermehren und sichern sollte. Dazu gehörte es nunmehr auch, dass sich der Herrscher um den Gesundheitszustand seiner Bevölkerung sorgte und staatliche Maßnahmen zu dessen Erhalt oder Wiederherstellung traf. Es ging in erster Linie darum, den unteren Bevölkerungsschichten, die sich zunehmend in Manufakturen
und städtischen Ballungszentren befanden, den Zugang zur medizinischen Versorgung zu ermöglichen. In Zeiten des Merkantilismus geschah dies freilich weniger aus
karitativen Gesichtspunkten, sondern vielmehr aus ökonomischen Überlegungen.\footnote{Wolfgang Uwe Eckart, Geschichte der Medizin, 2. Auflage, Berlin, 1994, S. 201--203.} Dadurch kam es zum einen zu der bereits oben erläuterten Medikalisierung der Patienten. Andererseits bedingte dies unter anderem eine Veränderung des ärztlichen Berufsbildes im Sinne einer Professionalisierung insgesamt, einer verbesserten sozialen Stellung und eines wachsenden Standesbewusstseins.\footnote{Eckart, Geschichte der Medizin, S. 212.} Des Weiteren gelten diese frühen Ideen einer öffentlichen Gesundheitspflege, damit zusammenhängende medizinalpolizeiliche Vorstellungen des 18. Jahrhunderts sowie die Ideen der diätetisch-physikalischen Chemie als Grundlage für die Entwicklung einer wissenschaftlichen Hygiene, welche vor allem im 19. Jahrhundert zu entscheidenden Entdeckungen führte (Asepsis, Antisepsis u. a.).\footnote{Eckart, Geschichte der Medizin, S. 231.}\\
\\
Dennoch wurde diese Maßnahme der Pflichtimpfung bereits von den Zeitgenossen äußerst kontrovers aufgenommen. Dies bezog sich einerseits auf die Impfung als neue medizinische Maßnahme, andererseits auf den mittels Gesetz ausgeübten Zwang.\footnote{Eckart, Jenner, S. 19.} Die von Impfgegnern vorgebrachten Ängste standen keineswegs unberechtigt im Raum. Zum einen war noch nicht geklärt, wie der Prozess der Immunisierung durch die Schutzimpfung eigentlich funktionierte, zum anderen dauerte es nicht lange, bis es zu ersten Komplikationen im Zuge der Vaccination kam. Diese wurden zum Beispiel herbeigeführt durch Personen, die bei der Impfung nicht exakt dem vorgesehenen Prozedere folgten. Weiters war es nicht leicht, das Serum zu gewinnen oder zu transportierten. So kam es bei einem achtlosen Umgang mit dem Impfstoff leicht zu Verunreinigungen und sogar zu Vermischungen der Kuhpockenviren mit den Menschenpockenviren. Ebenso wurde zu Beginn direkt von Mensch zu Mensch geimpft, was bedingt durch mangelnde Hygiene zu  Krankheitsübertragungen zum Beispiel von Syphilis führte. Das Problem konnte erst damit behoben werden, als man begann, ausschließlich von Kuh zu Mensch zu impfen.\footnote{Edward Jenner, Enxyclopaedia Britannica: \url{http://www.britannica.com/biography/Edward-Jenner} 8.3.2016.}

\newpage

\subsection{Von der Zellularpathologie zu den Antitoxinen}
Im 19. Jahrhundert kam es zu grundlegenden gesellschaftlichen Veränderungen. Die technisch-industrielle Revolution und die
allmähliche Entstehung von mechanischen Produktionsprozessen lockte die durch diverse Krisen in Not geratene
Landbevölkerung mehr und mehr in die Stadt. Dies führte zusehends zur Entstehung so genannter Ballungszentren. Die dort herrschenden
dramatischen Zustände, bedingt durch Überbevölkerung, Hungersnöte und hygienische Missstände, begünstigten die
epidemische Verbreitung von Infektionskrankheiten mit hohen Todesraten: Diphtherie, Tuberkulose, Typhus und natürlich an oberster Stelle die Cholera. Dies führte einerseits zu den frühen \textit{Sanitary Movements}\footnote{Hygienebewegung, zunächst ausgehend von England und den Ideen des Rechtsanwalts Edwin Chadwick (1800--1890). Epidemische Krankheitsausbrüche in Großstädten waren ein wesentlicher Ansporn für öffentliche Gesundheitsmaßnahmen durch staatliche Stellen. Es folgte die schrittweise Institutionalisierung des öffentlichen Gesundheitswesen, wie bereits angedeutet. Vgl.: Roy Porter, Die Kunst des Heilens. Eine medizinische Geschichte der Menschheit von der Antike bis heute, Berlin, 2000, S. 413--420 oder
History of public health: \url{http://priory.com/history_of_medicine/public_health.htm} 15.5.2016.}, andererseits aber auch zur zunehmenden naturwissenschaftlichen Erforschung von den Ursachen der Seuchen, woraus sich wiederum die Bakteriologie entwickelte.\\

Den Beginn der Bakteriologie, die prägende medizinische Leitidee des 19. Jahrhunderts, markiert wohl Rudolf Virchow (1821--1902). 1858 veröffentlichte er seine revolutionäre Forschung, in der er eine neue Krankheitslehre begründete, nach der alle Krankheiten des Organismus auf die Veränderung einer Körperzelle zurückgeführt werden können. Grundgedanke der Bakteriologen war die nicht ganz neue Idee, dass Krankheiten durch Mikroorganismen hervorgerufen wurden und nicht etwa spontan entstanden.\footnote{Porter, Die Kunst des Heilens, S. 431.} Dieser Lehrsatz, der im Grunde bis heute gültig ist, löste unter anderem die noch immer vorhandene neohumoralistischen Auffassungen\footnote{Humoralpathologie: Vier-Säfte-Lehre, ein in der Antike entwickeltes medizinisches Konzept zur Erklärung allgemeiner Vorgänge im
Körper und Krankheitserscheinungen.} des Wiener Pathologen Carl von Rokitansky (1804--1878) ab.\\
Der Chemiker Louis Pasteur (1822--1895) und der Mediziner Robert Koch (1843--1910) gelten wohl als die berühmtesten Vertreter der neuen Wissenschaft. Im Jahr 1878 stellte Pasteur an der französischen \textit{Academie de Medicine} offiziell seine Infektionstheorie vor: Er war, gemeinsam mit zwei weiteren Forschern zu der Überzeugung gekommen, dass Mikroorganismen Gärung, Fäulnis und Krankheiten auslösten, dass jeweils ein Organismus eine Krankheit auslöse und man dieser durch einen entsprechenden Impfstoff (analog zu den Pocken) vorbeugen konnte. Bereits 1879 stellte er diese Behauptung mittels der Untersuchungen der Geflügelcholera und des Milzbrandes erfolgreich unter Beweis.\footnote{Porter, Die Kunst des Heilens, S. 437.}\\
Pasteurs deutschem Kontrahenten Robert Koch gelang der Nachweis dafür, dass jede Krankheit einem eigenen Erreger zuzuordnen ist.\footnote{Hudemann-Simon, Eroberung der Gesundheit, S. 22--23.} Basierend auf seinen Forschungen entwickelte er folgendes, bis heute gültiges, Erreger-Postulat. Dieses besagt, dass ein Erreger zunächst mikroskopisch nachweisbar und identifizierbar sein muss. In weiterer Folge ist es möglich, den Erreger zu isolieren und in Reinkultur nachzuzüchten. Überimpft man diese Reinkultur auf ein Versuchsobjekt, ruft dies die Grundkrankheit mit identen Krankheitszeichen hervor. Der Erreger kann dann im überimpften Organismus erneut auf diese Weise nachgewiesen werden.\footnote{Eckart, Geschichte der Medizin, S. 236.}


Von da an folgten die bakteriologischen Entdeckungen Zug um Zug: 1879 die Entdeckung des Erregers für Wundfieber, 1882 jener für Tuberkulose. 1883 wies Koch das Cholerabakterium im menschlichen Darm nach und belegte, dass dieses hauptsächlich in verschmutztem Wasser lebte. 1885 erfolgte die erste erfolgreiche Tollwutimpfung, der nächste Meilenstein, der zum Verständnis über die Wirkung von Impfungen beitrug.\footnote{Hudemann-Simon, Eroberung der Gesundheit, S. 22--23.}\\
Diese Welle an kurz aufeinanderfolgenden Entdeckungen schürte die Hoffnung auf eine ebenso rasche Entwicklung an Heilmitteln, schien die Angelegenheit dank der neuen Methoden doch geradezu einfach zu sein: Isolation des krankheitserregenden Mikroorganismus und Entwicklung eines dazu passenden Impfstoffes. Doch neben den zahlreichen Erfolgen gab es auch dramatische Rückschläge, wie etwa das 1890 von Koch entwickelte Tuberkulin. Ohne das Mittel auf tatsächliche Wirksamkeit oder Sicherheit zu prüfen, wurde es in der Öffentlichkeit freudig begrüßt und binnen eines Jahres erhielten tausende Menschen eine Behandlung damit. Die Enttäuschung war entsprechend groß, als man feststellte, dass es nicht das erhoffte Wundermittel war, sondern im Gegenteil für Patienten mit Lungentuberkulose sogar gefährlich sein konnte.\footnote{Porter, Die Kunst des Heilens, S. 441--445.}\\
\\
Als Beginn der Ära der Serumtherapie gilt das von Emil von Behring (1854--1917), Shibasaburo Kitasato (1852--1931) und
dessen Mitarbeitern entwickelte Mittel gegen Diphtherie und in weiterer Folge gegen Tetanus. Behrings Ausgangsidee war,
dass es gelingen müsste, die Erreger von Infektionskrankheiten mit Gegengiften (Antitoxien) zu bekämpfen, die der Körper
im Rahmen der Abwehrreaktion selbst herstellt.\footnote{Wolfgang Uwe Eckart, Der Beginn der Ära von Serumtherapie und Impfung,
in: Ärzte Zeitung, Heft 48, 2004, S. 19.} Die Diphtherie, der "`Würgeengel der Kinder"', zählte im 19. Jahrhundert zu den Haupttodesursachen
bei Kindern, denn sie verlief fast immer infaust und trat, vor allem in Großstädten, oft epidemisch auf.
Aber auch die Tetanusinfektion galt mit einer Todesrate von rund 40 Prozent
als extrem gefährlich. Das Diphtherieantitoxin kam erstmals 1891 in einer Berliner Klinik erfolgreich zur Anwendung.
Nach der Einführung des Mittels sank die Todesrate in Folge der Diphtherie radikal.
Die erste aktive Immunisierung durch eine Diphtherieschutzimpfung erfolgte 1913. Ab 1915 trug die in den Armeen eingeführte
Tetanusschutzimpfung massiv zur Reduzierung der Todesfälle durch Wundstarrkrampf bei.\footnote{Porter, Die Kunst des Heilens,
S. 442 u. 447.} Emil von Behring erhielt für seine Arbeit 1901 den ersten Nobelpreis für Medizin.\\
Bemerkenswert ist hier die Entdeckung, dass auch zellfreie Seren, gewonnen aus immunisierten Tieren, virulente Bakterien töteten, was zu der Vermutung führte, dass nicht nur Bakterienzellen selbst, sondern ein von ihnen gebildetes Toxin krankheitserregend ist. Auf dieser Basis trieb man die Forschung voran und in den so genannten "`goldenen Jahren"' der neuen Bakteriologie (von 1879 bis 1900) wurde jährlich mindestens ein Erreger von schweren Erkrankungen identifiziert. Auch eine ganze Reihe von Schlangengiftantitoxinen wurde in dieser Zeit entwickelt. Die Serumtherapie war allerdings nicht unproblematisch: Die Antitoxinproduktion war nicht kontrollierbar
und die Mittel schwankten daher stark in Reinheit und Konzentration. Dazu kam die Serumkrankheit\footnote{Fieber, Ausschlag,
Gelenkschmerzen. Es handelt sich dabei um eine allergische Reaktion auf die injizierten Antigene, eine so genannte
Überempfindlichkeitsreaktion. Vgl.: Serumtherapie: \url{http://www.spektrum.de/lexikon/biologie/serumkrankheit/61229} 19.5.2016.}
als verbreitete Nebenwirkung sowie Todesfälle nach Antitoxintherapien.\footnote{Porter, Die Kunst des Heilens, S. 443.}\\
Andere Krankheiten wie etwa Scharlach erwiesen sich als hartnäckig und ein Teil der entwickelten Impfstoffe, wie jener gegen Scharlach, Tuberkulose oder die Pest, waren nur mäßig bis gar nicht erfolgreich. Behandlungen gegen diese Krankheiten wurden erst im Zuge der Debatte über die körpereigene Immunität und die Entdeckung von Antibiotika generiert. Des Weiteren war die Entwicklung von neuen Impfstoffen technisch mit gewissen Schwierigkeiten verbunden, wie etwa dem gefahrlosen Züchten von Viren im Labor. Erst Anfang des 20. Jahrhunderts fand man mit dem Hühnerei ein steriles und gegen Ansteckung gesichertes Medium, um Viren in großen Mengen züchten zu können.\footnote{John Rowan Wilson, Polio! Die Geschichte eines Impfstoffes, Wien, 1963, S. 52.}

\subsection{Vom Keuchhusten zur Mengigokokkenimpfung}
Die bahnbrechenden Erfolge des 19. Jahrhunderts in der Bakteriologie und der Medizin insgesamt führten wie geschildert zum Rückgang einiger gefürchteter Infektionskrankheiten, die in erster Linie Kinder betrafen. Trotz der weiteren großen Entdeckungen wie die der Chemotherapie oder die des Penicillins, gab es eine ganze Reihe weiterer so genannter "`Kinderkrankheiten"', die die Menschen am Beginn des 20. Jahrhunderts fürchteten.\\
Eine von ihnen war der \textit{Keuchhusten}. Er wurde erstmals im 16. Jahrhundert von dem französischen Arzt Guillaume de Baillou (1538--1616) als eigene Krankheit beschrieben. Der englische Arzt Thomas Sydenham (1624--1689) prägte 1679 den Begriff "`Pertussis"' (aus dem Lateinischen für starker/heftiger Husten). Der Erreger wurde 1906 von den Bakteriologen Jules Bordet (1870--1961) und Octave Gengou (1875--1957) isoliert und zu Ehren seines Entdeckers \textit{Bordetella pertussis} genannt.\footnote{Ulrich Heininger (Hrg.), Pertussis bei Jugendlichen und Erwachsenen, Stuttgart, 2003, S. 1.} 1915 kam erstmals ein Pertussisimpfstoff in den USA zur Zulassung.\footnote{Vaccine Timeline: \url{http://www.immunize.org/timeline/} 20.5.2016.} Da es aber Probleme bei der Verträglichkeit bei erwachsenen Versuchspersonen gab, kam es zu keiner allgemeinen Standardisierung. In Österreich und Deutschland wurde erst in den 1960er Jahren ein Pertussis-Ganzkeimimpfstoff, in Kombination mit Diphtherie-Tetanus, flächendeckend eingesetzt, was zu einem umgehenden Rückgang der Krankheitszahlen führte. Die Wirksamkeit, Sicherheit und Verträglichkeit des Ganzkeim-\\impfstoffes war jedoch international umstritten, was zu einer Einschränkung der Impfungen und der Impfempfehlungen der deutschen STIKO (Ständige Impfkommission am Robert Koch Institut) und des österreichischen OSR (Oberster Sanitätsrat) führte.\footnote{Heininger, Pertussis, S. 24 u. 54.} So wurde die Keuchhustenimpfung in Österreich bis einschließlich 1991 nur in Klammer in der allgemeinen Impfempfehlung angeführt.\footnote{Mutz, Spork, Geschichte der Impfempfehlungen in Österreich, S. 96.} \\
Seit 1996 steht der in Japan entwickelte, so genannte azelluläre Pertussis-Impfstoff zur Verfügung. Dieser enthält nur mehr jenen Bestandteil
des Erregers, welcher eine entsprechende Immunantwort im Impfling hervorruft und damit besser verträglich ist. Pertussis zählt ebenso wie
Masern, Kinderlähmung, Pocken, Tuberkulose, Typhus und viele weitere Infektionskrankheiten gemäß dem Epidemiegesetz zu den meldepflichtigen Krankheiten.\footnote{Gesamte Rechtsvorschrift für das Epidemiegesetz: \url{https://www.ris.bka.gv.at/GeltendeFassung.wxe?Abfrage=Bundesnormen&Gesetzesnummer=10010265} 20.5.2016.} Gemäß aktueller Impfempfehlung in Österreich wird eine Impfung gegen Keuchhusten bereits im Säuglingsalter im Rahmen der Sechsfachimpfung vorgenommen. Seit Beginn des 21. Jahrhunderts ist eine DTP-Impfung auch für Jugendliche und Erwachsene empfohlen.\footnote{Keuchhusten: \url{http://www.reisemed.at/krankheiten/keuchhusten-pertussis} 27.5.2016.} Dennoch kam es seit etwa 2008/2009 zu einem deutlichen Anstieg der Keuchhustenfälle in Österreich, vor allem in der Gruppe der Sieben- bis 15-Jährigen und den 60- bis 80-Jährigen. Als mögliche Gründe dafür werden vom Institut für Spezifische Prophylaxe und Tropenmedizin der Medizinischen Universität Wien sinkende Durchimpfungsraten, bessere Diagnostik sowie das Nachlassen der Antikörper-Konzentration nach fünf bis sechs Jahren nach der Grundimmunisierung genannt.\footnote{Keuchhusten, Standard Online, 17. Jänner 2014: \url{http://derstandard.at/1389857404641/Oesterreich-Keuchhusten-erlebt-Renaissance} 27.5.2016.}\\
\\
Eine andere, gefürchtete Infektionskrankheit war (und ist es in einigen Regionen der Welt noch heute)
die \textit{Poliomyelitis}. Der Virus der Kinderlähmung befällt die Nervenzellen des Rückenmarkes und verursacht Lähmungen, die zum Tod führen können. 1784 beschrieb der englische Arzt Michael Underwood (1737--1820) erstmals das klinische Erscheinungsbild der Krankheit in seinem Werk \textit{"`Treatise of the Diseases of Children"'.} Um die Mitte des 19. Jahrhunderts gab es erste Überlegungen und Theorien dazu, dass es sich bei der spinalen Kinderlähmung um eine virale und damit übertragbare Krankheit handeln könnte. Bis Ende des 19. Jahrhunderts galt Polio als endemische Krankheit. Sie zirkulierte nahezu ununterbrochen und die Rate der Neuerkrankungen blieb in etwa beständig. Erst mit dem 20. Jahrhundert wird Polio zu einer epidemischen Krankheit.\footnote{Wilson, Polio!, S. 28.} \\
1908 gelang den österreichischen Wissenschaftlern Karl Landsteiner (1868--1943) und Erwin Popper (1879--1955) der Nachweis, dass es sich um eine Infektionskrankheit handelt und 1948 schließlich erfolgte erstmals die Isolierung des Poliovirus durch Thomas Weller (1915--2008) und Frederick Robbins (1916--2003),
welche dafür den Nobelpreis erhielten.\footnote{Herwig Kollaritsch, Maria Paulke-Korinek, Poliomyelitis, S. 25, in: Österreichische Ärztezeitung, Heft 22, Wien, November
2014, S. 24--33. Onlineausgabe: \url{http://www.aerztezeitung.at/fileadmin/PDF/2014_Verlinkungen/State_Polio.pdf} 14.6.2016.}\\
Der erste wirksame Totimpfstoff (IPV) dagegen wurde in den 1940er Jahren von dem amerikanischen Arzt und Forscher Jonas Salk (1914--1995) entwickelt. Dieser Impfstoff wurde 1955 in den USA freigegeben, was zu einem radikalen Rückgang der
Polioerkrankungen führte und Jonas Salk auf einen Schlag zu einer weltweiten Berühmtheit machte. Dies nimmt nicht wunder, wenn man
bedenkt, dass man vor der Entdeckung der Impfprophylaxe von weltweit rund 600.000 Poliofällen ausging.\footnote{Tilli Tansey, Pioneer
of polio eradication, in: Medical History, Nature, Issue 520, 2015, S. 620.} Im Nachkriegsjahr 1947 wurden allein in Österreich rund
3.500 Polio-Patienten verzeichnet, wovon 315 starben. In "`regulären Jahren"' wurden 500 bis 1.000 Fälle pro Jahr gemeldet.
\footnote{Kollaritsch, Paulke-Korinek, Poliomyelitis, S. 25.}\\
Albert Sabin (1906--1993) führte die Forschung weiter und entwickelte 1960 einen Lebendimpfstoff, welcher als Schluckimpfung verabreicht
wurde und 1960/61 erstmals in Österreich zur Anwendung kam. In den 1990er Jahren kam es jedoch in Österreich zu einer abermaligen
Umstellung auf den Totimpfstoff, da der Lebendimpfstoff das Risiko einer vakzine-assoziierten paralytischen Poliomyelitis (VAPP)\footnote{Rückmutation des Lebendimpfstoffes zu einem gefährlichen Krankheitserreger, führt zu Impf-Poliomyelitis. Vgl.: Poliomyelitis RKI-Ratgeber für Ärzte: \url{https://www.rki.de/DE/Content/Infekt/EpidBull/Merkblaetter/Ratgeber_Poliomyelitis.html} 20.7.2017.}
barg. Seit 2001 wird zur Polioimpfung in allen Altersgruppen ausschließlich der IPV empfohlen und angewendet.\footnote{Mutz, Spork, Geschichte der Impfempfehlungen in Österreich, S. 95.}\\
Mithilfe des Polioimpfstoffes konnte die Kinderlähmung in den meisten westlichen Ländern gebannt werden. In Österreich wurde
1982 der letzte Fall von Kinderlähmung gemeldet. 2002 bestätigte die WHO die Elimination der Poliomyelitis in Europa. Andere
Regionen der Welt, wie Afrika, Afghanistan oder der Jemen, sind nach wie vor von regelmäßigen Polioausbrüchen bedroht.\footnote{Peter Kriwy, Gesundheitsvorsorge bei Kindern. Eine empirische Untersuchung des Impfverhaltens bei Masern,
Mumps und Röteln, Wiesbaden, 2007, S. 20.}\\
\\
Die \textit{Masern} waren eine weitere, gefährliche Krankheit, für welche in der zweiten Hälfte des 20. Jahrhunderts ein Impfstoff entwickelt
wurde.\footnote{Bereits 1758 wurden Versuche mit einer aktiven Masernimpfung, ähnlich der Pockeninokulation angestellt, blieben aber ohne nennenswerten Erfolg. Ab 1919 experimentierte man mit einer passiven Immunisierung mittels Masernrekonvaleszenzserums, welche einen temporär begrenzten Schutz bot. Vgl.: Heinz Spiess, Schutzimpfungen, Stuttgart, 1958, S. 277--278.} Seine Erstzulassung erfolgte 1963 in den USA und noch im selben Jahr war er in Österreich erhältlich. Seit 1967 ist ein
Lebendimpfstoff verfügbar. Dieser wird bis heute verwendet, da der Totimpfstoff
zu atypischen Masern\footnote{Hohes Fieber, blutiger Ausschlag, Ödeme, Lymphknotenvergrößerung. Vgl.: Mutz, Spork, Geschichte
der Impfempfehlungen in Österreich, S. 95.} führen kann, was beim Lebendimpfstoff nicht vorkommt. Für die WHO gilt der Masernimpfstoff als unentbehrliches Arzneimittel,\footnote{WHO Model List Essential Medicines: \url{http://www.who.int/medicines/publications/essentialmedicines/18th_EML.pdf} 20.5.2016.}
trotzdem greift in Österreich mehr und mehr die Überzeugung um sich, die Masern seien eine harmlose Kinderkrankheit, was nicht
zuletzt die steigende Zahl der Erkrankungen belegt. So wurden etwa für 2015 insgesamt 309 Fälle im nationalen Masern Surveillance
System erfasst.\footnote{Österreich hat zweithöchste Masernrate in Europa, Die Presse Online, 25.5.2016:
\url{http://diepresse.com/home/leben/gesundheit/4996225/Osterreich-hat-zweithochste-MasernRate-in-Europa} 8.6.2016.}
Seit 1993 wird in Österreich die Masernimpfung in Kombination mit Mumps und Röteln (MMR) empfohlen. Diese
führte auch zu einer Unterbindung des Röteln-Wildvirus, das man bis dahin bewusst in Österreich zirkulieren ließ.
Dahinter steckte der Gedanke, dass die Infektion mit dem Wildvirus eine bessere Langzeitimmunität versprach und zur Vermeidung der
Röteln-Embryopathie\footnote{Erkrankt eine schwangere Frau an den Röteln, kann dies zu Fehlbildungen bis hin zur Fehlgeburt führen.
Vgl.: Röteln-Embryopathie: \url{http://flexikon.doccheck.com/de/RC3B6telnembryopathie} 27.5.2016.} ausschließlich Mädchen
ab dem 13. Lebensjahr zu impfen wären.\footnote{Mutz, Spork, Geschichte der Impfempfehlungen in Österreich, S. 95.}
Ähnlich wie bei Masern gibt es auch gegen den Mumps keine ursächliche Behandlung und es können ausschließlich
pflegerische Maßnahmen getroffen werden.\footnote{Mumps: \url{http://www.reisemed.at/krankheiten/mumps} 27.5.2016.}\\
\\
Dieser historische Abriss zeigt sehr anschaulich, dass die Menschen beständig danach strebten, Impfstoffe für alle möglichen Krankheiten zu entwickeln. So kamen zum Beispiel 2005 die Impfstoffe für Windpocken und Rotaviren auf den Markt, 2006 wurde ein HPV-Impfstoff vorgestellt und zuletzt erschien ein Impfstoff gegen den Meningokokkenstamm B. Letzerer wurde im Jänner 2013 von der EU-Kommission zugelassen.\footnote{Neue Impfung gegen Meningitis, Der Standard Online, 23.4.2014: \url{http://derstandard.at/1397521376319/Neue-Impfung-gegen-Meningitis} 8.6.2016.}\\
Die Einführung, respektive Zulassung eines neuen Impfstoffes sorgt (und sorgte) stets für großes Aufsehen,
gefolgt von öffentlichen Risiko-Nutzen-Diskussionen. Zuletzt geschehen ist dies 2006 bei der Erstzulassung des
HPV-Impfstoffes. Die Humane Papilloma-Viren (HPV) sind weitverbreitete Viren, welche primär über Sexualkontakt
übertragen werden und unter anderem für die Entstehung von einigen Krebsarten verantwortlich sind. Besonders zwei Stämme,
HPV-16 und HPV-18, gelten als häufigste Ursache für Gebärmutterhalskrebs (Zervixkarzinom). Es existieren
jedoch rund 100 verschiedene Virentypen, wobei viele davon als harmlos eingestuft werden.\footnote{HPV-Impfung: \url{http://www.netdoktor.at/gesundheit/impfung/hpv-impfung-5339} 14.6.2016.}
Gegen diese beiden gefährlichen Virusstämme soll die 2006 vorgestellte Impfung Schutz bieten.
Diese wirkt jedoch nicht, wenn bereits eine HPV-Infektion vorliegt, weshalb eine
Impfung vor dem ersten Sexualkontakt empfohlen wird. Aber bereits kurz nach deren Einführung
überschlugen sich die Berichte und Artikel über Wirkung und Schaden der Impfung. 2013 kam es in Frankreich zu
ersten Klagen gegen die Herstellerfirma Sanofi Pasteur MSD wegen fahrlässiger Körperverletzung, wie der Kurier vom
25.11.2013 berichtete.\footnote{HPV-Impfstoff: Vier Frauen klagen Hersteller: \url{http://kurier.at/wissen/hpv-impfstoff-vier-frauen-klagen-hersteller/37.536.723}
2.6.2016.}
In dem Zeitungsbericht ist des Weiteren die Rede davon, dass mithilfe der Impfung in Österreich jährlich
rund 700 Krebsfälle vermieden werden können. Das Portal "`Medizin Transparent"' warnt jedoch zur Vorsicht
bei der Nennung derartiger Zahlen. Da die Ausbildung von Zervixkarzinomen mitunter Jahrzehnte dauern kann,
ließen sich aktuell keine validen Aussagen über eine Reduktion der Krebsrate durch die Impfung treffen. Es
könnte jedoch die Wirksamkeit anhand der Reduktion der Krebs-Vorstufen-Raten gezeigt werden.\footnote{HPV-Impfung: nüchterne Fakten statt hitziger Diskussionen: \url{http://www.medizin-transparent.at/hpv-impfung} 2.6.2016.}




\subsection{Impfempfehlung in Österreich}
Aufgrund der steigenden Anzahl an Impfstoffen stellte sich -- nicht nur für die Medizin -- die Frage, welche Impfungen mit welchen
Impfstoffen in welcher Kombination zu empfehlen seien. Dem Bedürfnis nach einer Antwort folgend erschien 1959 in der Wiener Medizinischen Wochenschrift
zum ersten Mal eine allgemeine Impfempfehlung in Österreich. Diese beinhaltete die Tuberkuloseimpfung (BCG) bei Neugeborenen, DPT ab
dem vierten Lebensmonat, die gesetzlich verankerte Pockenimpfung im zweiten Lebensjahr sowie die Polioimpfung.\footnote{Mutz, Spork,
Geschichte der Impfempfehlungen in Österreich, S. 96.}\\
Im Jahr 1974 kam es in Österreich zur Einführung des Mutter-Kind-Passes. Dieser enthält gegenwärtig
eine allgemeine Impfempfehlung, einen Impfpass für das Kind sowie ein Impf-Gutscheinheft, mittels welchem ein Großteil der empfohlenen
Impfungen kostenlos durchgeführt werden kann. Diese Maßnahme trug erheblich zur breiten Akzeptanz der Impfprogramme bei.\footnote{Mutz, Spork, Geschichte der Impfempfehlungen in Österreich, S. 96.} \\
Seit 1984 ist es der Oberste Sanitätsrat (OSR), welcher alljährlich einen offiziellen Impfplan für Österreich ausgibt. Dieser setzt sich aus Experten diverser medizinischer Fächer, der Ärzte- und Apothekerkammer sowie den Sozialversicherungen
zusammen und berät das Gesundheitsministerium in medizinischen Fragen.\footnote{Oberste Sanitätsrat: \url{http://www.bmg.gv.at/home/Ministerium/Oberster_Sanitaetsrat/} 27.5.2016.}
1997 wurde in Zusammenarbeit von Bundesministerium für Gesundheit, den Bundesländern sowie dem Hauptverband der Sozialversicherungsträger
ein gemeinsames Impfkonzept ausgearbeitet. Darin wurde festgelegt, dass österreichische Kinder bis zum 15. Lebensjahr bestimmte,
ausgewählte Impfstoffe kostenlos erhalten. Welche Impfungen das Programm enthält orientiert sich abermals an der Empfehlung des
OSR.\footnote{Kunze, Forum Impfschutz. Das österreichische Impfsystem und seine Finanzierung, S. 9.}\\
Die letzten Erweiterungen des österreichischen Kinder-Impfprogrammes erfolgten 2012 mit der Aufnahme der Pneumokokken- und der
Menigokokkenimpfung.\footnote{Für die Menigogokkenstämme A, C, Y und W135.} 2014 wurde schließlich die HPV-Impfung
(Humane Papilloma Viren) für alle Kinder ab dem vollendeten neunten Lebensjahr dem Programm hinzugefügt.\footnote{Kinderimpfprogramm: \url{https://www.bmgf.gv.at/home/Service/Gesundheitsleistungen/Kostenloses_nbsp_Kinder-Impfprogramm} 11.9.2017.}


\section{Unfälle, Schäden, Kritiker und Gegner}
Nach dieser Einführung in die Geschichte der Impfungen, stellt sich natürlich die Frage, ob diese Reihe an wissenschaftlichen Entdeckungen auch eine Kehrseite hatte. Sprich, ob es im Laufe der Zeit zu impfbedingten Unfällen oder Schäden im Entwicklungsprozess kam. Daneben wurde bereits mehrfach erwähnt, dass die Methode der Schutzimpfung bereits seit ihrer Einführung auf Gegenwind stieß. Da das Kerninteresse dieser Arbeit auf der Impfdebatte liegt, soll also auch die Entwicklung impfgegnerischer Ideen und Strömungen nachvollzogen werden.

\subsection{Was ist ein Impfschaden?}
Grundsätzlich lässt sich festhalten, dass es sich bei einer Impfung um die absichtliche Zuführung (abgeschwächter)
Krankheitserreger am gesunden Menschen handelt, um dadurch eine Immunreaktion hervorzurufen.
Damit wird der Impfstoff als biogenes Arzneimittel bezeichnet, da er einerseits chemisch im Labor hergestellt wird,
andererseits eine natürliche biologische Reaktion im Körper provoziert. Dennoch handelt es sich um ein Arzneimittel
und diese können bekanntlich Nebenwirkungen aufweisen. Um es mit den Worten des deutschen Arztes und Pharmakologen
Gustav Kuschinsky (1904--1992) zu sagen: \textit{"`Wenn behauptet wird, dass eine Substanz keine Nebenwirkungen hat,
so besteht der dringende Verdacht, dass sie auch keine Hauptwirkung besitzt."'}\footnote{Zitiert nach: \url{http://www.reisemed.at/impfungen/impfreaktionen-und-impfnebenwirkungen} 31.5.2016.} \\
Diese Impfnebenwirkungen werden in drei Kategorien eingeteilt:
\begin{enumerate}
  \item{\textbf{Impfreaktion/Impfkrankheit}: Darunter versteht man jene Beschwerden, die im Rahmen der natürlichen Immunantwort des Körpers auftreten.
    Dies können Ausschläge, Rötungen, leichtes Fieber, Gliederschmerzen oder eine abgeschwächte Form der geimpften Krankheit
    (zB. Impfmasern) sein.\footnote{Ursula Wiedermann-Schmitz, u. A., Reaktionen und Nebenwirkungen nach Impfungen. Erläuterungen und Definition in
    Ergänzung zum Österreichischen Impfplan, 2013, in: \url{http://bmg.gv.at/cms/home/attachments/1/5/5/CH1100/CMS1386342769315/impfungen-reaktionen_nebenwirkungen.pdf} 31.5.2016 S. 4.}}
  \item{\textbf{Impfnebenwirkung}: Davon spricht man, wenn es nach der Impfung zu einer schwereren Erkrankung kommt, welche eine
    vorübergehende oder längere Therapiebedürftigkeit nach sich zieht oder zu bleibenden Schäden führt.
    \footnote{Impfnebenwirkung: \url{http://www.reisemed.at/impfungen/impfreaktionen-und-impfnebenwirkungen} 31.5.2016.}}
  \item{\textbf{Impfschaden}: Das ist eine "`\textit{über die Ausmaße hinausgehende gesundheitliche Schädigung durch eine
    Schutzimpfung}"'\footnote{Impfschaden: \url{https://www.gesundheit.gv.at/Portal.Node/ghp/public/content/Nebenwirkungen_von_Impfungen_LN.html} 31.5.2016.},
    womit in der Regel neurologische Erkrankungen wie Lähmungen oder geistige Behinderung infolge von Hirnhautentzündungen gemeint sind. Betont werden muss, dass es sich hierbei um einen juristischen, keinen medizinischen Begriff handelt.\footnote{Impfschaden: \url{https://www.gesundheit.gv.at/Portal.Node/ghp/public/content/Nebenwirkungen_von_Impfungen_LN.html} 31.5.2016.}}
\end{enumerate}
Kommt es zu einer derartigen unerwünschten Impfnebenwirkung oder einem Impfschaden, muss dieser im Rahmen des Arzneimittelgesetzes
§75a im Bundesamt für Sicherheit im Gesundheitswesen gemeldet werden. Wird von den Behörden ein
entsprechender Impfschaden festgestellt, muss eine staatliche Entschädigung geleistet werden.
Höhe und Ausmaß sind im Impfschadensgesetz von 1973 geregelt.\footnote{Impfschaden und Entschädigung: \url{https://www.gesundheit.gv.at/Portal.Node/ghp/public/content/Nebenwirkungen_von_Impfungen_LN.html} 31.5.2016.}\\
Auf internationaler Ebene gibt es für einen Impfschaden eine weitere Definition durch die WHO. Diese spricht von einer
\textit{Adverse Events Following Immunization} (AEFI). AEFI bezeichnet jegliches unerwünschtes
gesundheitliches Ereignis nach einer Impfung, unabhängig eines kausalen Zusammenhanges. Den schlimmstmöglichen Fall, nämlich ein lebensbedrohlicher
Zustand oder der Tod nach AEFI, bezeichnet die WHO als \textit{Serious Adverse Event} (SAE).\footnote{Wiedermann-Schmidt, Reaktion nach Impfungen, S. 5.}\\
Beachtet werden muss hier, dass es sich um Definitionen der Schulmedizin respektive der Rechtswissenschaften handelt. Einige Alternativmediziner
stimmen diesen Beschreibungen nicht zu und verstehen bereits temporäre Verhaltensänderungen wie Unwohlsein, Fieber, vorübergehende Änderung im Schlafverhalten und ähnliches als schweren Impfschaden.\footnote{Jana Gärtner, Elternratgeber im Wandel der Zeit.
Deskriptive Ratgeberanalyse am Beispiel der sogenannten Klassischen Kinderkrankheiten unter Berücksichtigung der Impfdebatte, Berlin, 2010, S. 91.}

\subsection{Unfälle und Schäden durch mangelhafte Impfstoffe}
Befasst man sich mit den Errungenschaften der medizinischen Entwicklung kommt man nicht umhin, auch einen Blick auf etwaige negative Folgen zu werfen, die stets bei der Erprobung neuer Techniken und eben auch Arzneimittel auftreten können. Im Besonderen interessieren an dieser Stelle jene Ereignisse, die im Rahmen der Impfstoffentwicklung zu schwerwiegenden Unfällen führten. Auch hier ist eine vollständige Darstellung aller Ereignisse nicht möglich, wodurch das Augenmerk beispielhaft auf die größeren Ereignisse gelegt wird, begonnen bei der Pockenimpfung.\\
Zu Beginn sah die Methode der Kuhpockenimpfung vor, dass direkt von Mensch zu Mensch immunisiert werden sollte. Dies bedeutete, dass man die Sekrete von impfkranken Kindern gezielt sammelte und aus deren Pusteln neuen Impfstoff generierte. Man hatte dazu etwa in Österreich eigene "`Regenerieranstalten"' eingerichtet. Die (aus heutiger Perspektive) logische Folge dieses Vorgehens war die direkte Übertragung von Krankheiten wie Syphilis, Tuberkulose oder Hepatitis. In Zeiten vor der Erfindung von Antibiotika und der wissenschaftlichen Hygiene waren das in jedem Fall tödliche Krankheiten. Dieses schwerwiegende Problem der Übertragung bereits bestehender Vorerkrankungen des Impflings bei einer Impfung von Arm zu Arm beseitigte man schließlich 1873, als der OSR beschloss, zu jeder Impfung ausschließlich Tierlymphen (also von Tier zum Menschen) zu verwenden.\footnote{Flamm, Vutuc, Geschichte der Pocken-Bekämpfung in Österreich, S. 269.}\\
\\
Als chronologisch nächstes großes Unglück im Rahmen der Impfstoffentwicklung gilt der so genannte \textit{Lübecker Totentanz} von 1930. Dabei handelt es sich um jenes Ereignis, bei welchem nach Verabreichung der BCG-Tuberkulose-Impfung 77 Säuglinge zu Tode kamen.\\
Die Tuberkuloseimpfung wurde Anfang des 20. Jahrhunderts von den französischen Wissenschaftlern
Albert Calmette (1863--1933) und Camille Guerin (1872--1961) entwickelt. Dem Impfling werden in diesem Verfahren abgeschwächte Erreger der Rindertuberkulose
verabreicht, welche die körpereigenen Abwehrstoffe anregen sollen. Das entspricht exakt dem Prozedere  der Pockenimpfung.
1921 wurde die Impfung erstmals erfolgreich in Frankreich durchgeführt und verbreitete sich danach rasch über den Kontinent.
1929/30 erreichte sie erstmals Lübeck. Das dortige Krankenhaus erwarb zu diesem Zweck eine entsprechende BCG-Kultur aus Paris,
um daraus einen Impfstoff zu generieren. Es kam jedoch in Folge von unhygienischen Laborbedingungen und der Beauftragung von zur Impfstofferzeugung
nicht befähigtem Personal zu einer Kontamination des Impfstoffes mit einem virulenten Erreger, welcher den Säuglingen "`verfüttert"',
also über die Nahrung eingegeben, wurde. 72 der 77 verstorbenen Säuglinge starben nachweislich an impfbedingter Tuberkulose.
Es folgte ein weltweit aufsehenerregender Prozess, bei welchem der Klinikleiter Georg Deycke (1865--1938) und der Leiter des Lübecker Gesundheitsamtes
Ernst Altstaedt (1885--1953) als Verantwortliche zu zwei Jahren beziehungsweise 15 Monate Gefängnis verurteilt wurden. Entsprechend kam es zu
einem großen Vertrauensverlust in die Impfung und es dauert bis in die 1950er Jahre, die BCG-Impfung allgemein zu rehabilitieren.\footnote{Lübecker Totentanz, Wiener Zeitung Online, 3.2.2012: \url{http://www.wienerzeitung.at/themen_channel/wissen/geschichte/432666_Luebecker-Totentanz.html} 2.6.2016 und
Lübecker Impfunglück: \url{http://flexikon.doccheck.com/de/L\%C3\%BCbecker_Impfungl\%C3\%BCck} 2.6.2016.}
1991 wurde schließlich in Österreich die allgemeine BCG-Impfung, welche seit 1949 durchgeführt wurde, auf besonders
gefährdete Personengruppen begrenzt. Seit 2001 steht die BCG-Impfung nicht mehr auf der Liste der in Österreich empfohlenen
Impfungen.\footnote{Mutz, Spork, Geschichte der Impfempfehlungen in Österreich, S. 95.}\\
\\
Ähnliches weltweites Aufsehen erregte der \textit{Cutter Unfall} 1955, welcher sich im Rahmen der Einführung der von Jonas
Salk entwickelten Polioimpfung ereignete. Bedenkt man, dass die Polioepidemien jährlich tausende von Kinder betrafen und jene,
welche die Krankheit überlebten, in der Regel dauerhaft schädigte, wundert es nicht, dass die Impfung höchst erwartungsvoll und bereitwillig aufgenommen wurde. So erhielten im April 1955, im Rahmen des ersten öffentlichen
Polio-Impfprogrammes in den USA, rund 200.000 Kinder den Salk-Polio-Impfstoff. Aber bereits nach wenigen Tagen tauchten die
ersten Berichte von Lähmungserscheinungen bei Impflingen auf und die Kampagne wurde nach nur einem Monat gestoppt.\footnote{Michael Fitzpatrick, The Cutter Incident. How America's First Polio Vaccine Led to a Growing Vaccine Crisis, in:
Journal of the Royal Society of Medicine, Issue 99 (3), London, 2006, S. 156.}
Ursache dafür war abermals eine Verunreinigung des Totimpfstoffes mit einem virulenten Wildvirus. Betroffen war ausschließlich jener Impfstoff, welchen das kalifornische Familienunternehmen \textit{Cutter Laboratories} hergestellt hatte, und dessen Verunreinigung von der zuständigen Kontrollbehörde nicht entdeckt wurde.\footnote{Kinderlähmung, in: Die Welt Digital,
24.10.2012: \url{http://www.welt.de/gesundheit/article110213993/Eine-Welt-ohne-Kinderlaehmung-ist-zum-Greifen-nah.html} 2.6.2016.}
Von dem Vorfall betroffen waren rund 40.000 Kinder, bei welchen nach der Impfung Polio diagnostiziert wurde. Etwa 200
davon erlitten bleibende Schäden und zehn Kinder überlebten die Tragödie nicht.\footnote{Fitzpatrick, The Cutter Incident, S. 156. Anm: Die Angaben über die betroffenen Personen variieren mitunter je nach Literatur. Heinrich Spiess spricht etwa von 204 Betroffenen und elf Todesfällen. Vgl.: Spiess, Schutzimpfungen, S. 239.}
Der Vertrauensverlust auf allen Seiten war enorm und der Cutter-Unfall hatte für alle Beteiligten weitreichende Folgen,
ganz besonders für Jonas Salk. Denn ihm wurde auf Grund dieses Ereignisses der Nobelpreis verweigert und das, obwohl der Unfall
nachweislich durch einen Produktionsfehler bedingt wurde und in keinerlei Zusammenhang mit der Wirksamkeit des Salk-Impfstoffes stand.
Auf behördlicher Ebene kam es zu einer Verschärfung der Sicherheitskontrollen im Rahmen der Impfstoffherstellung.
Die Folgen für die US-Pharmaindustrie waren jedoch verheerend. Obwohl die \textit{Cutter Laboratories} nicht der Fahrlässigkeit
schuldig gesprochen wurden, mussten sie den Betroffenen hohe Schadensersatzleistungen auszahlen. Dies war wegweisend für die amerikanische Gerichtsbarkeit. In den folgenden Jahrzehnten kam es zu einer Welle an Schadensersatzklagen gegen Impfstoffhersteller.
Das führte dazu, dass viele Firmen die Impfstoffproduktion generell einstellten und so gab es 1984 nur mehr eine Firma in den USA, welche den DPT-Impfstoff herstellte. Erst 1986 kam es zu einer entsprechenden gesetzlichen Regelung im Rahmen des
\textit{National Vaccine Injury Compensation Programm}.\footnote{Vaccine Injury Compensation Programs: \url{http://www.historyofvaccines.org/content/articles/vaccine-injury-compensation-programs} 8.6.2016.}
Zu guter Letzt führte der Cutter-Unfall zur Umstellung auf den Salbin-Lebendimpfstoff, der die bereits erläuterten Probleme
mit sich brachte.\footnote{Mutz, Spork, Geschichte der Impfempfehlungen in Österreich, S. 95.}\\
\\
Als jüngeres Beispiel für Probleme mit einer Impfung kann der Sechsfachimpfstoff Hexavec angeführt werden, welcher 2004 von der europäischen Zulassungsbehörde der \textit{European Medicines Agency} (EMA) vom Markt genommen wurde. Auftauchende Spekulationen darüber, dass der Impfstoff mit ungeklärten Todesfällen bei Säuglingen in Zusammenhang standen, konnte den offiziellen Berichten zufolge wissenschaftlich nicht belegt werden. Stattdessen wurde die Zurücknahme damit begründet, dass die enthaltene Hepatitiskomponente nicht die gewünschte Langzeitwirkung aufwies.\footnote{Lenzen-Schulte, Impfung. 99 verblüffende Tatsachen, S. 80.} Auch hier gibt es Spekulationen darüber, dass man der Öffentlichkeit den wahren Grund der Zurücknahme, nämlich die Todesfälle, aus Prestigegründen nicht nennen wollte.\footnote{Rolf Schwarz, Impfen -- eine verborgene Gefahr? Impftheorie und Infektionstheorie auf dem Prüfstand, München, 2012, S. 58. \textit{Anmerkung der Autorin: Dieses Werk ist als wissenschaftlich fragwürdig einzustufen.}} Es ist offenbar nicht unüblich oder selten, dass ein Impfstoff vom Markt genommen wird und durch einen anderen ersetzt wird. In der Regel erfährt der Laie davon nur selten und auch über die Gründe kann oft nur spekuliert werden.

\subsection{Eine kurze Geschichte der Impfgegner}
Wie bereits mehrfach angeführt, wurde Jenners Pockenimpfung nicht von jedermann positiv aufgenommen. Bedenkt man jedoch die genannten Umstände (Nichtwissen um die Funktion der Immunisierung, Übertragung von Krankheiten durch die Impfung, ...), wundert es kaum, dass die Menschen Wirkung und Nützlichkeit in Frage stellten. Dazu kam der damit einhergehende intensivierte Kontakt zur naturwissenschaftlichen Medizin sowie ein präventivmedizinischer Eingriff in das gesunde Individuum.\footnote{Wolff, Medizinkritik der Impfgegner, S. 83.}\\

In der Mitte des 19. Jahrhunderts kann man jedoch eine Veränderung in dieser frühen Impfdiskussion erkennen. Sie trat aus dem Medizinerkreis heraus und wurde zu einer "`öffentlichen Bewegung"', die vielfach auch von Nichtärzten getragen wurde. Vor allem dort, wo eine gesetzliche Impfpflicht herrschte, begannen sich Impfgegner in Vereinen zu organisieren sowie Zeitschriften und Petitionen zu drucken. Dies hatte nun zweierlei Effekt: Zum einen griff die Debatte das Selbstverständnis der Ärzteschaft in einer noch nie dagewesenen Art und Weise an. Zum anderen stellte (und stellt) die latente Skepsis und Ablehnung der Maßnahme ein Hindernis für die staatlich gewünschte Durchimpfungsrate dar.\footnote{Wolff, Medizinkritik der Impfgegner, S. 84.}\\
Als Beispiel für die Institutionalisierung der Impfgegnerschaft sei hier Großbritannien, die "`Heimat"' der Pockenimpfung, angeführt: Neben der \textit{Anti Compulsory Vaccination League} gab es noch die \textit{National Association for the Repeal of the Contagious Disease Act}, die \textit{London Society for the Abolition of Compulsory Vaccination} sowie eine \textit{Mothers Anti-Compulsory Vaccination League}. Dazu wurden impfkritische Zeitschriften wie der \textit{Vaccination Inquirer} verlegt. Diese Initativen und Zeitschriften nahmen ihren Anfang ebenfalls im 19. Jahrhundert.\footnote{Porter, Porter, The politics of Prevention, S. 231--252.} Als wohl größter Erfolg (bezogen auf das 19. Jahrhundert) dieser politisch äußerst aktiven britischen Impfgegner gilt ein 1898 in Großbritannien erlassenes Mandat, wonach es möglich war, eine Befreiung von der 1853 eingeführten Impfpflicht zu erlangen, wenn man diese aus Gewissensgründen ablehnte (\textit{conscientious objection}).\footnote{Porter, Porter, The politics of Prevention, S. 234.}\\
Das Engagement der Impfgegner ging bis hin zu großen, internationalen Kongressen wie etwa dem ersten Impfgegnerkongress in Paris 1879, dem zweiten internationalen Kongress der Impfgegner und Impfzwanggegner in Köln 1881 oder jenem in Berlin 1899.\footnote{Vgl.: Peter Baldwin, Contagion and the State in Europe, 1830--1930, Cambridge, 2004, S. 293, Fußnote 196--197.} \\
\\
Auch in Österreich lassen sich bereits frühe Spuren einer Impfgegnerbewegung finden. So könnte man zum Beispiel in der Nichteinführung einer allgemeinen Impfpflicht in der Habsburgermonarchie, und zwar entgegen der Forderung führender Mediziner und der Empfehlung des Obersten Sanitätsrates, einen Beweis für deren Präsenz sehen.\footnote{Entsprechende Bemühungen für eine Übernahme der Impfpflicht nach deutschem Vorbild sind belegt durch Sitzungsprotokolle und Gesetzesentwürfe. Vgl.: Flamm, Vutuc, Geschichte der Pocken-Bekämpfung in Österreich, S. 270 und Kaiser, Pocken und Pockenschutzimpfung, S. 199.} Als zweites Beispiel kann die Einführung von Pasteurs Tollwutimpfung in den 1880ern genannt werden. Denn auch dieser ausgesprochen aufwendigen, neuen Methode stand man in Österreich skeptisch gegenüber, was zu Auseinandersetzungen von Impfbefürwortern und -gegnern im österreichischen Abgeordnetenhaus führte. Auf Grund massiver Gegenstimmen von führenden Medizinern wie Anton Ritter von Frisch (1848--1917)\footnote{Vorstand der chirurgischen Abteilung der Allgemeinen Poliklinik Wien. Von Frisch wurde zu Pasteur nach Paris gesendet, um das Verfahren dort zu erlernen. Nach eigenen Tierversuchsreihen, welche nicht den erwünschten Erfolg brachten, äußerte er sich in einer Publikation negativ über Pasteurs Methode: Heinz Flamm, Pasteurs Wut-Schutzimpfung - vor 130 Jahren in Wien mit Erfolg begonnen und doch offiziell abgelehnt, in: Wiener Medizinische Wochenschrift, Heft 165, Wien, 2015, S. 322--339.} und dessen Lehrer Theodor Billroth (1829--1884)\footnote{Theodor Billroth, "`Vater der Chirurgie"', unterstützte von Frischs Behauptung, dass man nicht ausschließen könnte, dass die Impfung selbst die Wutkrankheit auslösen könne: Flamm, Pasteurs Wut-Schutzimpfung, S. 332. Theodor Billroth: \url{http://geschichte.univie.ac.at/de/node/33917} 11.9.2017.},
kam es erst verzögert, nämlich 1894, zur Eröffnung der ersten Tollwut-Impfstation in Österreich.\footnote{Flamm, Pasteurs Wut-Schutzimpfung, S. 335--338.} Einen Eindruck dahingehend, in welche Richtung sich die frühe Impfdiskussion in der zweiten Hälfte des 19. Jahrhunderts bewegte, gibt ein kurzer Auszug aus dem Sitzungsprotokoll der 216. Sitzung der X. Session vom 23. April 1888 des österreichischen Abgeordnetenhauses. Darin schließt der Arzt Dr. Roser seine Rede über eine Ablehnung der Tollwutimpfung mit den Worten: \textit{''So will ich nur erwähnen, dass Pasteur ein Franzose (und Jude), Wiedersperg ein Cheche und Billroth ein Deutscher ist.''}\footnote{Zitiert nach: Vgl.: Flamm, Pasteurs Wut-Schutzimpfung, S. 335.}\\
\\
Als Beispiel eines führenden Impfgegners im deutschen Sprachraum kann Carl Georg Gottlob Nittinger (1807--1874) angeführt werden. 1848 veröffentliche er sein erstes impfkritisches Werk. Bis zu seinem Todesjahr 1874 war die Zahl der Publikationen auf 25 Buchtitel angewachsen.\footnote{Wolff, Medizinkritik der Impfgegner, S. 84} Zu Nittingers Schriften zählen Titel wie "`Die Impfvergiftung des Württembergischen Volkes"'\footnote{Carl Georg Gottlob Nittinger, Über die 50jährige Impfvergiftung des württembergischen Volkes, Stuttgart, 1852.}, "`Die Impfung ein Mißbrauch"'\footnote{Carl Georg Gottlob Nittinger, Die Impfung ein Mißbrauch: Spiegel für die Schrift: "`Würdigung der großen Vortheile der Kuhpocken-Impfung für das Menschengeschlecht von Dr. Michael Reiter, Stuttgart, 1853.} oder "`Gott und Abgott oder die Impfhexe"'\footnote{Carl Georg Gottlob Nittinger, Gott und Abgott oder die Impfhexe, Stuttgart, 1863.}. \\
Zur Frage der Impfung vertrat Nittinger folgende Meinung: \\
\textit{"`Die giftige Jauche der Kuh kann blos die Elemente zu einer neuen Krankheit abgeben, die Gesundheit vergiften, oder die Elemente einer Krankheit vermehren, die Krankheit vergiften; nie aber werden kranke Elemente die Gesundheit bauen, bewahren. Jeder Theil des menschlichen Körpers, der eine Ader, einen Nerv besitzt, kann dieser Ver- und Uebergiftung verfallen und kann dieselbe fortleiten. Darin liegt das Schrecklichste der Impfung."'}\footnote{Nittinger, Die Impfung ein Mißbrauch, S. 49.}\\
Als Vertreter der Naturheilkunde empfahl der Mediziner Nittinger für die Pocken folgende Behandlung: \\
\textit{"`Werden die Blattern mit dem Compass des Wassers kühl behandelt, mit Eis, Obst, Limonade, Weinstein, Epsomsalz, Waschen, Baden, Wiklungen bei guter Lüftung, säuerlichen Speisen: so sind sie die leichteste und in ihren Folgen die reinigendste Fieberform, wonach später die Schönheit, die Kraft, die Zeugung aufs lieblichste in die Blüthe kommen."'}\footnote{Carl Georg Gottlob Nittinger, Die Impfregie mit Blut und Eisen, Stuttgart, 1868, S. 51.}\\
Man könnte ihn damit auch dem damals verbreiteten therapeutischen Nihilismus zuordnen, welcher "`künstliche"' Heilmittel weitgehend ablehnte. Diese Strömung nahm ihren Ursprung Anfang des 19. Jahrhunderts und hatte auch in der so genannten Wiener Medizinischen Schule eine große Anhängerschaft.\footnote{William M. Johnston, Österreichische Kultur und Geistesgeschichte. Gesellschaft und Ideen im Donauraum 1848 bis 1938, 4. Auflage, Wien, 2006, S. 230--235.} \\
Einer von Nittingers Mittstreitern war der Linnicher Arzt Heinrich Oidtmann (1838--1890).\footnote{Heute besser bekannt als Begründer des ältesten, noch bestehenden Glasmalereibetriebes Deutschlands. Vgl.: Heinrich Oidtmann: \url{http://www.glasmalerei-oidtmann.de/chronik.html} 12.7.2016.} Oidtmann sah in den Pocken nicht nur eine "`Lumpenkrankheit"', die ihren Ausgang in den untersten Bevölkerungsschichten nahm, sondern stellte zudem einen Zusammenhang zwischen den Schafpocken und den Menschenpocken her. Demnach wurde die Krankheit vom Schaf auf den Menschen übertragen, indem man zum Beispiel Schafswolle als Bettunterlage in den Kinderbetten verwendete.\footnote{Reinhold Gerling, Blattern und Schutzpocken-Impfung. Öffentliche Anklage: Impfgegner c/a Gesundheitsamt. Kritische  Beleuchtung  und  Widerlegung  der  Irrthümer der  im  Kaiserlichen  Gesundheitsamt  bearbeiteten  Denkschrift  zur Beurtheilung  des  Nutzens  des  Impfgesetzes, Berlin, 1896, S. 58.}\\
Der Arzt Max von Niessen (1860--?) kann ebenfalls als Vorkämpfer der Impfgegner-Bewegung im  deutschsprachigen Raum gesehen werden. Er gehörte zu jenen Vertretern, welche sich im beginnenden 20. Jahrhundert an der aufkeimenden Rassenhygiene orientierten und durch die Impfung eine Schädigung des Erbgutes und damit eine Schwächung des Volkes befürchteten.\footnote{Max von Niessen, Gibt es Naturpockenschutz durch die Kulturpockenverseuchung der Vakzination? Dresden, 1935, S. 14.}\\
\\
Als der am häufigsten zitierte Impfgegner des 20. Jahrhunderts gilt ohne Zweifel der deutsche Arzt Gerhard Buchwald (1920--2009). Er hielt unzählige Vorträge und veröffentlichte eine große Zahl an Publikationen gegen die Impfung. Zudem war er bis einschließlich 2005 ärztlicher Berater des deutschen Schutzverbandes für Impfgeschädigte.\footnote{Gerhard Buchwald, Der Impf-Unsinn. Vorträge des Jahres 2004, Norderstedt, 2004, S. 8. \textit{Anmerkung der Autorin: Dieses Werk ist als wissenschaftlich fragwürdig einzustufen.}} Gelegentlich wird er auch als der "`deutsche Impfgegnerpapst schlechthin"' bezeichnet.\footnote{Wolfgang Maurer, Impfskeptiker -- Impfgegner. Von einer anderen Realität im Internet, S. 68, in: Pharmazie in unserer Zeit, Volume 37, Jänner, 2008, S. 64--70.} In einem seiner vielzitierten Bücher etwa stellte Buchwald einen Zusammenhang zwischen Impfung, Dummheit und steigenden Kriminalitätsraten her:\\
\textit{"`Zur Erklärung zunehmender Dummheit und zunehmender Gewaltkriminalität brauchen wir nicht die ausgefallensten Theorien heranziehen, denn die Lösung liegt auf der Hand: Intelligenzverlust führt zur Kriminalität. Um es deutlich zu sagen: Ursachen dieser Entwicklung sind die Impfungen.\grqq}\footnote{Anita Petek-Dimmer, Rund ums Impfen, Buchs, 2004, G. Buchwald. Nachwort zur 1. Auflage, S 177. \textit{Anmerkung der Autorin: Dieses Werk ist als wissenschaftlich fragwürdig einzustufen.}}\\
In einem weiteren, oft bemühten Zitat attestiert er den Kindern der dritten Welt eine allgemeine Unterentwicklung der Gehirne und Nervensysteme, wodurch sie die Impfung besser vertragen würden:\\
\textit{"`In der Dritten Welt ist sicher vieles anders als bei uns; Kultur, Zivilisation und Wohlstand. Wahrscheinlich sind nicht nur die dortigen Länder in ihrer Gesamtheit unterentwickelt, möglicherweise sind dies auch die Nervensysteme der Neugeborenen und Kleinkinder. [...] Trotz zunächst noch bestehender kindlicher Unreife der Gehirne unserer Kinder, scheinen diese im Gegensatz zu den Gehirnen der Kinder der Dritten Welt doch hoch entwickelt zu sein, um auf Impfungen entsprechend zu reagieren.\grqq}
\footnote{Gerhard Buchwald, "`Gedanken zu Publikationen eines Impfgegners"'. Richtigstellung zur Veröffentlichung des Herren Dr. W. Ehrengut, in: Naturheilpraxis, Heft 5, o. A., 1989, S. 5--10. \textit{Anmerkung der Autorin: Dieses Werk ist als wissenschaftlich fragwürdig einzustufen.}}\\
Obwohl Buchwald ein studierter Mediziner war, steht das Fehlen jeglicher wissenschaftlicher Grundlagen derartiger Aussagen außer Zweifel.\\
\\
Zu den aktivsten Impfgegnern des 21. Jahrhunderts im deutschen Sprachraum zählen Stefan Lanka, Ryke Geerd Hamer, Hans Tolzin, August Zöbl oder Johann Loibner. Sie betreiben Vereine\footnote{Verein Netzwerk Impfentscheid, AEGIS -- Aktives Eigenes Gesundes Immunsystem.}, Verlage\footnote{Klein-klein Verlag, Verlag Netzwerk Impfentscheid, AEGIS Verlag, Kopp Verlag.}, Internetportale\footnote{Impfen nein danke: \url{http://www.impfen-nein-danke.de/} 12.7.2016.}, halten Vorträge, verfassen Publikationen, produzieren DVDs oder posten Videos auf YouTube. Einige der genannten Persönlichkeiten stehen für die extremste Form der verschwörungstheoretischen Impfgegner, die sich nicht selten in einem politisch rechten Rand\footnote{Wie der umstrittene Ryke Geerd Hamer und dessen Neue Germanische Medizin.} bewegen und zum Beispiel die Existenz von Viren grundsätzlich leugnen wie Stefan Lanka.\\
Als Österreichs führender Impfgegner kann erwähnter Johann Loibner angeführt werden. Dem steirischen Allgemeinmediziner wurde 2009 auf Grund seiner öffentlichen Ablehnung der Impfprophylaxe die Zulassung entzogen. 2013 wurde Loibner jedoch juristisch rehabilitiert, gab jedoch 2015 seine ärztliche Praxis auf.\footnote{Johann Loibner: \url{http://dr.loibner.net/} 12.7.2016 und Artikel Standpunkte: Höchstgericht kippt Berufsverbot für Impfkritiker, Springermedizin, 17.9.2013: \url{http://www.springermedizin.at/artikel/36633-standpunkte-hoechstgericht-kippt-berufsverbot-fuer-impfkritiker} 16.6.2016.} Seine Hauptaktivität hat sich nun auf den Verein AEGIS Österreich verlegt, in dessen Namen er Vorträge hält sowie Bücher und Internetartikel zu aktuellen Themen publiziert. So findet sich auf der AEGIS-Homepage etwa ein kurzer Artikel über das Ebolavirus. Darin stellt er indirekt die Behauptung auf, dass es sich dabei im Grunde um die Vertuschung einer neuen Pockenepidemie handeln würde:\\
\textit{"`Seit Jahren werden über die Medien Berichte über Ebolafieber und des Ebolavirus auf der ganzen Welt verbreitet. Da dank der Pockenimpfung die Pocken angeblich besiegt wurden, können oder dürfen es nicht die Pocken sein. Auch wenn die Symp-\\tome des nun so genannten Ebolafiebers absolut mit den Symptomen der schweren Pocken übereinstimmen, wird nicht von Pocken gesprochen."'}\footnote{Pocken sind tot, Ebola ist auferstanden (Dr. Loibner): \url{http://www.aegis.at/wordpress/pocken-sind-tot-ebola-ist-auferstanden-dr-loibner/} 12.7.2016. \textit{Anmerkung der Autorin: Dieser Aufsatz ist als wissenschaftlich fragwürdig einzustufen.}}\\
\\
Daneben gibt es auch religiöse Gruppen, wie die Anthroposophen, Christliche Wissenschaft, verschiedene Freikirchen oder auch Sekten wie Scientology, welche Impfungen ablehnend oder zumindest skeptisch begegnen. Die \textit{Churce of Christian Science} etwa betrachtet Krankheit als Folge mangelnden Verständnisses von Gottes Ratschluss, was Krankheit zu einer Illusion und jegliche Behandlung obsolet macht. In zahlreichen weiteren religiösen Gruppierungen wird die ärztliche Behandlung und damit gleichzeitig die präventive Impfung abgelehnt.\footnote{Boris Velimirovic, Impfgegner, S. 96, in: Irmgard Oepen, Amardeo Sarma (Hg.), Parawissenschaften unter der Lupe, Münster, 1995, S. 43--47.}

\section{Kategorisierung und Auswertung der Argumente}
Nachdem nun die Impfgeschichte von mehreren Seiten beleuchtet wurde, soll die Aufmerksamkeit im Folgenden auf die Sammlung, Kategorisierung und Auswertung der verwendeten Argumente gelegt werden.\\
\\
Zunächst wurden zwei Auflistungen, eine für die Pro- und eine für die Contraargumente in einer Excel-Datei angelegt. Erfasst wurde dabei auf jeder Seite zunächst das Argument für oder gegen die Impfung, die Quelle (das Werk aus dem die Argumente entnommen wurden), die Profession des Autors, die Gruppe (im Sinne von Befürworter, Skeptiker oder Gegner), das Erscheinungsjahr des Werkes, aus welchem das Argument entnommen ist, die induktive Kategorie, welcher das Argumente zugeordnet wurde, die Art der Impfung sowie ein Zusatzfeld für Notizen und Anmerkungen zum Kontext des Argumentes. Als Quellen dienen wie erläutert medizinhistorische Werk des 19. Jahrhunderts sowie Quellen aus dem 20. und 21. Jahrhundert aus den Themenkreisen  Impfen, Kinder- und Infektionskrankheiten. Daneben wurden auch Elternratgeber, welche sich mit Impfungen auseinandersetzen, herangezogen. Das Material kann somit wie im Ablaufmodel erläutert aus unterschiedliche Textarten bestehen. Die gesammelten Argumente können im Anhang nachgelesen werden. Sie wurden jeweils im Volltext als Originalzitat übertragen. In Ausnahmefällen wurden sehr ausschweifende Formulierungen in älteren Werken ausgelassen und mittels eckiger Klammer und drei Punkten [...] gekennzeichnet. Die Auswahl der Textstellen, welche in die Liste aufgenommen wurden, orientiert sich an den in Kapitel 5.3 geschilderten Kriterien, wonach der Sinn eines Argumentes darin liegt, sein Gegenüber von der jeweiligen Meinung zu überzeugen. Ein Argument besteht aus einer oder mehreren Aussagen und mindestens einer Schlussfolgerung. Überträgt man das auf Druckwerke zum Thema Impfen, kann davon ausgegangen werden, dass die Kernaussage der Autoren darin besteht, dass die Impfung entweder gut oder schlecht ist. Dem folgend müssen die ausgewählten Argumente als Schlussfolgerung auf die gestellte Annahme des Autors verstanden werden.\\
\\
Von den gelesenen Büchern konnten aus 33 Werken insgesamt 251 Argumente ausgewertet werden, davon entfallen 125 auf die Contra- und 126 auf die Proseite. Die Gesamtzahl von 33 Büchern ergibt sich aus dem Bemühen, auf der Impfbefürworter und der Impfgegnerseite eine annähernd gleiche Anzahl von Argumenten zu haben, um einen symmetrischen Vergleich zu ermöglichen. Entsprechend wurde bei 125 auf der einen und 126 Argumenten auf der anderen Seite die Auswertung zu einem Ende gebracht. Diese Grenze ist eine rein subjektive und dem Rahmen der Arbeit geschuldet. Die Zahl der Argumente und damit die der zu lesenden Bücher aus den drei Jahrhunderten kann dem Forschungszweck gemäß beliebig weitergeführt werden.\\
Wie sieht es nun mit der Verteilung der Werke auf die unterschiedlichen Jahrhunderte aus: 17 der gelesenen Bücher stammen aus dem 19. Jahrhundert, die restlichen 16 verteilen sich zu jeweils acht Werken gleichmäßig auf das 20. und 21. Jahrhundert. Für die Differenz bei der Aufteilung gibt es verschiedene Gründe. Einer liegt in dem erläuterten starken Ungleichgewicht der Publikationen. Wie im Kapitel Forschungsstand geschildert, finden sich im 19. Jahrhundert eine Vielzahl an Veröffentlichungen zu den genannten Themenkreisen, wohingegen es im 20. Jahrhundert zu einem deutlichen Einbruch kommt. Erst gegen Ende des 20. und dem Beginn des 21. Jahrhunderts nehmen die Publikationen zu den Themen Impfen, Kinder- und Infektionskrankheiten sowie Elternratgeber wieder deutlich zu. Ein anderes Problem stellt die Verfügbarkeit der Werke dar. Während jene des 19. Jahrhunderts erstaunlich gut zugänglich sind, stößt man bei Werken aus dem 20. Jahrhundert mitunter an Grenzen. Als Beispiel kann der Elternratgeber von Christine Weiskopf\footnote{Christina Weiskopf, Abenteuer Impfung. Was Eltern über Kinderkrankheiten und Impfungen wissen sollen, Lappersdorf, 2007.} genannt werden, welcher zum Beispiel nicht mehr verlegt wird. Ein anderes Problem ergibt sich daraus, dass sich nicht jedes Buch aus den ausgewählten Themenkreisen für die Auswertung eignet. Darunter fallen zum Beispiel jene Werke, die sich zwar mit Impfungen auseinandersetzten, aber überwiegend medizinische Aspekte und Abläufe erläutern. Andere Werke wenden sich einer allgemeinen Systemkritik zu, schildern persönliche Einzelfälle oder betrachten Impfung von einem zu allgemeinen Gesichtspunkt aus, ohne dabei tatsächliche Argumente für oder gegen eine Impfung anzuführen. So etwa bei Wilhelm Ressel\footnote{Wilhelm Ressel, Das Impfgeschäft als starrstes Dogma der modernen orthodoxen Medizin. Richtigstellung falscher und gefährlicher zunftwissenschaftlicher Ueberlieferungen. Zugleich und hauptsächlich ein Weckruf an Deutschlands Zeitungs=Redakteure, Dresden, 1910.}, Karl Krafeld\footnote{Karl Krafeld, Impfen -- Völkermord im dritten Jahrtausend?, Stuttgart, 2003.} oder Harris L. Coulter\footnote{Harris L. Coulter, Barbara L. Fisher, Dreifach-Impfung. Ein Schuß ins Dunkle, Schäftlarn, 1996.}.\\
Betreffend der Berufszuordnung der Autoren ergibt sich, dass studierte Mediziner die überwiegende Zahl der Verfasser darstellen, daneben finden sich ein Pfarrer (Johann Kumpfhofer), ein Wundarzt (C. R. Aiking), zwei Heilpraktiker (Rolf Schwarz und Christina Weiskopf) und ein Lehrer (Friedrich Becker). Der Autor David Zimmer konnte keiner Profession zugeordnet werden.

\subsection{Kategorisierung der Contraargumente}
Wie eingangs erläutert, erschien bei näherer Betrachtung des Materials eine induktive Kategorisierung der Argumente als praktikabel. Die Kategorienbildung erfolgte durch das beschriebene Vorgehen der Zusammenfassung gemäß der qualitativen Inhaltsanalyse nach  Philipp Mayring in mehreren Arbeitsschritten: Zunächst wurden die Argumente im Originaltext gesammelt und erfasst. Nach etwa 50 \% des Materials wurde die Liste überarbeitet und ein erster Versuch unternommen, die Argumente zu unterteilen. Diese ersten Kategorien wurden in weiteren Schritten überarbeitet und zusammengefasst. Der Name der jeweiligen Kategorie ist entweder ein Begriff aus dem Text oder eine Eigenschaft, welche die Art der Argumente unter einem Wort subsumiert. Schließlich erschienen die nun alphabetisch geschilderten induktive Kategorien als sinnvoll:\\
\\
In die Kategorie \textbf{Fragwürdigkeit} fallen die Argumente, welche Zweifel am Impfsystem selbst aufwerfen sollen. Sie verweisen auf ungeimpfte Ärzte, etwaige fehlende Ausbildung der Impfärzte, zu kurze Studienzeiträume, Fehler bei der Impfstoffherstellung oder stellen Statistiken in Frage.\\
\\
\textbf{Gesundheitsschädigung} steht hier als Kategorienname für jene Argumente, die der Impfung eine allgemein schädliche Wirkung attestieren. Die Autoren halten die Impfung für gefährlich, giftig, nachteilig oder ungesund. Die Bandbreite geht hier von der Schilderung einfacher Nebenwirkungen bis hin zum ausgemalten Verfall der Gesellschaft, wie ihn etwa Nittinger skizziert.\\
\\
Die Kategorie \textbf{Gewissen} fast jene Argumente zusammen, welche an das Gewissen des Lesers appellieren. Die Beanstandung des Impfzwangs als "`Beleidigung des gesunden Menschenverstandes"' fällt genau so in diese Sparte wie der Verweis auf impfbedingt überfüllte Friedhöfe oder der Eingriff in die menschliche Integrität.\\
\\
Autoren, welche in Impfungen eine negative Beeinflussung des Körper sehen, wurden in der Kategorie \textbf{Beeinträchtigung} zusammengefasst. Dies ist der Fall, wenn der kindliche Organismus durch die Impfung an einer natürlichen Entwicklung gehindert wird, weil zum Beispiel das Durchmachen der Krankheit das Krebs- oder Allergierisiko senkt, eine Impfung hingegen den Lebensweg verändert, die Persönlichkeitsentwicklung stört oder die Krankheiten ohnehin nur ins Erwachsenenalter verschiebt, wo sie fatalere Folgen haben können.\\
\\
Mit der Risikowahrnehmung der Impfung befassen sich jene Argumente, welche in der Kategorie \textbf{Kosten/Nutzen} subsumiert wurden. Sie stellen den Nutzen der Impfung zu Ungunsten von hohen Kosten in Frage. Der Begriff "`Kosten"' wird hier einerseits für den finanziellen Aspekt verwendet, dient aber auch als Synonym für "`Risiko"'.\\
\\
Die Kategorie \textbf{Machtinteressen} bezieht sich auf Autoren, die hinter der Impfung eine große gewinnorientierte Interessengemeinschaft sehen, welche die Bevölkerung mittels der Impfung bewusst und absichtlich schadet, um daraus Profit zu machen. Zu dieser Lobby gehören in erster Linie die Pharmaindustrie, aber auch die Wissenschaft und der Staat.\\
\\
Einige Autoren der Gegnerseite sehen in der Impfung einen Widerspruch gegen die Natur, die göttliche Ordnung oder beidem zu gleich. Diese Argumente wurden in der Kategorie \textbf{Religions-/Naturgesetz} zusammengefasst.\\
 \\
Unter dem Begriff {\textbf{Zwecklosigkeit} werden jene Argumente subsumiert, welche die Impfung für unnötig und überflüssig erklären, da entweder die zu impfende Krankheit als harmlos eingestuft wird, gesunde Ernährung und volles Stillen die Kinder vor einer Ansteckung bewahrt oder man trotz Impfung erkranken kann.\\
\\
Die letzte Kategorie der Contraseite ist Zusammengefasst unter der Bezeichnung \textbf{ohne Wirkung}. Diese Argumente sprechen den Impfungen ganz generell jegliche positive Wirkung ab. Es ist die Rede von fehlenden wissenschaftlichen Beweisen und es wird  dafür plädiert, dass der Rückgang der Seuchen einzig und allein in der Verbesserung der Hygiene und der Lebensbedingungen liegt.


\subsection{Kategorisierung der Proargumente}
Nach der Erläuterung der induktiven Kategorien der Impfgegnerseite, folgt nun die Einteilung der Befürworterseite, ebenfalls in alphabetischer Reihenfolge:\\
\\
In der Kategorie \textbf{einziges Mittel} finden jene Argumente Platz, welche in einer Impfung die absolut einzige Möglichkeit sehen, sich vor einer Krankheit zu schützen, da es gegen diese keine andere, sichere Behandlungsmöglichkeit gibt.\\
\\
Auf den \textbf{Erfahrungswert} beziehen sich die Argumente, die den praktischen Beweis für die Wirksamkeit der Impfung betonen. Dies kann sich auf allgemeine, gesellschaftliche, respektive wissenschaftliche,  als auch auf persönliche Erfahrungswerte des Autors beziehen.\\
\\
Gleich wie bei den Impfgegnern, gibt es auch bei den Befürwortern eine Kategorie mit der Bezeichnung \textbf{Gewissen}. Auch hier werden alle Argumente zusammengefasst die an das Gewissen des Lesers appellieren, nur in diesem Fall um ihn dadurch von der Notwendigkeit der Impfung zu überzeugen.\\
\\
Die Argumente der Kategorie \textbf{Gesundheitsförderung} beziehen sich darauf, dass der Patient von der Impfung ausschließlich profitieren kann und sein Gesundheitszustand -- in welcher Form auch immer -- nach der Impfung merklich besser ist als davor. Sei es, dass er vorher kränklich, nachher gesund ist, besser gegen Allergien geschützt oder ganz allgemein das Immunsystem gestärkt wird.\\
\\
Eine weitere, zumindest dem Namen nach deckungsgleiche Kategorie ist jene der \textbf{Kosten/Nutzen}. Sie stellt den positiven Nutzen der Impfung über eine geringe Möglichkeit Schäden davonzutragen.\\
 \\
Die \textbf{Obrigkeit} als Kategoriengruppe subsumiert alle Argumente, welche die Regierenden als positives Vorbild anführen, welche die Impfung für gut befunden haben und denen es damit zu folgen gilt.\\
\\
Die Gruppe von Argumenten, welche unter der Bezeichnung \textbf{Religion} zusammengefasst wurden, beziehen sich auf die gottgegebene elterliche Pflicht, ihre Kinder zu schützen oder verweisen auf den göttlichen Segen, welcher von der Impfung ausgeht.\\
\\
Die Kategorie \textbf{Sicherheit} betont, dass Impfungen weder schädliche Nebenwirkungen haben noch anderweitige Erkrankungen auftreten. Auch Folgeerkrankungen, wie sie oft nach einer schweren Infektion auftreten (Narben, Taubheit, Blindheit, Lähmung, ...) werden nach einer Impfung ausgeschlossen.\\
\\
Die letzte Kategorie der Proseite trägt die Bezeichnung \textbf{Statistik}. Diese Gruppe von Argumenten verweist auf offizielle statistische Auswertungen, wie niedrigere Sterblichkeitsraten nach Krankheit, Vergleichsstudien von Geimpften und Ungeimpften zugunsten ersterer oder hebt den statistisch Rückgang einer Krankheit hervor.

\subsection{Auswertung der Impfbefürworterargumente}
Nach dieser Erläuterung über die inhaltlichen Codierung der einzelnen Kategorien, stellt sich als nächstes die Frage, wie viele der gesammelten Argumente welchen Kategorien zugeordnet werden konnten:\\
\\
Mit 47 korrespondierenden Argumenten stellt die Kategorie \textbf{Sicherheit} die Bedeutendste der Proseite dar. Besonders häufig findet man diese im frühen 19. Jahrhundert, vor allem bei den ersten Verfechtern der Kuhpockenimpfung, wie etwa Husson (1801)\footnote{H. M. Husson, Historische und medizinische Untersuchungen über die Kuhpockenkrankheit, Marburg, 1801.}, Aikin (1802)\footnote{C. R. Aikin, Kurze Uebersicht der wichtigsten Erfahrungen über die Kuhpocken, Pesth, 1802.} oder de Carro (1802)\footnote{Johann de Carro, Beobachtungen und Erfahrungen über die Impfung der Kuhpocken, Wien, 1802.}. Zwischen 1801 und 1830 konnten hier immerhin 34 verschiedene Variationen zugeordnete werden, in welchen die Autoren versuchen, die absolute Sicherheit und Harmlosigkeit der Kuhpockenimpfung zu belegen. Ihnen geht es vor allem um die Kuhpocken selbst, welche im Vergleich zu den echten Menschenpocken (unter normalen Umständen) nicht letal endeten, keine Spätfolgen mit sich brachten und -- was überraschenderweise eher selten betont wird -- nicht von Mensch zu Mensch übertragbar waren, somit im Vergleich zur Inokulation keine Epidemien hervorbringen konnte. Im späten 19., 20. und 21. Jahrhundert findet diese Argumentationskette seltener Anwendung. Während früher hauptsächlich die Sicherheit Betonung findet, steht später das Fehlen respektive Ausbleiben von Nebenwirkungen wie Allergien (bei Impfungen allgemein) oder Autismus (bei MMR) im Vordergrund.\\
\\
Die kleinste Kategorie der Proseite stellt mit drei Argumenten jene der \textbf{religiösen Motive} dar, welche die Eltern direkt ermahnen, ihrer gottgegebenen Pflicht nachzukommen und nach allen Möglichkeiten ihre Kinder zu schützen. Dies ist eindeutig ein Phänomen des frühen 19. Jahrhunderts und konnte nur bei dem Salzburger Arzt d'Outrepont (1803)\footnote{Joseph d'Outrepont, Belehrung des Landvolkes über die Schutzblattern. Nebst einem kurzen Unterrichte über die Impfung derselben für die Wundärzte, Salzburg, 1803.} und dem Linzer Pfarrer Kumpfhofer (1808)\footnote{Johann Kumpfhofer, Predigt von der Pflicht der Eltern ihren Kindern die Kuhpocken einimpfen zu lassen, Linz, 1808.} ausgemacht werden. Die Impfung wird dabei als "`Segen des Himmels"' gewertet, der von liebenden und gottesfürchtigen Eltern nicht verschmäht werden dürfe.\\
\\
Die Kategorien \textbf{Obrigkeit} und \textbf{Gewissen}, mit jeweils fünf Argumenten, abermals von Kumpfhofer (1808)\footnote{Kumpfhofer, Predigt von der Pflicht der Eltern, 1808.}, de Carro (1802)\footnote{de Carro, Beobachtung und Erfahrung über die Impfung, 1802.}, d'Outrepont (1803)\footnote{d'Outrepont, Belehrung des Landvolkes, 1803.} und Krauss (1820)\footnote{Georg Friedrich Krauss, Die Schutzpockenimpfung in ihrer endlichen Entscheidung, als Angelegenheit des Staats, der Familien und des Einzelnen, Nürnberg, 1820.}, können ebenfalls als Besonderheit des frühen 19. Jahrhunderts gewertet werden. Hier wird zum einen der Landesfürst ins Feld geführt, der in väterlicher Sorge nicht nur seine eigenen Kinder impfen lässt, sondern mittels Erlass diese Wohltat auch seinen Untertanen angedeihen lässt, womit es per se schon zur Pflicht wird, dies dankbar anzunehmen. Bei letzterem werden auch die Eltern direkt angesprochen und darauf hingewiesen, welches Leid sie sich und den geliebten Kindern mit diesem einfachen Mittel ersparen würden und durch Unterlassen der Impfung bewusst und absichtlich herbeigeführt hätten.\\
\\
Die dritte Kategorie mit fünf Argumenten ist jene der \textbf{Kosten/Nutzen}. Diese findet man 1802, 1888 und 1901, womit alle ausschließlich die Pockenimpfung betreffen. Es wird zum Beispiel das leichte Unwohlsein durch die Impfung dem klaren Nutzen des Schutzes vor den gefährlichen Menschenblattern vorangestellt. Auch auf den monetären Nutzen wird hingewiesen, da man die Kinder nicht zum Schutz vor den Pocken aufs Land schicken musste, was einen erheblichen finanziellen Aufwand bedeutete.\\
\\
Die nächst größere Kategorie ist mit elf zugeordneten Argumenten jene der \textbf{Statistik}, im Sinne der modernen Statistik. Diese Argumente setzten im späten 19. Jahrhundert ein und werden sowohl im 20. als auch im 21. Jahrhundert zugunsten der verschiedenen Impfungen ins Feld geführt. Wobei im Vordergrund die Betonung des Rückgangs der betreffenden Krankheiten seit Einführung der Impfung steht.\\
\\
19 Argumente entfallen auf die Kategorie \textbf{Gesundheitsförderung}. Diese finden sich hauptsächlich im frühen 19. Jahrhundert (1801--1830), vereinzelt auch noch 2008 bis 2013. Sie beziehen sich auf die unterschiedlichsten Impfungen, deren positive Auswirkungen auf das Immunsystem und die allgemeine körperliche Konstitution. Im 21. Jahrhundert finden sich zudem vereinzelt Autoren, welche in der Impfung ein Schutzmittel gegen die ein oder andere Krebsart sehen.\\
\newpage
14 Argumente entfallen auf die Kategorie \textbf{Erfahrungswert}. Diese finden sich sowohl bei den frühen Autoren wie Husson (1801)\footnote{Husson, Historische und medizinische Untersuchungen über die Kuhpockenkrankheit, 1801.} als auch bei den späteren wie
Bernheim-Karrer (1922)\footnote{Jakob Bernheim-Karrer, Gesundheitspflege des Kindes, 2. Auflage, Zürich, 1922.} oder Munch (2013)\footnote{Theodor Munch, Der große Bluff. Irrwege und Lügen der Alternativmedizin, Berlin, 2013.}. Diese betonen in ihren Argumenten basierend auf allgemeinen oder persönlichen Erfahrungswerten, den Rückgang von Infektionskrankheiten, welcher ausschließlich auf die Impfungen zurückgeführt wird.\\
\\
Die verbleibenden 13 Argumente entfallen auf die Kategorie \textbf{einziges Mittel}. Sie werden sowohl von Autoren des 19. Jahrhunderts wie Husson (1801)\footnote{Husson, Historische und medizinische Untersuchungen über die Kuhpockenkrankheit, 1801.} als auch von jenen des 21. Jahrunderts (Hirte, 2008)\footnote{Martin Hirte, Impfen, Pro \& Contra. Das Handbuch für eine individuelle Impfentscheidung, München, 2008.} angeführt. Mit Bezug auf die Pocken- als auch auf die Tetanusimpfung betonen sie deren besondere Wirksamkeit als präventives Schutzmittel vor allem in Anbetracht fehlender, wirkungsvoller Behandlungsalternativen nach einer erfolgten Ansteckung.

\subsection{Auswertung der Impfgegnerargumente}
Auf der Contraseite konnte wie erwähnt gleichfalls eine Kategorie "`\textbf{Gewissen}"' erfasst werden. Mit drei Argumenten aus der zweiten Hälfte des 19. Jahrhunderts (Schreiber (1834)\footnote{Carl Schreiber, Band I: Gründe gegen die allgemeine Kuhpockenimpfung, 2. Auflage, o. A., 1834, Reprint Göttingen, 1998.}, Becker (1872)\footnote{Friedrich Becker, Impfen oder Nichtimpfen. Beitrag zur Lösung der grossen Tagesfrage über den Impfzwang und zur Behandlung der Blatternkrankheit, Berlin, 1872.} und Oidtmann (1879)\footnote{Heinrich Oidtmann, Die historische und statistische Misshandlung der Impf-Frage im Reichstage zu Berlin 1878, Wien, 1879.}) handelt es sich hierbei um die kleinste Gruppe der Gegenseite. Diese Autoren appellieren an ihre Leser und fordern von ihnen gesunden Menschenverstand und Vernunftrecht sowie einen Gang auf den Friedhof ein.\\
\\
Mit vier Aussagen ebenfalls zu den kleinen Gruppen zählen die Sparten \textbf{Kosten/Nutzen} und \textbf{Religions-/Naturgesetz}. Während erstere hier ausschließlich als Phänomen des 21. Jahrhunderts auftaucht, stammen letztere aus dem späten 19. oder frühen 20. Jahrhundert. Bei der ersten Kategorie geht es den Autoren um eine Risikoeinschätzung der Impfungen und der Frage, ob Kosten oder Nutzen höher zu bewerten sind. Die Autoren der zweiten hier genannten Kategorie lehnen die Impfung ab, da sie diese als Anmaßung empfinden und in ihr einen Hohn gegen die göttlichen Naturgesetze sehen.\\
\\
Mit 48 Aussagen vom 19. bis ins 21. Jahrhundert ist die Kategorie \textbf{Gesundheits-\\schädigung} die Größte  und bezieht damit alle Impfungen mit ein. Die Autoren sprechen von Schäden am Immunsystem und gefährlichen Nebenwirkungen, bis hin zu einem allgemeinen Verfall der Gesellschaft, ausgelöst durch die seit Generationen erfolgte Impfung. Für sie stellt Impfen eine wie auch immer geartete Gefahr dar, da sie davon ausgehen, dass Impfungen Krankheiten verschieben, verändern oder gar neue hervorbringen kann. Zeitgenössische Autoren des 21. Jahrhunderts sprechen auch von einem überflüssigen Unterfangen, da sie in den zu impfenden Krankheiten zum Beispiel Raritäten des heutigen Krankheitsspektrums sehen.\\
\\
Die zweitgrößte Kategorie \textbf{Fragwürdigkeit} beinhaltet immerhin 18 Argumente von Autoren aus dem 19., 20. und 21. Jahrhundert. Sie betreffen damit alle behandelten Impfungen. Es werden Fragen über Statistiken, Impfstoffherstellung, Impfmethoden, falsche respektive fehlende Studien und vieles mehr aufgeworfen. \\
\\
Eine gänzlich andere Einstellung zu diesem Thema haben jene Autoren, welche der Kategorie \textbf{ohne Wirkung} zugeordnet wurden. Diese 16 Argumente sprechen der Impfung jegliche Form der positiven Wirkung ab oder sind vom Fehlen wissenschaftlicher Beweise für die versprochene Wirkung überzeugt. Man  findet diese Argumente bei den ersten impfgegnerischen Autoren bis hinein in die Gegenwart. \\
\\
Ganz ähnlich betrachten die Angelegenheit die Autoren der Kategorie \textbf{Zwecklosigkeit}, für welche die Prozedur im besten Fall unnötig ist, da nicht jeder zwangsläufig die zu impfende Krankheit bekommen muss oder man den Kindern durch richtige Ernährung und langes Stillen genug Schutz bieten kann. Diese Einstellung findet man über alle drei betrachteten Jahrhunderte hinweg und sie betrifft damit Impfungen im allgemeinen.\\
\\
Mit elf zuordenbaren Argumenten befindet sich die Kategorie \textbf{Beeinträchtigung} zahlenmäßig in der unteren Hälfte der Argumentenliste. Auffallend ist, dass hier nur Argumente aus dem 20. und 21. Jahrhundert zugeordnet werden konnten. Diese Autoren empfinden Impfungen als einen Eingriff in den menschlichen Organismus, der dadurch natürliche Prozesse des Wachstums und der Entwicklung verhindert und den Betroffenen zum Beispiel anfälliger für Folgeerkrankungen macht.\\
\\
Zu Letzt stehen noch jene neun Argumente, welche hinter den Impfprogrammen "`\textbf{Machtinteressen}"' sehen. Sie werfen Pharmafirmen, Wissenschaft und Staat vor, dass diese Impfprogramme aus rein wirtschaftlichen Gründen verfolgt und die Gesundheitsprävention der Bevölkerung nur das eigentliche Interesse nach Profitmaximierung verschleiern soll. Diese Art der mitunter verschwörungstheoretischen Argumentation findet sich in den ausgewerteten Büchern im 20. und 21. Jahrhundert.




\section{Vergleich}
Es ist nun also gelungen, die Argumente für die Impfbefürworter als auch für die Impfgegner in jeweils neun Kategorien einzuteilen, welche im vorangegangenen Kapitel erläutert wurden. Was bei der Auswertung vermutlich als erstes ins Auge sticht, ist das bereits erwähnte Ungleichgewicht bei der Verteilung der Argumente auf die drei Jahrhunderte, wobei das 19. Jahrhundert mit 17 Argumenten die Mehrheit darstellt. Daraus den Schluss zu ziehen, dass sich die Menschen im 19. Jahrhundert mehr mit dem Thema Impfen auseinandergesetzt haben als jene im 20. und 21. Jahrhundert wäre allerdings falsch. Die Gründe für diese Unterschiedlichkeit in der Anzahl der Bücher wurden im Kapitel acht ausführlich erläutert und liegen unter anderem in der Zugänglichkeit und Verfügbarkeit wie auch den Inhalten der Werke, wonach nicht jeder Autor, der sich mit dem Thema Impfen befasst, in seinem Buch auch tatsächlich Argumente dafür oder dagegen liefert. Die erwähnte Problematik der Varianz an Publikationen in den unterschiedlichen Jahrhunderten erscheint hier besonders interessant und soll im Sinne des Vergleiches näher beleuchtet werden. \\
Kurz nach der Veröffentlichung von Edward Jenners Publikation über die Kuhpocken 1799 folgte zu Beginn des 19. Jahrhunderts eine ganze Reihe von Werken, die sich mit der Wirkung der Vaccination im Vergleich zur Inokulation und der darin mitschwingenden Hoffnung auf eine baldige Ausrottung der Blattern befassten. Das erste in dieser Arbeit aufgenommene Werk, welches sich gegen die Impfung aussprach und mit deren negativen Folgen auseinandersetzte, stammt aus dem Jahr 1834 (Carl Schreiber).\footnote{Die frühesten Hinweise auf impfgegnerische Werke sind: "`Johann Christian Ehrmann, Ueber den Kuhpockenschwindel bei Gelegenheit der abgenöthigten Verteidigung gegen die Brutalimpfmeistere, den Herrn Dr. und Hofrath Sömmering und den Herren Dr. Lehr, Frankfurt a. M., 1801"'. Johann Christian Ehrmann (1749--1827), deutscher Arzt und Satiriker. Das genannte Werk ist in der Niedersächsischen Staats- und Universitätsbibliothek einsehbar. "`Johann Valentin Müller, Beweis dass die Kuhpocken mit den natürlichen Kinderblattern in keiner Verbindung stehen, und also ihre Einimpfung kein untrügliches Verwahrungsmittel gegen die natürlichen Blattern sein könne, dem Publicum zur Beherzigung gewidmet, Frankfurt a. M., 1801"'. Johann Valentin Müller (1756--1813), deutscher Arzt. Dieses Werk ist unter anderem in der Berliner Staatsbibliothek oder der Universitätsbibliothek Kiel einsehbar.}
 Die Impffrage wurde zudem in medizinischen Fachzeitschriften wie der Wiener Medizinischen Wochenschrift oder den Zeitschriften der Impfgegnervereine thematisiert. Auch einige Karikaturen finden sich im 19. Jahrhundert zu diesem Thema. Nach dem Jahrhundertwechsel nahmen die Publikationen deutlich ab und finden sich nurmehr sehr spärlich.\footnote{Naheliegend ist hier die Vermutung, dass dieses "`Einbrechen"' an Impfpublikationen mit den Folgen des Nationalsozialismus und der problematischen Rolle der Medizin in dieser Zeit zusammenhängt. Hierüber können an dieser Stelle allerdings nur Mutmaßungen angestellt werden, dies ist ein Thema für eine eigene Forschungsarbeit.} Erst in der zweiten Hälfte des 20. Jahrhunderts, ab den 1980/90er Jahren entstanden wieder vermehrt Werke dazu. Möglicherweise führte die Ausweitung der kostenlosen Impfprogramme in Österreich zu einer stärkeren Auseinandersetzung mit dem Thema. Im 21. Jahrhundert nahm die Zahl der Publikationen deutlich zu und seit 2010 erscheinen beinahe im Jahrestakt eine Vielzahl neuer Werke. Mit steigender Popularität der Elternratgeber, den neuen Medien, des Fernsehens und des Internets lässt sich auch eine Verlagerung der Impfdebatte feststellen. Gerade das Internet und die sozialen Medien haben zu einer neuerlichen Verstärkung der Diskussion beigetragen. So wird das Thema auf zahllosen Homepages (Google liefert bei der Schlagwortsuche "`Impfen"' immerhin 2.250.000\footnote{Abfragedatum: 24.4.2017.} Millionen Ergebnisse) genauso diskutiert wie in frei zugänglichen Internetforen (hier vor allem so genannte "`Mütterforen"'), auf Facebook oder auf YouTube. Dazu kommt natürlich noch das Fernsehen und neuerdings sogar das Kino. Als jüngstes Beispiel kann hier der Film "`Vaxxed"' von Andrew Wakefield angeführt werden, welcher 2016 in den USA erschienen ist und dessen Filmstart in Europa kontrovers diskutiert und mitunter boykottiert wurde.\footnote{Umstrittene Doku, Martin Both vom 2.4.2017: \url{https://www.merkur.de/bayern/umstrittene-doku-vaxxed-laeuft-in-muenchen-grosses-kino-fuer-impfgegner-8069609.html} 12.5.2017.} Da dieses Feld der Impfdiskussion in den neuen Medien ein sehr spezielles und vor allem umfassendes ist, dessen Bearbeitung andere Forschungfragen und Methoden bedingen würde, konnte es in dieser Arbeit nicht eingehend berücksichtigt werden.\\
\\
Wie sieht es also bei den Kategorien selbst aus? Hier ist einerseits interessant, welche Ähnlichkeiten und Verschiedenheiten die Für- und Widerseite aufweisen, andererseits soll auch jede Seite für sich betrachtet werden, um zu sehen, ob und inwiefern sich die Art der Argumente und Ausführungen zu dieser Frage verändert haben.\\
Auf Anhieb können zunächst drei Kategorien ausgemacht werden, welche sich sowohl auf der Pro- als auch auf der Contraseite finden: \textbf{Gewissen}, \textbf{Kosten/Nutzen} und \textbf{Gesundheitsschädigung/-förderung}.\\
\\
Die Kategorie des \textbf{Gewissens} findet sich bei Befürwortern und Gegnern ausschließlich im 19. Jahrhundert. In beiden Fällen sollen hier in erster Linie die Eltern dazu animiert werden, ihre Kinder zum Schutz deren Lebens impfen oder eben nicht impfen zu lassen. Die Autoren führen den Eltern dabei die schlimmst mögliche Konsequenz ihres Handelns, nämlich die Verantwortung für den Tod des Kindes, vor Augen, welche man nicht nur vor sich und seinem Gewissen, sondern auch vor Gott nicht rechtfertigen kann. Hier werden mitunter sehr ausführliche Schilderungen über das Leid, welches die Blattern verursachen können angeführt, um das Argument noch zu unterstreichen: "`\textit{Wenn ihr denn aber noch anstehet, euren Kindern die Schutzblattern einimpfen zu lassen; so stellet euch izt im Gedanken jenen alle Augenblicke möglichen Zeitpunkt vor, wo eure Kindern von der Blatternseuche ergriffen da liegen werden, in wildem, heftigen Fraisen, [...] vom Scheitel bis zur Fußsohle voll Beulen, daß ihr nicht wisset, wo ihr sie angreifen sollt,  alles Augenlichtes beraubt, und überhaupt, mit einem Worte, kaum kenntlich mehr nach ihrer vorigen Gestalt. Stellet euch vor, wie ihr dann mit langem, zerrissenen Herzen bey dem Jammerlager [...] da stehen werdet, zwar bereit, ihnen zu helfen, aber unvermögend ihren Jammer zu stillen. [...]}."'\footnote{Kumpfhofer, Predigt von der Pflicht der Eltern, S. 14--15.}\\
Die frühen Impfgegner appellierten darüber hinaus im Sinne der Aufklärung an die menschliche Vernunft, welche in der Impfung -- und den damit einhergehenden gesetzlichen Zwangsmaßnahmen -- eine Verletzung der Integrität des Einzelnen sah: "`\textit{Weil das Impfzwanggesetz eine so beispiellose Beleidigung des gesunden Menschenverstandes enthält und jeden freidenkenden Arzt vor den Einwürfen vorurtheilsfreier Laien beschämt, darum werde ich nicht rasten, bis wenigstens bei uns Aerzten, die wir ja auch Kinder haben, die Freiheit der Wissenschaft im modernen Staat (Virchow) für uns Impfketzer wieder zurückerobert ist.}"'\footnote{Oidtmann, Die historische und statistische Misshandlung der Impf-Frage, S. 38.}\\
Was hier besonders ins Auge sticht, ist die Betonung des Werts des Einzelnen, weniger des Erwachsenen bezogen auf das Vernunftrecht, sondern vielmehr die Hervorhebung des Kindes als "`schützenswertes Wesen"'. Beides entspringt der neuen zeitgenössischen Strömung der Aufklärung, in welcher das Konstrukt "`Kindheit"' als eigene Lebensphase überhaupt erst entworfen wurde. Bis dahin galten Kinder als minderwertige Erwachsene, die ihren Beitrag zum Erhalt der Familie zu leisten hatten, soweit es ihre beschränkten Möglichkeiten erlaubten.
Philosophen wie John Locke (1632--1704)\footnote{\textit{Some thoughts concerning education} erschien 1683 erstmals anonym. Vgl: Theodor
Fritzsch, John Locke. Gedanken über Erziehung, Leipzig, 1920, Online in: \url{http://gutenberg.spiegel.de/buch/gedanken-uber-erziehung-6212/1}
12.5.2017.} oder Jean Jacques Rousseau (1712--1778)\footnote{\textit{Èmile ou De l'éducation} erstmals 1762 erschienen. Tatsächlich ein philosophisch-theoretisches Werk und kein praktisches Erziehungshandbuch. Winfried Böhm und Michel Soetard, Jean-
Jacques Rousseau. Der Pädagoge, Paderborn, 2012, S. 16--17.} zum Beispiel widmeten
sich als eine der Ersten dem Thema der Kindheit und der Erziehung und gelten als die Autoren, welche die Pädagogik grundlegend revolutionierten,
indem sie die Zeit der Kindheit als schützenswerte Lebensphase in den Mittelpunkt stellten.\footnote{Böhm und Soetard, Jean-
Jacques Rousseau, S. 79--88.} Diese Werke führten in den folgenden Jahrhunderten zu einer gänzlichen Neubetrachtung der
Bereiche Erziehung, Bildung und Ausbildung der Kinder, welche mehr und mehr an Wert gewannen und entsprechend mehr Aufmerksamkeit erhielten. Man
erkannte ihre Individualität und Persönlichkeit bereits in jungen Jahren an. Im ausgehenden 19. Jahrhundert setzte die Reformpädagogik mit Maria
Montessori (1870--1952)
\footnote{Erste Ärztin Italiens. Begründete eine eigene Pädagogik, die so genannte "`Montessori-Pädagogik. Sie plädierte
für einen achtsamen Umgang mit Kindern um diese in einer vorbereiteten Umgebung liebevoll und respektvoll zu begleiten. Vgl.: Maria Montessori:
\url{https://montessori.at/montessori/ms-paedagogik/} 11.9.2017 oder Ingeborg Becker-Textor (Hrg.), Maria Montessori. Zehn Grundsätze des
Erziehens, Freiburg, 2012.}
oder Johann Heinrich Pestalozzi (1746--1827)\footnote{Pestalozzi war ein Anhänger Rousseaus und nahm dessen
theoretisches Werk über die Erziehung sehr genau, was in der praktischen Anwendung bei seinem eigenen Sohn in einer Katastrophe endete. Vgl.:
Böhm, Soetard, Jean-Jacques Rousseau, S. 15. Er gilt weiters als Wegbereiter einer allgemeinen Schulbildung, unabhängig von Geschlecht oder
Herkunft der Kinder. Vgl.: Heinrich Pestalozzi: \url{http://www.heinrich-pestalozzi.de/} 11.9.2017 oder Daniel Tröhler, Johann Heinrich
Pestalozzi, Wien, 2008.}
 neue Maßstäbe, welche das Individuum Kind noch mehr betonten. Nach den Umbrüchen im Nationalsozialismus und der kulturellen Veränderungen nach 1960 nehmen Kinder heute einen besonderen Stellenwert in der Gesellschaft ein. Jede Mutter und jeder Vater nimmt für sich das Vorrecht in Anspruch, als Einziger für das Wohl seiner Kinder Sorge tragen zu können und stellt besonders die Individualität seines Kindes in den Vordergrund, welche in weiterer Folge auch bei einer medizinischen Behandlung oder der Impfentscheidung eingefordert wird. Schlagwörter wie "`demokratische Erziehung"' oder "`bedürfnisorientierte Erziehung"', welche die freie Entfaltung des Kindes fördern sollen, stehen im Mittelpunkt. Erziehungsratgeber gibt es im Überfluss und auch die Industrie hat das "`Produkt Kind"' schon lange für sich entdeckt. Viele Eltern begegnen ihren Kinder heute als gleichberechtigte Individuen oder bemühen sich um ein freundschaftliches Verhältnis auf Augenhöhe, was einige Fachleute mittlerweile äußerst kritisch betrachten, wie etwa das viel diskutierte Werk "`Warum unsere Kinder Tyrannen werden"'\footnote{Michael Winterhoff, Warum unsere Kinder Tyrannen werden oder: Die Abschaffung der Kindheit, Gütersloh, 2008.} bezeugt. Diese "`Entdeckung des Kindes"' kann nicht überschätzt werden, insofern ist es nicht verwunderlich, dass sich diese gesellschaftliche Tendenz in der Impffrage wiederfindet und der Schutz des Kindes auch heutzutage sowohl als Pro- als auch als Contraargument herangezogen wird.\\
\\
Bei der Kategorie \textbf{Kosten/Nutzen} fällt bei einem Vergleich der beiden Seiten sofort ins Auge, dass sich die Impfgegner der ausgewerteten Werke erst im 21. Jahrhundert dieses Argumentes bedienen, während sie sich bei den Befürwortern überwiegend im 19. Jahrhundert, bis 1901 finden. Auf der Contraseite wird entweder auf ein ohnehin geringes Krankheitsrisiko oder die Ungefährlichkeit der zu impfenden Krankheiten hingewiesen, welche in keiner Relation zu leichten oder schweren Neben- und Langzweitwirkungen durch die Impfung stehen, wie zum Beispiel Martin Hirte beschreibt: "`\textit{Die Impfung aller Säuglinge ist unnötig und teuer. Das Erkrankungsrisiko bei Kindern steht in keinem Verhältnis zu den potenziell schweren Nebenwirkungen.}"'\footnote{Hirte, Impfen, Pro \& Contra, S. 215.}
 Auch die hohen Kosten der allgemeinen Impfprogramme für das Gesundheitssystem werden betont. Bei den Befürwortern stehen die harmlosen Folgen der Kuhpockenimpfung im Vordergrund, wie etwa ein leichtes Unwohlsein nach der Impfung zugunsten der Vermeidung einer folgenschweren Blatternerkrankung. Das Risiko einer Übertragung von Syphilis oder Wundkrankheiten wird hier bei korrekter Anwendung der Prozedur ebenfalls als verschwindend gering eingestuft.
Auf den ersten Blick doch sehr unterschiedlich geht es sowohl bei Gegnern als auch Befürwortern um Überlegungen zur Schwere von Krankheiten und inwieweit die Impfung dagegen einen Schutz bietet oder für den Einzelnen und oder die Gesellschaft mehr Schaden als Nutzen oder umgekehrt nach sich zieht. Auch ein Hinweis auf den monetären Aspekt findet man auf der Proseite des frühen 19. Jahrhunderts bei de Carro, der auf den Kostenfaktor für die Eltern hinwies, wenn diese ihre Kinder zum Beispiel aufs Land senden mussten, um sie vor einer Pockenepidemie in der Stadt zu schützen oder sie nach einer Inokulation, ebenfalls zum Vermeiden einer Ansteckung der anderen Familienmitglieder, absondern wollten.\footnote{Carro, Impfung der Kuhpocken, S. 114.}\\
Erst durch das Gegenüberstellen beider Kategorien wird hier eine gesellschaftliche Veränderung sichtbar. Die frühen Autoren der Proseite im 19. Jahrhundert definieren die leichten Nebenwirkungen in Form von Unwohlsein oder eventueller Ansteckung mit einer schwereren Krankheit als verhältnismäßig harmlos im Vergleich zur Alternative, nämlich einer potentiell tödlichen Blatternerkrankung. Man war also bereit, ein geringeres Übel in Kauf zu nehmen, um ein Größeres auszuschließen. Doch durch die Verbesserung der Behandlungsmethoden, der Lebensbedingungen und den Rückgang schwerer Infektionskrankheiten veränderte sich diese Einstellung nachweisbar. Der Arzt Gustav Paul hält bereits 1901 fest: "`\textit{Eine Widerlegung der Behauptung der meist sehr temperamentvollen Impfgegner über Volksvergiftung durch die Impfung und über die Nutzlosigkeit derselben gegen die Blatternansteckung lässt sich ja in einer blatternfreien Zeit -- und gerade da sind die Impfgegner am lautesten -- durch Vorführung gegentheiliger Beweise am Krankenbette und in der Todtenkammer nicht liefern.}"'\footnote{Gustav Paul, Der Nutzen der Schutzpocken-Impfung. Vortrag gehalten am 30. März 1901 in der 87. Vollversammlung des Vereins für Kindergärten und Kinderbewahranstalten in Österreich, Wien, 1901, S. 15.} Heute, im 21. Jahrhundert, wo die Menschen so gesund sind wie noch nie zuvor, ist man weniger bereit, auch nur geringe Nebenwirkungen (und sei es nur Fieber oder schlaflose Nächte) in Kauf zu nehmen, um die Kinder vor einer nunmehr den meisten jungen Eltern weitgehend unbekannten Krankheit zu schützen. Möglicherweise noch in der Überzeugung, dass das Erkrankungsrisiko in keinem Verhältnis zu potentiellen Nebenwirkungen durch die Impfung steht.\footnote{Vgl.: Martin Hirte, Impfen, Pro \& Contra, S. 215.} Dieses Phänomen wird heute als "`Präventionsparadoxon"' bezeichnet, was sich wie folgt definiert:

 Es zirkuliert eine gefährliche, meist lebensbedrohliche Infektionskrankheit auf die die Entwicklung einer Impfung folgt. Es kommt zu einem Rückgang der Inzidenz. Die geringeren Erkrankungszahlen führen zu einer veränderten Wahrnehmung in der Bevölkerung, da es immer schwerer wird, die Gefährlichkeit dieser Krankheit plausibel zu vermitteln. Das bedeutet, dass die Risikowahrnehmung der Krankheit einerseits sinkt, potentielle Nebenwirkungen der Impfungen, egal in welchem Schweregrad, andererseits als gravierender eingestuft werden als sie vielleicht sind.\footnote{Präventionsparadoxon: \url{http://www.leitbegriffe.bzga.de/bot_angebote_idx-161.html} 14.5.2017.} Das führt dazu, dass die Impfraten sinken und die Frage nach dem "`warum eigentlich gegen eine Krankheit impfen, die es bei uns nicht mehr gibt"' im Raum steht ohne jedoch dabei zu hinterfragen, wie es zu dieser Situation gekommen ist.\\
\\

Die nächste namentlich ähnliche Gruppe ist jene, welche in der Impfung eine \textbf{Gesundheitsförderung} respektive \textbf{Schädigung} erblickt. Sie findet sich wie erwähnt in allen drei Jahrhunderten, stellt aber unter den Befürwortern mit 19 Argumenten eine mittlere Kategorie dar, während sie bei den Gegnern die Größte formiert. Die ersten Aussagen auf der Proseite stammen bereits aus den Jahren 1801 und 1803. Die Autoren sahen in den Kuhpocken eine vorteilhafte Veränderung der Konstitution und durch das einhergehende Fieber eine Art von Reinigung des Körpers, welche sogar die Lebensdauer verlängerte. Die Kuhpocken wurden nicht unbedingt als Krankheit, sondern vielmehr als Wohltat angesehen, welche imstande war, aus kränklichen und schwachen Kindern gesunde zu machen und sogar von Skropheln, Milchschorf, Kopfgrind und trockenem Husten zu heilen: "`\textit{Das Fieber, welches die Entwickelung der Kuhpockenkrankheit anzeigt, erhebt das Lebensprinzip, beugt veilleicht, durch die Bewegung, welche es in der thierischen Oekonomie erregt, einer gefährlichen Krankheit vor; sie führt eine heilsame Crise herbey, welche zu einer Art von Reinigung bestimmt; stellt im Individuum das durch so verschiedene Ursachen gestörte Gleichgewicht wieder her, und in diesem Sinne kann man sagen, daß sie die Wahrscheinlichkeit der Lebensdauer vermehrt.}"'\footnote{Husson, Historische und medizinische Untersuchungen über die Kuhpockenkrankheit, S. 139.}\\
Mögliche negative Auswirkungen der Vaccination kommen nicht zur Sprache. Auch einige Autoren der Gegenwart schreiben der Impfung eine positive Wirkung auf das Immunsystem zu, indem sie zum Beispiel das Allergierisiko reduzieren, wohingegen das Durchmachen echter Infektionskrankheiten, wie Masern, Mumps, Diphtherie oder Keuchhusten zu einer nachweislichen Schwächung der Konstitution führen und Kinder für Infekte anfälliger machen. Der Arzt Lenzen-Schulte (2008) will zudem aus Studien herausgelesen haben, dass die Pocken- und Tuberkuloseimpfung imstande sei, vor dem schwarzen Hautkrebs zu schützen: "`\textit{Zahlreiche Hinweise zeigen, dass Impfungen sogar einen Mehrwert für das Immunsystem haben. Einzelne Studien deuten darauf hin, dass die Pockenimpfung ebenso wie die Impfung gegen Tuberkulose vor dem Schwarzen Hautkrebs schützt.}"'\footnote{Lenzen-Schulte, Impfungen, S. 41.}\\
Die Impfgegner vermerken in diesem Zusammenhang -- wie nicht anders zu erwarten -- allerdings das Gegenteil, beginnend damit, dass die Kuhpocken selbst ja eine Krankheit darstellen, die über die Impfung künstlich im Menschen hervorgebracht wird. Der Arzt Carl Schreiber (1834) zum Beispiel sah  einen Widerspruch darin, dass man den Menschen zuerst absichtlich krank machen muss, um ihn vor einer anderen Krankheit zu schützen: "`\textit{Der Mensch wird also durch die Kuhpockenimpfung zuerst augenscheinlich krank gemacht, muß dann auf eine mehr oder weniger sichtbare Weise krank bleiben, um gegen die natürlichen Blattern einigermaßen geschützt zu seyn, und dies sind ihre unabänderlichen Folgen; aber es gibt auch noch andere größere, welche nicht in der Berechnung des Impfarztes liegen, und theils [...] als selbstständige Nachkrankheiten auftreten.}"'\footnote{Carl Schreiber, Band I: Gründe gegen die allgemeine Kuhpockenimpfung, 2. Auflage, o. A., 1834, Reprint Göttingen, 1998, S. 58--59.}\\
 Auch das Risiko der Übertragung anderer Krankheiten wie Krätze, Skropheln oder Syphilis wurden hier besonders betont. Der engagierteste Impfgegner des 19. Jahrhunderts, Dr. Carl Georg Gottlob Nittinger, sah in der Vaccination gar "`\textit{das größte Gift der Erde}"'\footnote{Nittinger, Gott und Abgott der Impfhexe, S. 29.}, welches im Stande ist, "`\textit{das ganze Menschengeschlecht zu vergiften}"'\footnote{Nittinger, Gott und Abgott der Impfhexe, S. 29.}. In seinen Büchern beschreibt er sehr anschaulich einen gesellschaftlichen Verfall, gekennzeichnet durch sinkende Volkszahlen, steigende Sterbeziffern, spürbares Herabsinken der Diensttüchtigkeit der Rekruten sowie das Verkommen der Menschen selbst, welches sich etwa in der Gesichtsfarbe äußert, die nunmehr "`\textit{Mulatten, Eskimos oder Affen}"'\footnote{Carl Georg Gottlob Nittinger, Das falsche Dogma von der Impfung und seine Rückwirkung auf Wissenschaft und Staat, München, 1857, S. 29.} gleiche. Der Arzt und Anhänger Nittingers, Germann (1875) bezeichnete die Vaccination gar als "`\textit{neuen Sündenfall der Menschheit}"'\footnote{Heinrich Friedrich Germann, Historisch-Kritische Studien über den jetzigen Stand der Impffrage, 2. Band, Leipzig, 1875, S. 24.}, welcher die Sterbefälle und den Typhus vermehrt hätte, ohne dem Menschen dabei zu nützen. Während Schreibers Argument absolut nachvollziehbar ist, erscheinen die Kommentare von Nittinger und Germann betreffend der Impfung sehr extrem und auch im 20. und 21. Jahrhundert finden sich in dieser Kategorie mit Autoren wie Buchwald und Schwarz radikalere Positionen. Sie sehen in der Impfung zum Beispiel eine gravierende Störung des Immunsystems, welche alleinige Ursache für diverse Gemüts- und Charakterschäden sowie Verhaltensauffälligkeiten sei. Daneben führte die Impfung zu einer Veränderung des natürlichen Ökosystems, wodurch Krankheiten und Viren verfälscht und neue Krankheiten erzeugt werden können. Der "`Impfgegnerpapst"' des 20. Jahrhunderts, Gerald Buchwald, sah in der Impfung zudem eine mögliche Ursache für die Ausbreitung von AIDS: \textit{"`Impfungen sind immer ein Eingriff ins Immunsystem, sie sind damit auch ein Eingriff in das Ökosystem. Die Menschheit lernt allmählich, welche Folgen es haben kann, in dieses, von der Natur so weise geschaffene Ökosystem einzugreifen. [...] Nicht geklärt ist heute die drängende Frage, ob die jahrzehntelange Unterdrückung von Krankheiten zu einer gefährlichen Schwächung der Abwehrkräfte geführt und damit die Verbreitung der Immunschwächekrankheit AIDS mit ermöglicht hat."'}\footnote{Gerhard Buchwald, Impfen. Das Geschäft mit der Angst, 3. Auflage, Lahnstein, 1995, S. 177. \textit{Anmerkung der Autorin: Dieses Werk ist als wissenschaftlich fragwürdig einzustufen.}} \\
Andere Impfgegner des 21. Jahrhunderts versuchen den Zusammenhang zwischen Impfung und Zivilisationskrankheiten wie Diabetes herzustellen oder schildern eine Vielzahl an möglichen Impfschäden, welche sich in einem Spektrum von Schläfrigkeit bis hin zum plötzlichen Kindstod bewegen: "`\textit{Impfschäden können also durch verschiedene Inhaltsstoffe ausgelöst werden. Abgesehen von vorübergehenden lokalen Wirkungen an der Einstichstelle gibt es unterschiedlich heftige Reaktionen wie Fieber, unstillbares schrilles Schreien (schwere neurologische Störung), Appetitverlust, Schläfrigkeit, keuchhustenartiger Husten, Lugenentzündung, Durchfall, Hautausschlag, Krampfanfälle, Asthmaanfälle, Nervenentzündungen, Thrombozytopenie, Gehirnentzündung, allergischer Schock, plötzlicher Kindstod und vieles mehr.}"'\footnote{Schwarz, Impfen -- eine verborgene Gefahr, S. 49.}\\
\\
Versucht man nun die Kategorien \textbf{Gesundheitsschädigung} und \textbf{Gesundheitsförderung} zu vergleichen, werden sofort die Gegensätze offensichtlich. Was für die einen ein Segen ist, welcher im Stande zu sein scheint wahre Wunder zu vollbringen, ist für die anderen der denkbar schlimmste Weg in den menschlichen Organismus einzugreifen. Gemeinsam ist ihnen, dass auf beiden Seiten extreme Positionen auftauchen. Für die meisten Autoren dieser Kategorien gibt es nur "`Schwarz oder Weiß"', die Impfung ist entweder so gut, dass Sie Krebs vorbeugen kann oder so schlecht, dass Sie AIDS, Krebs und andere Krankheiten verbreitet. \\
\\
Eine sehr ähnliche Kategorie, welche sich hier direkt im Vergleich anschließen lässt, ist jene, welche als \textbf{Beeinträchtigung} bezeichnet wurde. Die hier zusammengefassten Autoren gehen davon aus, dass eine Impfung die natürliche Entwicklung des menschlichen, respektive kindlichen Organismus nachteilig beeinflusst oder die Ausrottung der Krankheit durch die Impfung gefährdet. Das Durchmachen der \glqq harmlosen Kinderkrankheiten\grqq senkt hingegen das Risiko für Allergien, Diabetes oder Krebs: "`\textit{Die allgemein mild verlaufende Erkrankung [Mumps] lässt sich homöopathisch gut behandeln, wie alle anderen Infektionskrankheiten auch (laut Gesetz nur Ärzten gestattet). Durch die Impfung kann es u.a. zu Allergien, Ohrspeicheldrüsenentzündungen, Hirnhautentzündungen, Bauchspeicheldrüsenentzündungen, Schwerhörigkeit und Diabetes Typ 1 kommen, sogar Todesfälle wurden bekannt. Langzeitfolgen sind kaum untersucht. Eine durchgemachte Mumpserkrankung dagegen fördert die Entwicklung und Reifung des Kindes und verringert das Risiko, später an Krebs, speziell der Eierstöcke, sowie an Multipler Sklerose zu erkranken}."'\footnote{Schwarz, Impfen -- eine verborgene Gefahr, 101.}\\
Diese Argumente sind zwar jenen der vorherigen Contrakategorie \textbf{Gesundheitsschädigung} durchaus ähnlich, unterscheiden sich aber insofern, als dass bei der Kategorie \textbf{Beeinträchtigung} die unmittelbaren Folgen der Impfung, im Sinne von direkten Nebenwirkungen (Unwohlsein, Fieber, bis hin zu neurologischen Störungen oder ähnlichem) im Fokus stehen, während Erstere ihr Augenmerk auf längerfristige Nachteile legen, wie eben das Fehlen einer natürlichen Immunität, die zudem vor Diabetes, Krebs oder Allergien schützen würde, was die Impfungen schlichtweg verhindern. Was damit ebenfalls beiden Kategorien gemeinsam ist, ist ein sehr ähnliches Verständnis von Krankheit. Die zu impfenden Krankheiten werden nicht als gefährlich wahrgenommen, sondern als Teil eines natürlichen Entwicklungsprozesses betrachtet. Durch die richtige Behandlung der jeweiligen Krankheit geht der Betroffene gereift und gestärkt hervor. Hier kann man einhaken und die Frage stellen: Was wird in diesem Fall als "`richtige Behandlung"' definiert? Für die Autoren der Kategorien \textbf{Gesundheitsschädigung} und \textbf{Beeinträchtigung} liegt die "`richtige Therapie"' überwiegend in der Homöopathie, wodurch sich in diesen Argumenten hier generell eine Tendenz zur Ablehnung der klassischen Schulmedizin und herkömmlichen Präventiv- und Heilmittel sowie zum damit einhergehenden Krankheitsverständnis herauslesen lässt.\\
\\
Weiters lässt sich hier die Kategorie \textbf{Zwecklosigkeit} der Contraseite anschließen. Diese Argumente wollen belegen, dass die Impfung keinen Gewinn für den Menschen bringt, da man zum Beispiel gar nicht davon ausgehen kann, dass auch wirklich jeder die entsprechende Krankheit bekommen würde, wie Carl Schreiber bereits 1834 feststellte: "`\textit{Die Voraussetzung, daß fast jeder Mensch die Pocken bekommen müsse und von ihren Gefahren bedroht werde, ist nicht begründet. Es gibt Gegenden, die verschont geblieben sind [...]. Selbst da, wo sie gewühtet haben, ist ein Theil der Menschen nicht von ihnen angesteckt worden.}"'\footnote{Schreiber, Gründe gegen die allgemeine Kuhpockenimfpung,  S. 12--13.}\\
Andere wiederum wollen belegen, dass man alleine durch langes Stillen und richtige, gesunde Ernährung (vermeiden von Einfachzucker) die Ansteckung mit bestimmten Krankheiten wie Poliomyelitis verhindern kann: "`\textit{Bei Vermeidung von raffinierten Kohlenhydragen, d.h. Fabrikzucker und Auszugsmehlen ist eine Ansteckung mit dem Kinderlähmungsvirus nicht möglich.}"'\footnote{Max Bruker, Ilse Gutjahr, Biologischer Ratgeber für Mutter und Kind, 5. Auflage, Lahnstein, 1987, S. 241. \textit{Anmerkung der Autorin: Dieses Werk ist als wissenschaftlich fragwürdig einzustufen.}} Zeitgenössische Autoren des 21. Jahrhunderts wie der Heilpraktiker Rudolf Schwarz und der Arzt Friedrich Graf stellen, ähnlich wie in der vorherigen Kategorie, die Gefährlichkeit der zu impfenden Krankheit wie HIB, Rotaviren oder Röteln in Frage. Eine Ansteckung kann durch einen achtsamen und aufgeklärten Umgang vermieden werden und ist bei einem inneren Gleichgewicht, wo der Körper im "`\textit{Frieden mit den inneren und äußeren Keimen ist}"'\footnote{Graf, Die Impfentscheidung, S. 95--96.} so gut wie ausgeschlossen. Was hier im Vergleich zu den vorhergegangenen beiden Kategorien \textbf{Beeinträchtigung} und \textbf{Gesundheitsschädigung} fehlt, ist der Verweis auf etwaige negative Folgen durch die Impfung. Im Gegenteil dazu steht hier die Vermeidung der Ansteckung und in Folge dazu der Impfung im Vordergrund. Das Krankheitsverständnis und die Ablehnung der Ideen der Schulmedizin decken sich hingegen mit jenen der vorhergegangenen Kategorien. Auffallend ist hier auch, dass sich die Kategorien \textbf{Zwecklosigkeit} und \textbf{Beeinträchtigung} hauptsächlich im 20., mehr noch im 21. Jahrhundert finden. \\
\\
Möchte man diesen Impfgegnern Argumente der Proseite gegenüberstellen, bietet sich zunächst die Kategorie \textbf{Sicherheit} an. Diese größte Kategorie der Befürworter umfasst alle Impfungen in allen drei behandelten Jahrhunderten. Die Ersten Argumente finden sich hier bereits mit Husson (1801) und Aikin (1802) und deren Ausführungen über die Sicherheit der Kuhpockenimpfung, vor allem im Vergleich zur bisher erfolgten Blatterninokulation. Das Ausbleiben von Folgeerkrankungen wurde hier ebenso betont wie die Tatsache, dass der Mensch an den Kuhpocken selbst nicht sterben kann. Auch der bessere Allgemeinzustand nach erfolgter Impfung wird mehrmals erwähnt. Es finden sich aber auch Hinweise, dass bei einer unbedachten Auswahl der Kinder, welche man zur Impfstoffabnahme heranzog, die Möglichkeit der Ansteckung mit Syphilis oder anderen Krankheiten bestand oder vereinzelt Menschen trotz Kuhpockenimpfung an den echten Blattern erkrankten, jedoch in einer modifizierten, leichteren und daher gefahrloseren Form. Die Verbreitung von Krankheiten durch die Impfung wird auch in Argumenten des späteren 19. Jahrhunderts thematisiert, jedoch relativiert und zwar mit dem Hinweis auf die eine neue Art der Impfstoffgewinnung aus Tierlymphen. Interessanterweise stammt die Mehrzahl an Aussagen der Kategorie "`Sicherheit"' aus dem 19. Jahrhundert. Für das 20. und 21. Jahrhundert konnten hier gerade einmal fünf Argumente zugeordnet werden. Diese wollen aufzuzeigen, das Impfungen keinen Autismus, keine Autoimmunkrankheiten oder Allergien auslösen, sondern sich positiv auf das Immunsystem auswirken: "`\textit{Mehrere Untersuchungen an Schülern weisen nicht nur nach, dass kein Zusammenhang zwischen Allergien und Impfung besteht. Eher ist es umgekehrt. [...] Impfungen schwächen nicht die Abwehr, Schüren keine Autoimmunkrankheiten. [...] Der schwerwiegende Vorwurf, die Impfung könnte Autismus begünstigen, stellte sich als wissenschaftlicher Betrug heraus.}"'\footnote{Lenzen-Schulte, Impfungen, S. 39, 40 u. 92.}
\\
Stellt man diese Kategorie nun den vorangegangenen der Impfgegnerseite gegenüber, so wird hier einerseits ein klassisches Krankheitsbild gemäß der traditionellen Schulmedizin offensichtlich, was sich stark von den alternativen Krankheitsbildern und der darin enthaltenen Systemkritik der Impfgegner unterscheidet. Andererseits fällt auf, dass die Impfbefürworter der Kategorie \textbf{Sicherheit} sehr wohl auch mögliche negative Wirkungen der Impfung in Erwägung ziehen, sie aber gleichzeitig durch die Betonung des korrekten Prozederes reduzieren, beinahe ausschließen, wohingegen positive Auswirkungen der Impfung in den vorherigen Kategorien der Impfgegnerseite keine Erwähnung finden. Außerdem zeigt sich, dass die Argumente der Kategorie \textbf{Sicherheit} weniger extrem sind, was nach den bisherigen Beobachtung eher eine Ausnahme zu sein scheint. Hier werden durch die Impfung keine Wunder erwartet und Nebenwirkungen durchaus eingestanden, oft jedoch auf ein nachlässiges Vorgehen bei der der Impfstoffgewinnung reduziert.\\
\\
Direkt daran anschließen lässt sich die Kategorie \textbf{Erfahrungswert} der Proseite. Auch diese Autoren sind über die Jahrhunderte hinweg bemüht, die Wirksamkeit und Sicherheit der Impfungen zu betonen, wobei der persönliche oder allgemein wissenschaftliche Erfahrungswert im Vordergrund steht.
So wird bereits bei Husson (1801) darauf verwiesen: \textit{"`Man kennt das WIE nicht, aber man weis daß die Sache ganz sicher ist. Man würde fruchtlos sich mit weitläuftigen (sic!) Untersuchungen über die Art, wie die Kuhpocken in uns die Fähigkeit, die Kinderblattern zu bekommen, zerstören können, erschöpfen: [...] es ist genug, daß die Thatsache wahr, und durch zahlreiche Erfahrungen bestätigt ist."'}\footnote{Husson, Historische und medizinische Untersuchungen,  S. 159.} Andere Autoren wie Franz Seraph Giel, bayrischer Zentralimpfarzt, oder Gustav Paul, k.k. Amtsarzt in Böhmen, bringen ihren persönlichen, positiven Erfahrungswert aus ihrer jeweiligen Tätigkeit ein.\footnote{Vgl.: Franz Seraph Giel, Die Schutzpocken=Impfung in Bayern, vom Anbeginn ihrer Entstehung und gesetzlichen Einführung  bis auf gegenwärtige Zeit. Dann mit besonderer Beobachtung derselben in auswärtigen Staaten, München, 1830, S. 20 und Paul, der Nutzen der Schutzpocken-Impfung S. 18.} Etwaige negative Folgen einer Impfung finden hier keinen Platz, womit sich diese Kategorie wieder an die bereits vorher festgestellte "`Schwarz-Weiß-Malerei"' anschließt.\\
\\
Die \textbf{Statistik} ist eine weitere Form der Argumentation der Befürworter. Im Mittelpunkt stehen hier vor allem jene Zahlen, welche den Rückgang gefährlicher Infektionskrankheiten sowie eine verringerte Sterblichkeitsrate an ebendiesen bezeichnen: "`\textit{Es ist heute bewiesen, daß uns mit der Diphtherieimpfung eine wirksame Bekämpfung der Diphtherie als Seuche möglich ist. Die Statistiken der Weltliteratur lassen erkennen, daß wir bei den Ungeimpften 5--10 mal häufiger als bei Geimpften mit Erkrankungen zu rechenen haben. Je nach Schwere der Epidemie ist die Letalität der Nichtgeimpften 2--11mal größer als bei Geimpften. Besonders gute Erfolge sind zu erwarten, wenn 70\% der Bevölkerung und mehr geimpft worden sind. Die Diphtherieimpfung kann heute als weitgehend ungefährlich und sehr wirksam bezeichnet werden.}"'\footnote{Heinz Spiess, Schutzimpfungen, Stuttgart, 1958, S. 37.}\\
Die mit elf Argumenten eher kleinere Kategorie findet sich ab dem ausgehenden 19. Jahrhundert. Denn, wenn gleich es bereits frühe Belege von statistischen Erfassungen von Steueraufkommen oder Bevölkerungszahlen gibt, ist die moderne Statistik, welche sich in den Argumenten widerspiegelt, eine Entwicklung des 19. Jahrhunderts, indem es dem Belgier Adolphe Quételet (1796--1874) erstmals gelang, aus der Befragung weniger eine Prognose für viele abzugeben.\footnote{Geschichte der Statistik: \url{https://de.statista.com/statistik/lexikon/definition/154/statistik_fuer_anfaenger_geschichte_der_statistik/} 25.4.2017.} Es ero"ffnete sich damit eine ganz neue Art und Weise der Argumentation, vor allem wenn man diese der Kategorie \textbf{Erfahrungswert} gegenüberstellt. Erfahrung ist bekanntlich etwas sehr subjektives, überhaupt wenn es um persönliche Erlebnisse, zum Beispiel während einer Epidemie, geht. Beim Anführen von Statistiken gewinnen die Argumente automatisch an wissenschaftlichem Gewicht, wodurch sie sich nicht so leicht entkräften lassen. \\
\\
Soweit zur Theorie, denn einige impfgegnerische Autoren versuchen dies dennoch. Diese Argumente wurden in der Kategorie \textbf{Fragwürdigkeit}
zusammengefasst. So stellte Nittinger (1863) etwa das Problem in den Raum: "`\textit{So schwierig es für die Chemie ist, Thiergifte zu prüfen,
so schwer ist es für das blose Auge, das Gift des Impfstoffes zu erkennen. Das "`ächte Jenner'sche Bläschen"', von dem der Stoff entnommen
werden soll, trägt keine specifischen Merkmale an sich und kein noch so "`gebildeter"', noch so "`gewissenhafter, noch so "`aufmerksamer"', noch
so "`wohlerzogener"' "`duly educated medical practitioner des englischen Blaubuchs"' kann sicher sein, ob er trotz aller Vorkehrungen statt der
ächten Jenner'schen Lymphe nicht den Mißgriff begehe, ein anderes Krankheitsprodukt, den Keim für syphilitische, scrofulose, kräzige, flechtige,
gichtische oder andere konstitutionelle Krankheiten
%durch die Vaccination zu übertragen.}"'
durch die Vaccination zu übertragen.}"'
\footnote{Nittinger, Gott und Abgott oder die
Impfhexe, S. 30.} Weiters stellt er den ordentlich gebildeten Impfarzt in Frage, wenn er folgert: "`\textit{Was sollen wir endlich zu dem
ordentlich gebildeten Impfarzt [...] sagen? Wo ist die Hochschule, welche die 542 Autoritäten zu Impfärzten erzog? [...] Wo haben sie ihr
Impfexamen erstanden? [...] Sie haben alle keine Erziehung zu Impfärzten genossen, sie sind so weit die Sonne scheint
Autodidakten.}"'\footnote{Nittinger, Gott und Abgott oder die Impfhexe, S. 63.} \\
Nittingers Zeitgenosse Oitdmann (1879) stellt die Glaubwürdigkeit des Impfsystemes in anderer Art und Weise in Frage, da diese bisher nur von
Ärzten untersucht worden sei. Die Prüfung der Angelegenheit durch eine nichtärztliche Kommission werde zwangsläufig zu einem anderen Ergebnis kommen, so Oitdmann. Wilhelm Ressel, ein Arzt des frühen 20. Jahrhunderts, geht sogar soweit zu behaupten, dass die Pocken, genau wie Pest und Cholera schon längst verschwunden wären, würde man die Pockenviren durch die Impfung nicht künstlich weiter im Umlauf halten: "`\textit{Wir Impfgegner behaupten sogar, und wohl mit Recht, daß die Pocken, genau wie Cholera und Pest, schon längst verschwunden wären, wenn man auch gegen die Pocken nicht impfen, ins Blut der Menschen nicht immer von neuem "`animalen"' Pockeneiter=Samen streute, es so für die Pocken empfänglicher machend. Wer Brennesselsamen sät, kann doch nur Brennesseln immer wieder ernten. Wer Schirlings=Unkrautsamen streut, dem wächst ein Schirlingsbeet. Und dem Impfgeschäft sollte das sich stets treu bleibende Naturgesetz eine -- fette Extrawurst braten und ihm aus Pockeneitersamen Gesundheit erblühen lassen?}"'\footnote{Ressel, Das Impfgeschäft, S. 13.} Weiters kritisiert er die "`\textit{schamlose Verlogenheit des Impfgeschäftes}"'\footnote{Ressel, Das Impfgeschäft, S. 29.}, welches zuerst die eine Impfmethode lobte, sie daraufhin verbot\footnote{Anm: zuerst die Inokulation, welche durch die Vaccination ersetzt wurde. In weitere Folge die Impfung von Arm zu Arm, welche durch tierische Lymphen getauscht wurde.} und die Schutzkraft der Impfung von lebenslang auf mittlerweile fünf Jahre reduzierte.\footnote{Ressel, Das Impfgeschäft, S. 30.}\\
Die Autoren des 21. Jahrhunderts bezweifeln wiederum vorher erwähnte Statistiken und Studien der Befürworter, welche in den Augen der Impfgegner weder objektiv noch ausgewogen sind, da etwa die Studienzeiträume viel zu kurz angelegt sind und eine grundsätzliche Langzeitstudie von geimpften und ungeimpften Kindern komplett fehlt. Es wird auch die Behauptung aufgestellt, dass Ärzte zu der am  wenigsten geimpften Bevölkerungsgruppe gehört.\\
Während also die Autoren der vorherigen Kategorie \textbf{Statistik} bemüht sind, dem Leser größtmögliche Sicherheit und Vertrauen zu vermitteln, wollen die Impfgegner der Letzeren genau diese in Frage stellen, indem sie das System selbst, die Impfstoffherstellung, Statistiken und Studien in Zweifel ziehen, was auch als Systemkritik im allgemeinen betrachtet werden kann.\\
\newpage
Nach den bisherigen Vergleichen verbleiben auf beiden Seiten noch drei Kategorien. Hier lassen sich einerseits jene der \textbf{Machtinteressen} mit \textbf{Obrigkeit} und andererseits \textbf{Religions-/Naturgesetz} mit \textbf{Religion} sowie \textbf{ohne Wirkung} und \textbf{Einziges Mittel} gegenüberstellen.\\
\\
Mit fünf Aussagen stellt die Kategorie \textbf{Obrigkeit} eine der kleinsten der Befürworter dar und begrenzt sich auf das frühe 19. Jahrhundert (bis 1820). Die Autoren wenden sich direkt an die Eltern, um ihnen die Obrigkeit, sprich den Landesfürsten, König oder Kaiser als positives Beispiel vor Augen zu führen. Denn wäre die fürstliche Regierung und der väterliche Landesherr nicht absolut vom positiven Nutzen der Vaccination überzeugt, würde er seine treuen Untertanen gewiss nicht auffordern, sich der Prozedur zu unterziehen, geschweige denn selbst seine eigenen Kinder impfen lassen, wie Kaiser Franz I. Welcher treue, gottesfürchtige Gefolgsmann kann sich hier noch verweigern? Unter den Zeitgenossen von d'Outrepont, Kumpfhofer und Krauss niemand wie es scheint. Denn die "`passende"' Gegenkategorie \textbf{Machtinteressen} setzt in den ausgewerteten Quellen erst im späten 20. Jahrhundert ein. Diese Impfgegner sehen in der Impfindustrie eine große gewinnorientierte Interessengemeinschaft, welche sich aus den Pharmafirmen, der Wissenschaft und dem Staat zusammensetzt und die Impfung einzig aus finanzieller Profitgier betreibt. Dies sogar in dem Wissen, dass die Prozedur für die Menschen im besten Fall wirkungslos, im schlimmsten Fall jedoch negative Folgen hat. Der Heilpraktiker Rolf Schwarz geht sogar soweit, der Pharmaindustrie eine gezielte Schädigung der Bevölkerung zu unterstellen: \textit{"`Die 1995 neu eingeführten azellulären [Anm: Keuchhusten] Impfstoffe sollen weniger milde Nebenwirkungen haben. Die schweren Komplikationen wie Krampfanfälle, Gehirnentzündungen und Lähmungen sind jedoch nicht zurückgegangen, in einer schwedischen Studie traten sie sogar vermehrt auf. Auch Asthma und Allergien kommen bei geimpften Kindern fünfmal häufiger vor als bei Ungeimpften. Die Pharmaindustrie benutzt ja den Keuchhusten-Impfstoff, um experimentell Allergien zu erzeugen!"'}\footnote{Schwarz, Impfen -- eine verborgene Gefahr, S. 84.} Weitere Autoren dieser Kategorie sind der Überzeugung, dass die Wissenschaft von der Pharmaindustrie finanziert wird und im Gegenzug entsprechende Ergebnisse liefert. Die staatlich geförderten Werbekampagnen für Impfprogramme dienen in weiterer Folge einzig und alleine dem Zweck, Angst zu erzeugen um die Impfmotivation zu steigern.\\
Die Befürworter der ersten Kategorie \textbf{Obrigkeit} plädieren also für absoluten Gehorsam gegenüber den Herrschenden. Dieser manifestiert sich dabei in einem absoluten Gott- oder Systemvertrauen dahingehend, dass die Regierenden niemals eine negative Entscheidung für ihre Untertanen treffen würden, um es etwas übertrieben zu formulieren. Die Contraseite der Kategorien \textbf{Fragwürdigkeit} und \textbf{Machtinteressen} hingegen stellen genau dieses, in ihren Augen blinde Gott- oder Systemvertrauen, in Frage. Neben berechtigter Systemkritik finden sich hier wie gezeigt auch Verschwörungstheorien, welche alles, was vom herrschenden System kommt, unbedingt in Zweifel ziehen und als falsch entlarven wollen.\\
Wie lässt sich dieser extreme Unterschied, welcher im Vergleich der beiden Kategorien ersichtlich wird, erklären? Natürlich wäre es vermessen zu behaupten, dass es zu Lebzeiten von Kumpfhofer, Krauss oder d`Outrepont keine Systemkritik gegeben hätte, immerhin waren dies die Jahrzehnte nach der französischen Revolution. Ein Grund für diesen Unterschied zwischen der öffentlichen und vor allem schriftlichen Systemkritik des 19. und des 21. Jahrhunderts kann in der Zensur gesehen werden: Die Idee der Kommunikationskontrolle durch eine staatliche oder religiöse Obrigkeit ist nichts neues, sondern findet sich bereits in der Antike, gewann aber durch den Buchdruck und die dadurch rasch gestiegene Zahl an Publikationen enorm an Bedeutung. Zunächst eher von kirchlicher Seite praktiziert, entdeckte auch bald die staatliche Obrigkeit deren Nutzen für sich. Die Zensur diente einerseits der Meinungskontrolle nach innen aber auch der Blockade "`gefährlicher Ideen"' von außen.\footnote{Jürgen Wilke, Zensur und Pressefreiheit, in: Europäischer Geschichte Online (EGO), Mainz, 2013: \url{http://ieg-ego.eu/de/threads/europaeische-medien/zensur-und-pressefreiheit-in-europa} 2.5.2017. } So verwundert es auch nicht, dass es nach der vorläufigen Lockerung der Zensur unter Joseph II. in der Folge der französischen Revolution wieder zu verschärften Reglementierungen kam, welche nach dem Wiener Kongress 1814/15 unter Staatskanzler Clemens Lothar Wenzel Fürst von Metternich (1773--1859) ihren vorläufigen Höhepunkt erreichte und nahezu alle Lebens- und auch Wissenschaftsbereiche umfasste.\footnote{Zensur in Österreich: \url{https://austria-forum.org/af/Wissenssammlungen/Essays/Medien/Pre\%C3\%9Ffrechheit_und_Zensur} 2.5.2017.} Ein Publikation ohne Pseudonym, mit einem so direkten Angriff auf den Staat oder die Obrigkeit wie bei Hirte (2008)\footnote{Hirte, Impfen, Pro \& Contra, 2008.} Graf (2013)\footnote{Graf, Die Impfentscheidung, 2013.} oder Buchwald (1995)\footnote{Buchwald, Impfen. Das Geschäft mit der Angst, 1995.} wäre unter diesen Umständen undenkbar, wenn nicht sogar lebensgefährlich gewesen.\\
\\
Die \textbf{Religion} als Argument bemühen bei den zum Vergleich herangezogenen Befürwortern lediglich zwei frühe Autoren, der Arzt d'Outrepont (1803)\footnote{d'Outrepont, Belehrung des Landvolkes, 1803.} und der Linzer Pfarrer Kumpfhofer (1808)\footnote{Kumpfhofer, Predigt von der Pflicht der Eltern, 1808.}. Bei der abgedruckten Predigt von Johann Kumpfhofer handelt es sich nun um ein ganz besonderes Werk, sie ist nämlich eine der wenigen publizierten Pockenpredigten\footnote{Die Pockenpredigt von Kumpfhofer dient hier exemplarisch als Beispiel für die "`Blatternpredigten"' des 19. Jahrhunderts. Ein weiteres Beispiel ist die Predigtsammlung des Eferdinger Stadtpfarrers Matthäus Priegl, Predigten zur Empfehlung der Blattern=Einimpfung, Krems, 1817. Ebenfalls angeführt werden kann der Berndorfer Pfarr-Koadjutor Gregor Krämer, Predigt zur Verhütung der Blatternpest, gehalten am Feste des heiligen Joseph, Salzburg, 1802. Vgl. hierzu: Pammer, Vom Beichtzettel zum Impfzeugnis, S. 21. u. 27 und Falk, Weiss, "`Hier sind die Blattern"', S. 163.}. Was hatte es damit auf sich? Nach Einführung der Impfung, zunächst in Niederösterreich durch Ferro und de Carro, folgten einige größere, von der Landesregierung organisierte Versuche mit der Kuhpockenimpfung in Kranken- und Findelhäuser. Diese überzeugten die zuständigen Behörden von der Wirksamkeit der Methode. In weiterer Folge entschieden sich die Regierenden der Habsburgermonarchie für die Organisation und Verbreitung der Impfung im großen Stil. Damit mussten einerseits die Ärzte instruiert und informiert werden und andererseits die Bevölkerung dazu ermutigt werden, sich diesem neuen Verfahren zu unterziehen, was sich als weit schwierigere Aufgabe herausstellte. Eine Möglichkeit hierzu war die Propaganda über die Kanzel: Die Seelsorger wurden von der Landesregierung dazu angehalten, mehrmals im Jahr über die Pflichten der Eltern gegenüber ihren Kindern, besonders in Hinblick auf deren Gesundheitspflege und Bewahrung vor den Pocken durch die Impfung zu predigen.\footnote{Pammer, Vom Beichtzettel zum Impfzeugnis, S. 13. u. 15.} Die Nutzung der Pfarren für religionsferne Zwecke war seit der Pfarregulierung unter Joseph II. bereits gängige Praxis. Die Kanzel diente dazu, Verordnungen und sonstige Maßnahmen der Bevölkerung bekannt zu machen. Bereits bei der Verbreitung der Variolation wurden die Priester in die Pflicht genommen.\footnote{Pammer, Vom Beichtzettel zum Impfzeugnis, S. 18.}\\
Auf der Gegnerseite finden sich im späten 19. Jahrhundert in einer kleinen Kategorie \textbf{Religions-/Naturgesetz} ebenfalls vier Aussagen, welche Gott oder das göttliche Naturgesetz gegen die Impfung ins Feld führen. Von diesen Autoren wird es gerade zu als anmaßend verstanden, dass der Mensch glaubte, er könne mittels der Impfung eine Seuche ausrotten. Dies sei alleine Gottes Sache, er habe die Seuchen geschickt, er allein könne sie wieder wegnehmen: "`\textit{Wer die kleinen Epidemieen, Schnupfen, Scharlach, Grippe u. a. nicht verbannen kann, verbannt auch den Fürsten aller Seuchen, die Blattern nicht, das ist nicht Menschenwerk der Jennerischen Sekte, sondern, Gottes Sache. [...] Zum Vernichten wie zum Schaffen gehört ein Gott und es verräth eine ungemeine Kindlichkeit zu glauben, Jenner sei der Gottsohn gewesen, welche die Blattern vernichtet, ausgerottet, verbannt habe. Der Impfschutz ist vor Gott nicht möglich!}"'\footnote{Nittinger, Das falsche Dogma, S. 37 u. 39.}\\
Auch Anfang des 20. Jahrhunderts findet sich noch ein Autor, welcher die Impfung als "`\textit{Hohn auf die göttlichen Naturgesetze}"'\footnote{David Zimmer, Der goldene Schatz der Kinderwelt. Ein Nachschlagbüchlein zur naturgemäßen, schnellen und einfachen Behandlung der am meisten vorkommenden Kinderkrankheiten, Wamsdorf, 1923, S. 9. } darstellt. Während jedoch die Art und Weise der religiösen Argumentation auf der Befürworterseite einem "`typischen"' Zeitgeist zugeordnet werden kann, erscheinen diese vier Kommentare eher als Ausnahme geprägt durch die Autoren, hier vor allem Nittinger und Germann. Ersterer zeichnete sich besonders durch religiöse Anrufungen in seinen Werken aus. Dies zeigen unter anderem bereits zitierte Buchtitel wie "`Gott und Abgott der Impfhexe"' oder "`Das falsche Dogma von der Impfung und seine Rückwirkung auf Wissenschaft und Staat"'. Aber auch Überschriften wie "`Bekenntnisse der impfprotestantischen Gemeinden"' und Begriffe wie "`Impfhexer"' oder "`fürchtigmachende Geisterseherei"' tauchen immer wieder auf und zeigen deutlich, in welcher gedanklichen Welt sich Nittinger bewegte.\\
Stellt man diese beiden Kategorien nun gegenüber, finden sich zwar auf beiden Seiten religiöse Argumente, diese weisen jedoch nicht viele Gemeinsamkeiten auf. Während sich auf der Proseite eine ganz klare gesellschaftliche Norm respektive die zeitgenössische Instrumentalisierung der Kirche für staatliche Zwecke der Gesundheitspflege widerspiegeln, erscheinen die religiösen Argumente der Contraseite eher als persönliche Einstellung. Nittinger und seine Anhänger vertraten damit ein theologisch geprägtes Krankheitsbild. Gemäß der Bibel ist Gott selbst Ursache für alles Erfahrbare, Gesundheit gilt in dieser Gedankenwelt als Belohnung, während Krankheit jene ereilt, welche sich außerhalb der Gebote oder der göttlichen Ordnung bewegen. Beides ist Gott gegeben.\footnote{Krankheit in der Bibel:\\ \url{https://www.bibelwissenschaft.de/de/wibilex/das-bibellexikon/lexikon/sachwort/anzeigen/details/krankheit-und-heilung-at/ch/b966be1d644e7c935682914461822921/} (26.10.2015) erneut abgefragt 11.9.2017.}
 Unbestreitbar hatte Nittinger großen Einfluss unter den Impfgegnern. In Zeiten einer allgemeinen Säkularisierung der Gesellschaft sollte diese jedoch nicht überbewertet werden. Um dies aber tatsächlich im Kontext einschätzen zu können, würde sich weiterführend eine Untersuchung über den Einfluss Nittingers auf seine Zeitgenossen anbieten.\\
\\
Zuletzt sollen nun auch die Kategorien \textbf{ohne Wirkung} mit 16 und \textbf{einziges Mittel} mit 14 Argumenten betrachtet werden. Sowohl bei den Impfgegnern als auch bei den Befürwortern finden sie sich in allen drei behandelten Jahrhunderten. Während die Autoren der Contraseite hier alle zu ihrer jeweiligen Zeit bekannten Impfungen ins Feld führen, beziehen sich die Aussagen der Proseite lediglich auf die Pocken-, Masern- und Tetanusimpfung. Wie die Bezeichnung ankündigt, sprechen Erstere der Impfung per se jegliche positive Wirkung ab, da etwa auch vaccinierte Personen geblattert haben, ohne Unterschied zu ungeimpften. Geimpfte sind zu dem weniger tüchtig und viel anfälliger für weitere Krankheiten. Auch die Revaccination kann hier die Situation nicht verbessern, sondern beweist erst recht die Wirkungslosigkeit der Impfung. Ebenso tut dies die Herabsetzung der Dauer des Kuhpockenimpfschutzes von vormals lebenslang auf einige Jahre. Weiters wird ein Zusammenhang zwischen dem Rückgang der Seuchen und der Impfung bis in unsere Zeit hinein vehement bestritten. Der Heilpraktiker Rolf Schwarz etwa hält diesen Zusammenhang schlichtweg für unmöglich: \textit{"`Das Überstehen einer Tetanus Erkrankung hinterlässt keine Immunität, also keinen Schutz vor einer erneuten Infektion (wie bei Pocken Tuberkulose, Diphtherie und Hib). Wie soll eine Impfung dann Schutz bieten, wenn es nicht einmal eine natürliche Immunität gibt? [...] Die Tetanusimpfung gibt es seit 1927. Tetanus wurde aber schon vorher kontinuierlich seltener, die Impfung hat diesen Rückgang nicht beschleunigt. Ein echter Wirksamkeitsnachweis ist übrigens nie erbracht worden, womit auch die offiziell empfohlene Impfabstände rein theoretischer Natur sind."'}\footnote{Schwarz, Impfen -- eine verborgene Gefahr, S. 72--73.} Dagegen wird die Verbesserung der Lebensumstände (Beseitigung des Hungers, bessere Hygiene ...) als einzig gültiges Argument für die Verminderung der Krankheitszahlen in den westlichen Ländern ins Feld geführt.\\
Die Befürworter der Kategorie \textbf{Einziges Mittel} hingegen erhoffen sich von der Durch-\\führung der Impfung die Ausrottung der jeweiligen Krankheit, zumindest aber den Schutz davor, besonders in Anbetracht von fehlenden Heilmitteln zur erfolgreichen Behandlung etwa der Blattern. Zum einen wird argumentiert, dass eine Infektionskrankheit die trotz erfolgreicher Impfung auftritt, deutlich milder verläuft und nicht tödlich endet. Zum Anderen wird die Isolierung von Kranken und die Einhaltung entsprechender Hygienemaßnahmen zwar als unerlässlich für die Bekämpfung von Infektionskrankheiten erachtet, der Impfschutz der pflegenden Personen trotzdem als unerlässlich angesehen, da diese die Krankheit sonst nach außen tragen könnten: "`\textit{Die Impfgegner wollen an die Stelle der Vaccination Isolierung der Pockenkranken und Desinfektion setzen. Beides ist natürlich notwendig, aber nicht allein wirksam. Eine Isolierung ohne Vaccination ist nicht durchführbar, da das den Kranken überwachende Personal dann selbst empfänglich wäre, den Infections-Keim aufnehmen und die Krankheit weiter verbreiten würde. Diese Methode hat auch schon früher gründlich Fiasko gemacht.}"'\footnote{M. Schulz, Impfung, Impfgeschäft und Impftechnik. Ein kurzer Leitfaden für Studierende und Arzte, Berlin, 1888, S. 32.}\\
Ähnliche Aussagen finden sich betreffend Tetanus oder Masern. Auch hier gibt es keine wirksame Behandlung der unmittelbaren Krankheit, es können lediglich die Symptome behandelt werden. Dazu kommt, dass eine überstandene Tetanuserkrankung keine dauerhafte Immunität mit sich bringt und selbst die kleinste Wunde sich infizieren kann, auch wenn sie noch so gründlich gereinigt wird.\\
Wir sehen erneut auf beiden Seiten extrem gegensätzliche Positionen im Sinne einer "`Schwarz-Weiß-Malerei"'. Die Impfung ist entweder gut oder schlecht, einen Mittelweg gibt es hier nicht, was wiederum als Gemeinsamkeit der Kategorien \textbf{ohne Wirkung} und \textbf{einziges Mittel} betrachtet werden kann. Die Behauptung, dass keine einzige Impfung eine Wirkung hat, respektive ein wissenschaftlicher Beweis für diese Wirkung fehlt\footnote{Vgl.: Schwarz, Impfen -- eine verborgene Gefahr, S. 20.}, erinnert stark an die Argumente der Kategorie \textbf{Fragwürdigkeit}. Sie können auch hier als starke Systemkritik bis hin zur Verschwörungstheorie eingestuft werden, da die von der Wissenschaft angeführten Beweise für einen Nutzen der Impfung schlichtweg als nichtig erklärt werden, wodurch zwangsläufig die Glaubwürdigkeit der Wissenschaft im gesamten und als System, angezweifelt wird.



\section{Fazit unter Betrachtung der Theorie und Beantwortung der Forschungsfragen}
"`\textit{Schon in den ersten sechs Wochen des Jahres sind in Österreich mehr
Masernfälle registriert worden als im gesamten Vorjahr. In einem vom
Gesundheitsministerium veröffentlichten Bericht ist von erheblichen
Impflücken die Rede -- auch was andere Krankheiten angeht. Trotzdem steigt die
Zahl der Impfskeptiker.}"'\footnote{Impfpflicht in Österreich (Onlineartikel
vom 8.2.2017): \url{http://orf.at/stories/2378556/2378574/} 20.5.2017.}
Die New Scientist Online berichtet am 10.5.2017 über einen Masernausbruch in
Minnesota: \textit{"`Minnesota measles outbreak follows anti-vaccination
campaign. The state of Minnesota is in the throes of its biggest measles
outbreak in 27 years. As of 5 May, 44 cases had been confirmed. Of these, 42
people were unvaccinated, and 38 belonged to the state's Somali-American community."'}
\footnote{Masernausbruch in den USA (Onlineartikel vom 10.5.2017): \url{https://www.newscientist.com/article/mg23431253-200-minnesota-measles-outbreak-follows-antivaccination-campaign/?utm\_term=Autofeed&utm_campaign=Echobox\&utm_medium=Social\&cmpid=SOC} 20.5.2017.}
Und erst kürzlich informiert
die Presse über die Einführung der Impfpflicht in Italien:\textit{ "`Italien
führt erstmals eine Impfpflicht für Kinder ein und will diese auch mit einer
besonderen Maßnahme abseits von Geldstrafen erzwingen: Kinder im Alter bis zu
sechs Jahren werden einem Beschluss des Ministerrats vom Freitag zufolge
nämlich ohne Grundimmunisierung nicht mehr zum Schulbesuch zugelassen."'}\footnote{Impfpflicht in Italien (Onlineartikel vom 19.5.2017): \url{http://
diepresse.com/home/ausland/aussenpolitik/5220924/Italien-fuehrt-Impfpflicht-
fuer-Kinder-ein} 20.5.2017. }\\
Dagegen publizierte der bekannte Impfgegner Hans Tolzin 2017 sein jüngstes Werk mit dem Titel "`Die Masern-Lüge! Was Sie unbedingt über Masern wissen sollten -- und was die Gesundheitsbehörden Ihnen verschweigen"'.\footnote{Hans U. P. Tolzin, Die Masern Lüge! Was Sie unbedingt über Masern wissen sollten -- und was die Gesundheitsbehörden Ihnen verschweigen, Rottenburg, 2017.}\\

Betrachtet man diese aktuellen Berichte erkennt man sogleich, dass die Frage nach "`Impfen: ja oder nein?"' auch am Ende dieser Arbeit und nach über 200 Jahren Impfgeschichte
nicht ein Stück an Aktualität verloren hat. Eher das Gegenteil scheint der
Fall zu sein. Vermehrte Ausbrüche von vermeintlich zurückgedrängten
Infektionskrankheiten wie Keuchhusten oder Masern heizen die Diskussion neuerlich an. Im Zusammenhang damit steht die ebenfalls diskutierte Frage nach den Ursachen für die neuerlich steigenden Krankheitszahlen. Ist es fehlendes Vertrauen der Bevölkerung in die Medizin? Die ungehinderte Verbreitung von Verschwörungstheorien durch extreme Lager der Impfgegner? Oder ist es die verstärkte Zuwanderung aus medizinisch schlechter
versorgten Ländern? Die Beantwortung dieser Frage wäre zwar höchst interessant, ist Rahmen dieser Arbeit jedoch nicht zu beantworten. Die Überlegung dient der Abrundung des Gesamtbildes. Denn all dies vermittelt den Eindruck, als dass sich die Situation das Impfen betreffend in den letzten 200 Jahren kaum verändert hat, was jedoch der Eingangs aufgestellten Theorie völlig widersprechen würde. Was trifft nun eher zu? Kann die Theorie gehalten werden, oder bleibt doch
alles gleich? Zur Beantwortung dieser Fragen bedarf es einer näheren Betrachtung der einzelnen Bestandteile der herangezogenen Theorie:\\
\\
\textit{
Die Entdeckung, Einführung und Verbreitung der Pockenimpfung zog eine Welle
von medizinwissenschaftlichen sowie medizin-hygienischen Erfindungen und
Medikalisierungsmaßnahmen nach sich. All diese Entwicklungen bedingten einen massiven Wandel in der Gesellschaft, welcher unter anderem das Gesundheits- und
Krankheitsverhalten der Menschen veränderte. Dieser
gesellschaftliche Wandel muss auch anhand der Argumente der Impfdebatte
nachvollziehbar sein.}\\
\\
Am Beginn steht also die Entdeckung, Einführung und Verbreitung der Pockenimpfung.
Wie ausführlich im Kapitel "`Historischer Kontext"' geschildert, fiel die
Entdeckung der Kuhpockenimpfung in die Zeit der Aufklärung, welche von großen
gesellschaftlichen, sozialen und wissenschaftlichen Umbrüchen geprägt war. Zunächst war das Wissen um den Schutz der Kuhpocken vor den echten Menschenpocken Teil bäuerlichen Alltagswissens einiger Regionen Europas. Edward Jenner war allerdings der Erste, der dieses Wissen durch gezielte wissenschaftliche Experimente überprüfte und seine Ergebnisse publizierte. Die rasche Verbreitung dieses Werkes in Europa, den USA bis nach Russland zeigt auf, wie bedrohlich die Pocken für Jenners Zeitgenossen gewesen sein mussten. Ebenfalls bemerkenswert ist es, wie rasch Regierungen und staatliche Behörden diese neue Idee der Impfprophylaxe aufgriffen und prüften, wie am Beispiel Niederösterreich dargelegt wurde.
Die niederösterreichischen Landesbehörden stuften die Impfung bereits 1802 als sicheres Mittel ein und waren bald um eine gezielte Verbreitung der neuen Methode bemüht. Im Laufe des 19. Jahrhunderts folgten in ganz Europa gesetzliche Maßnahmen, die die Menschen dazu bringen sollten,
sich mehr oder weniger freiwillig der Vaccination zu unterziehen. Dieses
Vorgehen entsprach zwar völlig der zeitlichen Strömung des Aufgeklärten
Absolutismus, stellte die einfache Bevölkerung jedoch vor eine gänzlich neue
und unbekannte Situation: Der Staat bediente sich der Medizin. Er stellte die medizinische
Wissenschaft in seinen Dienst und beauftragte ausschließlich universitär ausgebildeten Ärzte mit der
Durchführung der Schutzpockenimpfung, was zu einer umfassenden Medikalisierung der Bevölkerung
führte. Wie eingangs geschildert handelt es sich hierbei um einen der ersten Schritte hin zu einer
staatlich gelenkten Gesundheitsförderung.\\
\\
Der zweite Teil der Theorie besagt, dass auf die Vaccination eine ganze Reihe medizinwissenschaftlicher und medizin-hygienischer Erfindungen sowie Medikalisierungsmaßnahmen folgten. Wie im Kapitel 6.3. "`Von der Zellularpathologie zu den Antidoxinen"' erläutert, nahmen diese in der Mitte des 19. Jahrhunderts ihren Ausgang. Als \textit{medizinwissenschaftliche} Errungenschaften können die Entwicklung der Bakteriologie durch Virchow sowie die neuen wissenschaftlichen Methoden von Pasteur und Koch genannt werden, welche es erst möglich machten, die Krankheitserreger der einzelnen Infektionskrankheiten zu identifizieren und in weiterer Folge zu bekämpfen. Andererseits ist das ausgehende 19. Jahrhundert die Entstehungszeit der Hygiene und damit einhergehenden \textit{medizin-hygienischen} Erfindungen, sowohl im Rahmen der Krankenbehandlung (Desinfizieren von Räumen und Instrumenten, Waschen der Hände, Körperhygiene
der Patienten, verwenden von Handschuhen und vieles mehr) wie auch betreffend der öffentlichen Hygiene (schrittweise Einführung von Abwassersystemen in den Städten, öffentliche Bäder,
Gemeinschaftstoiletten in Wohnhäusern). All dies kann gleichzeitig als \textit{Medikalisierung} betrachtet werden, da es sich um Errungenschaften handelte, welche staatlich gefördert wurden und mit ihren Neuerungen, Vorschriften und Reglementierungen direkten Einfluss auf die Lebenswelt der Menschen hatten. Die zeitnahen politischen Umwälzungen der Märzrevolution 1849, daraus resultierende gesetzliche Änderungen sowie die Verfassung von 1867 ermöglichten zudem die Entstehung eines
Gesundheitssystems respektive einer Gesundheitsversorgung der einfachen Bevölkerung zunächst auf
Vereinsbasis. 1868 entstanden in Wien Allgemeine Arbeiter-, Kranken-
und Invalidenunterstützungskassen, die bereits 1873 innerhalb eines
Verbandes auf breiter Basis reorganisiert wurden. Dieser Verband gilt als die erste
Vereinigung von Krankenkassen in der Österreichisch-Ungarischen Monarchie. 1869
wurde der heute noch bestehende Oberste Sanitätsrat gegründet. Ihm wurde die
Organisation des öffentlichen Sanitätsdienstes und die Vorbereitung
gesetzlicher Grundlagen als Aufgabe übertragen. Das Reichssanitätsgesetz von
1870 reformierte die seit Maria Theresia kollegial organisierte
Gesundheitsbehörde zu staatlichen Sanitätsbehörden, welche dem
Innenministerium eingegliedert und den jeweiligen Ländern unterstellt wurden.
Dieses Gesetz regelte damals die wichtigsten Aufgaben der Gesundheitsbehörden
im Bereich der sanitären Aufsicht und der Seuchenbekämpfung.\footnote{Maria
M. Hofmacher u. Herta M. Rack, Gesundheitssysteme im Wandel: Österreich.
Kopenhagen, WHO Regionalbüro für Europa im Auftrag des Europäischen
Observatoriums für Gesundheitssysteme und Gesundheitspolitik, 2006, S. 15--18.}
Dazu kam der erwähnte indirekte Zwang (Zulassung zu Schulen oder öffentlichen Stellen) der ausgeübt wurde, um die Impfmotivation zu heben. Die Entstehung erster öffentlicher Krankenhäuser (1784 Eröffnung des AKH Wien unter Josef II.) führte nicht nur zur Trennung von Armen- und
Krankenhäusern, sondern vereinfachte den Zugang zur medizinischen Versorgung --
zumindest in den Städten. Mitte des 19. Jahrhunderts erfolgte außerdem die Vereinheitlichung der ärztlichen Ausbildung auf einer universitären Basis, was die Abschaffung des in Zünften organisierten Wundarztberufes zur Folge hatte. Diese gesetzlichen Änderungen die zur Entstehung des österreichischen Gesundheitssystems führten, können neben der Einführung der Pockenimpfung ebenfalls als \textit{Medikaliserungsmaßnahmen} betrachtet werden, da sie unter anderem die staatliche Nutzung der Medizin zum Zwecke der \textit{Public Health} institutionalisierten. Es ist leicht vorzustellen, dass all diese Entwicklungen, gesetzlichen Änderungen und medizinischen Errungenschaften das \textit{Gesundheits- und Krankheitsverhalten} der Bevölkerung veränderten. Begonnen damit, dass Gesundheit und
Krankheit aus dem privaten, persönlichen Raum heraustrat und etwas
öffentliches wurde. Die Hygienemaßnahmen verbesserten die Lebensbedingungen entscheidend und die frühen Formen der Krankenversicherungen boten der einfachen Bevölkerung erstmals ein gewisses Maß an Sicherheit im Falle von Krankheit oder eines Arbeitsunfalles. Im Laufe des 20. und 21. Jahrhunderts hat sich das Gesundheits- und Krankheitsverhalten der Menschen in Österreich dahingehend verändert, dass die kostenlose Versorgung medizinischer Bedürfnisse als selbstverständlich empfunden wird. Die österreichischen Bundesländer sind gesetzlich sogar dazu verpflichtet, eine allgemeine Krankenversorgung durch den Betrieb öffentlicher Spitäler aufrecht zu erhalten.\\
Der in der Theorie genannte und hier in aller Kürze zusammengefasste gesellschaftliche Wandel, der sich unter anderem im Gesundheits- und Krankheitsverhalten zeigt, spannt sich demnach in einem Bogen vom frühen Misstrauen gegenüber der medizinischen Wissenschaft bis hin zur gegenwärtig selbstverständlichen Inanspruchnahme eines öffentlichen
Gesundheitssystems und zwar in der Erwartung einer individuell abgestimmten Behandlung
durch einen gut ausgebildeten Facharzt.\\
\\
Dies führt zum eigentlichen Kern. Betrachtet man nun die Impfung als einschneidende Medikalisierungsmaßnahme, muss sich der attestierte gesellschaftliche Wandel auch in der darum geführten Debatte ausfindig machen lassen. \\
Durch die vergleichende Untersuchung der herangezogenen Werke konnte zunächst eine enorme Bandbreite der Argumentationen aufgezeigt werden. Begonnen bei jenen, welche Gehorsam gegenüber der Obrigkeit fordern, mit statistischen Auswertungen oder wissenschaftlichen
Beweisen von der Wirkung der Impfung zu überzeugen versuchen bis hin zu jenen, die bemüht sind, genau diese Ausführungen in jeglicher Art und Weise zu widerlegen. Innerhalb dieser Spanne von Pro zu Contra
lässt sich zudem ein großes Spektrum ausmachen, welches sich von bedingungsloser Akzeptanz über begründetes Hinterfragen bestehender Systeme bis hin zu surrealen Verschwörungstheorie bewegt.\\
Zudem konnten aus den Vergleichen der Kategorien insgesamt drei Erkenntnisse gewonnen werden:\\
\\
Die erste Erkenntnis, die gewonnen wurde, bezog sich auf das Kind selbst.
In einem kurzen Abriss wurde versucht, die Entwicklung von der Entdeckung der
Kindheit im 18. Jahrhundert bis zur Gegenwart nachzuvollziehen. Es wurde
veranschaulicht, wie das Kind vom unvollständigen, beeinträchtigten Erwachsenen zu
einem eigenständigen, schützenswerten Lebewesen wurde. Bis hin zur Gegenwart, in welcher sich
Eltern bewusst für Kinder entscheiden und diese in den Mittelpunkt ihres
Lebens rücken. Innerhalb der Argumente wurde dies zuerst in den Aussagen der Kategorie
\textbf{Gewissen} deutlich, wo erstmals besonders der
Schutz der Gesundheit der Kinder, sei es mit oder ohne Impfung, als
elterliche und göttliche Pflicht betont wurde. In den Pro- und Contrakategorien \textbf{Gesundheitsförderung} und \textbf{Gesundheitsschädigung} finden sich weiter Bezüge zum Kind bis in die Gegenwart. Die Impfbefürworter verweisen etwa darauf, dass geimpfte Kinder gesünder, blühender, klüger, schöner und vieles mehr sind. Impfgegner dieser Kategorie sehen diese Attribute bei ungeimpften. Das Wohl der Kinder dient in all diesen Fällen dem Unterstreichen des jeweiligen Argumentes. Es ist auch nicht übertrieben zu behaupten, dass die Autoren beider Seiten mit dem Heranziehen des Kindeswohls eine bestimmte Emotionalität beim Publikum auslösen wollen.\\
\\
Als nächstes konnte festgestellt werden, dass sich die Risikowahrnehmung der
Menschen verändert hat. War es bei der Einführung der Kuhpockenvaccination im 19. Jahrhundert noch selbstverständlich, dass die Impfung ein leichtes Unwohlsein aus-\\löste, was im Vergleich zu den echten Pocken kaum der Rede wert war. Ist eben dieses potentielle Unwohlsein im 21. Jahrhundert bereits für viele Eltern ein triftiger Grund seine Kinder nicht mehr zu impfen, da das Risiko besteht, dass die Kinder nach der Injektion einige Tage fiebern oder schlechter schlafen. Viele Eltern sind gegenwärtig nicht mehr bereit, auch nur kurzfristige Beeinträchtigungen der
kindlichen Gesundheit nach einer Impfung in Kauf zu nehmen, zugunsten einer
Krankheit, deren Ausmaß sie nicht kennen. Ganz im Gegenteil gibt es sogar
Diskussionen, ob ungeimpfte Kinder allgemein gesünder sein könnten. Die zu impfende Krankheit tritt in den Hintergrund und die Impfentscheidung selbst wird nicht mehr unbedingt danach getroffen, ob die Impfung gegen die spezielle Krankheit wirkt, sondern welche positive respektive negative Effekte sie darüber hinaus hat. Aufgezeigt werden konnte diese Veränderung in der \textbf{Kosten/Nutzen}
Kategorie und deren Aussagen auf der Proseite des 19. und auf der Contraseite
im 21. Jahrhundert. Ebenfalls ersichtlich ist diese Tendenz in den Kategorien \textbf{Gesundheitsschädigung} oder \textbf{Beeinträchtigung}. Dort finden sich in den Argumenten der Impfgegner zahlreiche Verweise auf ein Risiko zusätzlicher Erkrankungen nach erfolgten Impfungen. In den Kategorien \textbf{Gesundheitsförderung} und \textbf{Sicherheit} der Befürworter finden sich die entsprechenden Gegenstimmen. \\
\\
Als drittes konnte herausgestrichen werden, dass die Impfdebatte auf beiden
Seiten durch sehr viele extreme Positionen bestimmt wurde und wird. Zum
Unterstreichen der jeweiligen Meinung wird tendenziell einseitig
argumentiert, die Impfung ist entweder gut und gänzlich harmlos -- mögliche
Nebenwirkungen werden nicht erwähnt -- oder absolut schlecht -- ohne jegliche
positive Wirkung. Diese Extreme bewegen sich auch gerne außerhalb der
Wissenschaft. Damit einhergehend konnte vor allem aus den Kategorien \textbf{Zwecklosigkeit}
 und \textbf{Beeinträchtigung} der Impfgegner herausgearbeitet werden, dass stark negative Meinungen zum
Impfen oft mit einem bestimmten Krankheitsverständnis einhergehen. Bei diesen
Autoren wird Krankheit nicht als etwas potentiell gefährliches eingestuft,
sondern als wichtiger Teil der natürlichen Entwicklung betrachtet, durch
welche die Kinder reifen und gestärkt hervorgehen. Das viele
Infektionskrankheiten schwere Schäden nach sich ziehen oder gar zum
Tode führen können, wird nicht thematisiert. Damit verbunden ist oft eine
allgemeine Ablehnung der Schulmedizin und der ihr verbundenen Theorien. \\
\\
Der im Rahmen dieser Arbeit ausfindig gemachte und geschilderte gesellschaftliche Wandel in den genannten Teilbereichen steht beispielhaft für die Untermauerung der herangezogenen Theorie und stellt keinen Anspruch auf Vollständigkeit. Je nachdem aus welcher Perspektive man die Impfdebatte betrachtet, können weitere, unterschiedliche Tendenzen und Veränderungen fokussiert und untersucht werden. \\
\\
Zu guter Letzt bleibt noch die Beantwortung der eingangs gestellten Forschungsfragen:\\
Die Frage danach, \textbf{wer} die Impfdebatte führte und führt, wurde im Laufe dieser Arbeit mehrfach  thematisiert. Der deutsche Impfarzt Schulz stellte 1888 in seinem Werk über die Impfung diesbezügliche Überlegung an:\\
"`\textit{Es handelt sich um sehr seltene Ausnahmen, wenn ein wissenschaftlich gebildeter Arzt als Gegner der Impfung auftritt. Deshalb findet die Agitation ausschließlich ihren Boden in nicht sachverständigen Kreisen. Sie wird durch populäre Flugschriften gefördert, in öffentlichen Versammlungen getrieben, in welchen eine ruhige, sachliche Erwägung der hier in Betracht kommenden Frage von vorn herein ausgeschloßen ist.}"'\footnote{Schulz, Impfung, Impfgeschäft und Impftechnik, S. 29.}
Am Ende dieser Arbeit stellt sich heraus, dass es so einfach nicht ist. Zusammenfassend kann man sagen, dass sich an der Impfdebatte all jene Personen beteiligten und beteiligen, welche das Thema direkt betrifft: der Staat im Sinne der \textit{Public Health}, Pharmafirmen als umstrittene wirtschaftliche Profiteure, die Wissenschaft im Bemühen, Beweise für oder gegen die Wirksamkeit von Impfungen zu finden und an der Basis die breite Maße der Laien. Hier vor allem Eltern, welche die Entscheidung über eine Impfung ihrer Kinder zu treffen haben. Grundsätzlich lässt sich feststellen, dass sich die Teilnehmer dieser Jahrhunderte alten Diskussion im Grunde nicht geändert haben: Chronologisch betrachtet stand am Beginn eine kleine Gruppe von Laien, die um die Wirkung der Kuhpocken wussten. Es folgte der Arzt und Wissenschaftler Edward Jenner, der diese Wirkung mit Experimenten belegte und die Ergebnisse publizierte.
Die Ärzte Europas testeten die neue Methode erfolgreich, was die staatlichen Institutionen aufmerksam machte.
Letztere gingen bald daran, die Vaccination in der Bevölkerung bekannt zu machen. \\
\\
Auch die Überlegung dazu, \textbf{in welchen Medien} die Diskussion geführt wurde und wird, konnte bereits mehrfach aufgezeigt werden. Es kann festgehalten werden, dass Überlegungen über das Pro und Contra von Schutzimpfungen in allen jeweils gängigen zeitgenössischen Kommunikationsmedien behandelt wurden und werden: egal ob in Karikaturen, Flugblättern, Predigten, Vereinen, Zeitschriften, Büchern und Zeitungen im 19. und 20. Jahrhundert bis zu den gegenwärtig neuen und sozialen Medien des 21. Jahrhunderts sowie Fernsehen und Kino. Sinn und Zweck war und ist zu allen Zeiten, die angesprochene Zielgruppe auf möglichst breitem Wege von der jeweiligen Meinung zu überzeugen.\\
\\
Die Überlegung dazu, \textbf{wie} die Impffrage diskutiert wurde und wird lässt sich dank des angestellten Vergleiches gut beantworten. Wie eingangs zu erwarten, wurde und wird die Debatte in erster Linie von einer emotionalen "`Schwarz-Weiß-Malerei"' dominiert. Nicht selten ist die Erzeugung von Angst das Ziel dieser Bemühungen, und zwar auf beiden Seiten. Es geht entweder um die Angst vor den Folgen der jeweiligen Krankheit oder der Angst vor Impfnebenwirkungen. Eine Methode, dies zu erreichen, ist der Verweis auf tragische Einzelschicksale wie sie in den Werken der Impfgegner Harris Coulter oder Karl Krafeld angeführt werden. In letzterem finden sich unter anderem Kapitel mit Überschriften wie "`Mein Weg zum Thema Impfen"', "`Daniela Holzmann -- Impfungen -- Segen oder Fluch?"' oder "`Die Karriere eines Impfschadens"'. \\
Eine andere Methode zum Untermauern einer bestimmten Überzeugung, zumindest im 19. Jahrhundert, war der Verweis auf Aussagen und Argumente anderer Autoren, um diese zu demontieren. Das hier auch vor wissenschaftlich fragwürdigen Mittel nicht zurückgeschreckt wurde, belegt das Beispiel "`Germann gegen Giel"': Der Leipziger Arzt und Impfgegner Dr. Heinrich Friedrich Germann widmete dem königlich bayerischen Central-Impfarzt Dr. Franz Seraph Giel in seinem Buch "`\textit{Historische-Kritische Studien über den jetzigen Stand der Impffrage}"' aus dem Jahr 1875 ein ganzes Kapitel. Darin kommentiert Germann Giels Buch "`\textit{Die Schutzpocken=Impfung in Bayern vom Anbeginn ihrer Entstehung und gesetzlichen Einführung bis auf die gegenwärtigen Zeit}"' auf wenig schmeichelhafte Art und Weise: \textit{"`Dennoch schwelgt er} (Anm.: Giel) \textit{im Lobe der Vaccine noch ausgelassener als Dr. Krauss und mit etwas mehr Verworrenheit als Krauss, dass der Leser unwillkürlich in sich denkt, es müsse den Deklamationen ein Defekt der Gehirntäthtigkeit zu Grunde gelegen sein."'}\footnote{Germann, Historische-Kritische Studien, S. 54.}\\
Nach etwa eineinhalb Seiten von Zitaten aus Giels Buch schließt er diese ersten Ausführungen mit dem Vermerk, dass \textit{"`jeden, der an eine physische Unmöglichkeit voranstehender Ungereimtheiten, an plumpe Lüge, an groben Trug denkt und um Nachweise aus der Physiologie und Statistik bittet, den traktiert der arrogante Centralimpfarzt Dr. Giel, S. 162, mit dem Vorwurf von 'Vorurtheil, Unwissenheit, Aberglauben, Falschheit, Bosheit, Verstandesbeschränktheit`"'}\footnote{Germann, Historische-Kritische Studien, S. 56.}.\\
Dem Unterkapitel "`Proben, wie Giel sich selbst widerlegt"' folgt auf Seite 59 ein Absatz mit dem verheißungsvollen Titel "`Giel's Bekehrung"'. Wie anzunehmen versucht Germann hier zu Belegen, dass Giel seinen "`Fehler"' einsieht und im Laufe seiner eigenen Studien über die Impfung eines Besseren belehrt wird. Germann führt hierfür folgendes Zitat an:  \textit{"`Es laufen aus Deutschland, England und überallher üble Nachrichten ein, und als in unserer Gegend die bösartige Seuche ausbrach, sich in kurzer Zeit auf alle Ortschaften verbreitete und sogar an solchen Orten, wo fast Alles vorher geimpft war, da drangen sich mir Thatsachen auf, die mich im Glauben an die Vaccine wankend machten, S. 311, 312."'}\footnote{Germann, Historische-Kritische Studien, S. 59.}\\
Der Leser könnte hier tatsächlich den Eindruck gewinnen, dass der bayerische Zentralimpfarzt und Impfbefürworter von seiner Überzeugung abgefallen sei und seine Meinung über die Vaccination im Angesicht einer großen Epidemie revidiert hatte. Liest man diesen Absatz jedoch im Originalwerk von Franz Seraph Giel nach, trifft man auf folgenden Text: \\
\textit{"`Es war mir in den vielen Jahren, seitdem ich vaccinirte, noch nie ein Fall vorgekommen, der mich an der absoluten Schutzkraft der ächten Vaccine zweifeln ließ. Als aber in unserer Gegend eine bösartige Blatternepidemie ausbrach, die sich in kurzem auf alle Ortschaften verbreitete, und sogar auch an solchen Orten, wo faßt Alles vorher geimpft war, die wenigen einzelnen nicht geimpften Kinder ergriff, drangen sich Thatsachen auf, die mir in höchstem Grade auffallend gewesen syen, und mich im Glauben an die Vaccine wankend gemacht haben würden, wenn ich nicht kurz vorher die ausführliche Recension von Willam u. in der Hall'schen allgemeinen Literaturzeitung gelesen hätte."'}\footnote{Giel, Schutzpocken=Impfung in Bayern, S. 312--313.}\\
Von einer "`Bekehrung"' im Sinne eines Gesinnungswandels vom Impffreund zum Gegner kann hier also keinesfalls die Rede sein. Diese kurze und nur durch Zufall entdeckte, absichtliche Irreführung des Lesers des 19. Jahrhunderts kann dennoch als beispielhaft für diese Art der Vertretung extremer Positionen innerhalb der Impfdebatte betrachtet werden. Wenn auch im Zuge der Argumentensammlung durchaus gezeigt werden konnte, dass es einige Autoren gibt, die sich um einen Mittelweg bemühen und versuchen, ihre Aussagen wissenschaftlich objektiv zu belegen, belegt dieses Beispiel dennoch, warum besonders bei derart kontrovers geführten Diskussionen ein kritisches, hinterfragendes Literaturstudium unerlässlich ist.\\
\\
Am Ende steht noch die Frage, ob sich die Debatte im Laufe der Zeit verändert hat und wenn ja, in wie fern? Caroline Humm geht wie erwähnt in ihrer Dissertation aus dem Jahr 1986 davon aus, dass sich die Argumente über die Jahrhunderte hinweg nicht geändert haben.\footnote{Humm, Geschichte der Pockenimpfung S. 30.} Der selben Meinung ist auch das österreichische Forum Impfschutz zumindest die Contraargumente betreffend, wohingegen der Medizinhistoriker Eberhard Wolff immense Unterschiede in der Impfdiskussion von damals und heute verortet.\\
Natürlich kommt man nicht umhin festzustellen, dass sich einige Argumente und Tatsachen rund um die Impfung kaum verändert haben. So stellte bereits Johann de Carro in seinem Werk von 1802 Überlegungen dazu an, warum Eltern eine Impfung ihrer Kinder ablehnen. Zum einen sah er die Bedenken der Eltern vor einem negativen Ausgang. Zum anderen den \textit{"`Mangel der Aufmunterung von Seiten der Regierungen"'}. Und drittens die \textit{"`Menge der Vorurtheile gegen diese Operation"'}.\footnote{Johann de Carro, Beobachtungen und Erfahrungen über die Impfung der Kuhpocken, Wien, 1802, Vorrede, S. ohne Angaben.} Diese drei Punkte tauchten bereits mehrfach innerhalb der Argumentenliste auf. Ebenso wie Gedanken zur Frage nach den Ursachen für den Rückgang der Seuchen. Hier wird auf der Gegnerseite von jeher das Argument der verbesserten Hygiene und Lebensbedingungen als einzig gültige Ursache angeführt, wohingegen die Befürworter die Wirkung der Impfung als Grund anführen. Trotzdem, und wenngleich hier unterschiedliche Meinungen herrschen, lässt sich als Ergebnis dieser Arbeit festhalten, dass sich die Impfdiskussion eindeutig verändert hat, unter anderem bedingt durch den in der Theorie angenommenen und belegten gesellschaftlichen Wandel. Das konnte mitunter daran gezeigt werden, dass einige Argumente nur im 19. Jahrhundert herangezogen wurden, wie etwa jene der Kategorien \textbf{Obrigkeit}, \textbf{Gewissen}, \textbf{Religiöse Motive} oder \textbf{Religions-/Naturgesetz}.
Genauso finden sich Kategorien, welche erst seit späterer Zeit, dem 20. oder 21. Jahrhundert bemüht worden sind, wie etwa \textbf{Statistik}, \textbf{Beeinträchtigung} sowie \textbf{Kosten/Nutzen} und \textbf{Machtinteressen} auf der Gegnerseite. Hinzu kam die Weiterentwicklung der Wissenschaft, welche eine umfassendere Beweisführung der jeweiligen Argumente erlaubte. Der rasante technische Fortschritt innerhalb der Medienlandschaft, welche den Zugang zu Publikationsmöglichkeiten mehr und mehr erleichterte, oder die steigende Alphabetisierung der Bevölkerung, welche die Zielgruppe enorm erweiterte. Ferner veränderte sich die Anzahl der Impfung von einer Einzigen auf momentan elf im Mutter-Kind-Pass empfohlene Schutzimpfungen innerhalb der ersten zwei Lebensjahre. Die geänderte Risikowahrnehmung veränderte zudem noch die Erwartungshaltung an die Impfungen massiv. Dies musste zwangsläufig nicht nur auf die Diskussion selbst, sondern ebenfalls auf deren Bandbreite Einfluss nehmen.









\newpage

\section{Abkürzungsverzeichnis}
AEFI = Adverse Events Following Immunization\\
AEGIS = Verein Aktives Eigenes Gesundes Immun System\\
BCG = Bacillus Calmette-Guerin (Tuberkuloseimpfung)\\
BMG = Bundesministerium für Gesundheit\\
EMA = European Medicines Agency\\
DTP = Diphtherie-Tetanus-Pertussis Kombinationsimpfstoff\\
HPV = Humane Papilloma Viren\\
IPV = Inaktiver Polioimpfstoff\\
MMR = Masern-Mumps-Röteln Impfung\\
OSR = Oberste Sanitätsrat (in Österreich)\\
OPV = Oraler Polioimpstoff\\
SAE = Serious Adverse Event\\
STIKO = Ständige Impfkommission am Robert Koch Institut (in Deutschland)\\
UN = United Nationds, Vereinte Nationen\\
VAPP = Vakzine-assoziierte paralytische Poliomyelitis\\
UN = United Nations/Vereinte Nationen\\
WHO = World Health Organisation


\newpage
\section{Glossar}
Medikalisierung: Darunter versteht man jenen Prozess, bei welchem die menschliche Lebenswelt mehr und mehr in den Fokus der medizinischen Wissenschaft und des Staates gerät.\\
\\
Inokulation: Die absichtliche Übertragung der Kuhpocken auf den Menschen zum Zweck der Immunisierung vor den Menschenpocken.\\
\\
Variolation: Die absichtliche Einimpfung der echten Pocken zur Immunisierung im Falle einer Epidemie.

\newpage
\section{Literaturverzeichnis}

\begin{itemize}
\item{Peter Atteslander, Methoden der empirischen Sozialforschung, Berlin, 2010.}
\item{Peter Baldwin, Contagion and the State in Europe, 1830--1930, Cambridge, 2004.}
\item{Herve Bazine, The Eradiction of Smallpox. Edward Jenner and the First and Only Eradication of a Human Infectious Disease, San Diego, 2000.}
\item{Ingeborg Becker-Textor (Hrg.), Maria Montessori. Zehn Grundsätze des Erziehens, Freiburg, 2012.}
\item{Winfried Böhm und Michel Soetard, Jean-Jacques Rousseau. Der Pädagoge, Paderborn, 2012.}
\item{Max Brucker, Ilse Gutjahr, Biologischer Ratgeber für Mutter und Kind, 5. Auflage, Lahnstein, 1987.}
\item{Gerhard Buchwald, Der Impf-Unsinn. Vorträge des Jahres 2004, Norderstedt, 2004.}
\item{Gerhard Buchwald, "`Gedanken zu Publikationen eines Impfgegners"'. Richtigstellung zur Veröffentlichung des Herren Dr. W. Ehrengut, in: Naturheilpraxis, Heft 5, 1989.}
\item{Gerhard Buchwald, Impfen. Das Geschäft mit der Angst, 3. Auflage, Lahnstein, 1995.}
\item{Harris L. Coulter, Barbara L. Fisher, Dreifach-Impfung. Ein Schuß ins Dunkle, Schäftlarn, 1996.}
\item{Matthias Dahl, Impfung in der Pädiatrie und der "`informed consent"' -- Balanceakt zwischen Sozialpaternalismus und Autonomie, in: Ethik in der Medizin, Band 14, Heft 3, Stuttgart, 2002, S. 201--214.}
\item{Martin Dinges (Hg.), Medizinkritische Bewegungen im Deutschen Reich, 1870 bis 1933, Stuttgart, 1996.}
\item{Wolfgang Uwe Eckart, Der Beginn der Ära von Serumtherapie und Impfung, in: Ärzte Zeitung, Heft 48, 2004, S. 19.}
\item{Wolfgang Uwe Eckart, Robert Jütte, Medizingeschichte. Eine Einführung, Köln, 2007.}
\item{Wolfgang Uwe Eckart (Hg.), Jenner. Untersuchungen über die Ursachen und Wirkungen der Kuhpocken, Berlin, 2016.}
\item{Wolfgang U. Eckart, Geschichte der Medizin, 2. Auflage, Berlin, 1994.}
\item{Heinz Flamm, Christian Vutuc, Geschichte der Pocken-Bekämpfung in Österreich, in: Wiener klinische Wochenschrift, Heft 122, Wien, 2010, S. 265--275.}
\item{Heinz Flamm, Pasteurs Wut-Schutzimpfung - vor 130 Jahren in Wien mit Erfolg begonnen und doch offiziell abgelehnt, in: Wiener Medizinische Wochenschrift, Heft 165, Wien, 2015, S. 322--339.}
\item{Michael Fitzpatrick, The Cutter Incident. How America's First Polio Vaccine Led to a Growing Vaccine Crisis, in: Journal of the Royal Society of Medicine, Issue 99 (3), London, 2006, S. 156.}
\item{Jana Gärtner, Elternratgeber im Wandel der Zeit. Deskriptive Ratgeberanalyse am Beispiel der sogenannten Klassischen Kinderkrankheiten unter Berücksichtigung der Impfdebatte, Berlin, 2010.}
\item{Friedrich Graf, Die Impfentscheidung. Ansichten, Überlegungen und Informationen -- vor jeglicher Ausführung!, 5. Auflage, Ascheberg, 2013.}
\item{Heinz-Gerhard Haupt, Jürgen Kocka, Historischer Vergleich: Methoden, Aufgaben, Probleme. Eine Einleitung, in: Heinz-Gerhard Haupt, Jürgen Kocka (Hg.), Geschichte und Vergleich. Ansätze und Ergebnisse international vergleichender Geschichtsschreibung, Frankfurt/Main, 1996. S. 9--46.}
\item{Ulrich Heininger (Hrg.), Pertussis bei Jugendlichen und Erwachsenen, Stuttgart, 2003.}
\item{Marina Hilber, Institutionalisierte Geburt. Eine Mikrogeschichte des Gebärhauses, Bielefeld, 2012.}
\item{Martin Hirte, Impfen, Pro \& Contra. Das Handbuch für eine individuelle Impfentscheidung, München, 2008.}
\item{Maria M. Hofmacher u. Herta M. Rack, Gesundheitssysteme im Wandel: Österreich. Kopenhagen, WHO Regionalbüro für Europa im Auftrag des Europäischen Observatoriums für Gesundheitssysteme und Gesundheitspolitik, 2006.}

\item{Jakob Hort, Vergleichen, Verflechten, Verwirren. Vom Nutzen und Nachteil der Methodendiskussion in der wissenschaftlichen Praxis: ein Erfahrungsbericht, in: Agnes Arndt u. a. (Hg.), Vergleichen verflechten, verwirren? Europäische  Geschichtsschreibung zwischen Theorie und Praxis, Göttingen, 2011, S. 319--341.}
\item{Calixte Hudemann-Simon, Die Eroberung der Gesundheit 1750--1900, Frankfurt, 2000.}
\item{Caroline Marie Humm, Die Geschichte der Pockenimpfung im Spiegel der Impfgegner, München, 1986.}
\item{William M. Johnston, Österreichische Kultur und Geistesgeschichte. Gesellschaft und Ideen im Donaurraum 1848 bis 1938, 4. Auflage, Wien, 2006.}

\item{Marius Kaiser, Pocken und Pockenschutzimpfung. Ein Leitfaden für Amtsärzte, Impfärzte und Studierende der Medizin, Wien, 1949.}
\item{Karl Krafeld, Impfen -- Völkermord im dritten Jahrtausend? Stuttgart, 2003.}
\item{Michael Kunze, Forum Impfschutz: Das österreichische Impfsystem und seine Finanzierung. Lösungsvorschläge für alternative Finanzierungsformen, Wien, 2010.}
\item{Peter Kriwy, Gesundheitsvorsorge bei Kindern. Eine empirische Untersuchung des Impfverhaltens bei Masern, Mumps und Röteln, Wiesbaden, 2007.}
\item{Alexander Langbauer, Das Österreichische Impfwesen unter besonderer Berücksichtigung der Schutzimpfung, Linz, 2010.}
\item{Martina Lenzen-Schulte, Impfungen. 99 verblüffende Tatsachen, Wackernheim, 2008.}
\item{Francisca Loetz, Vom Kranken zum Patienten. "`Medikalisierung"' und medizinische Vergesellschaftung am Beispiel Badens 1750--1850, Stuttgart, 1993.}
\item{Wolfgang Maurer, Impfskeptiker -- Impfgegner. Von einer anderen Realität im Internet, in: Pharmazie in unserer Zeit, Volume 37, Jänner 2008, S. 64--70.}
\item{Philipp Mayring, Qualitative Inhaltsanalyse. Grundlagen und Techniken, 12. Auflage, Basel, 2015.}
\item{Philipp Mayring, Qualitative Inhaltsanalyse, in: Forum: Qualitative Sozialforschung, Volume 1, No. 2, Art. 20, Juni 2000.}
\item{Michael Memmer, Die Geschichte der Schutzimpfung in Österreich. Eine rechts-historische Analyse, in: Gerhard Aigner (Hrg.), Markus Grimm (Hrg.) u. a., Schutzimpfungen -- Rechtliche, ethische und medizinische Aspekte. Schriftenreihe Ethik und Recht in der Medizin, Band 11, Wien, 2016, S. 7--36.}
\item{Markus W. Moser und Beatrix Patzak, Variola: zur Geschichte einer museal präsenten Seuche, in: Wiener klinische Wochenschrift, Heft 120, Wien, 2008, S. 3--10.}
\item{Ingomar Mutz, Diether Spork, Geschichte der Impfempfehlungen in Österreich, in: Wiener Medizinische Wochenschrift, Heft 157/5, Wien, 2007, S. 94--97.}
\item{Michael Pammer, Vom Beichtzettel zum Impfzeugnis. Beamte, Ärzte, Priester und die Einführung der Vaccination, in: Institut für Österreichkunde (Hrg.), Österreich in Geschichte und Literatur, 39. Jahrgang, Heft 1, Jänner 1995, S. 11--29.}
\item{Anita Petek-Dimmer, Rund ums Impfen, Buchs, 2004.}
\item{Dorothy Porter, Roy Porter, The politics of Prevention: Anti-Vaccinationism and Public Health in nineteenth-century England, S. 231--252, in: Medical History, Issue 32, o. A., 1988.}
\item{Roy Porter, Die Kunst des Heilens. Eine medizinische Geschichte der Menschheit von der Antike bis heute, Berlin, 2000.}
\item{Rolf Schwarz, Impfen -- eine verborgene Gefahr? Impftheorie und Infektionstheorie auf dem Prüfstand, München, 2012.}
\item{Marcus Sonntag, Pockenimpfung und Aufklärung. Popularisierung der Inokulation und Vakzination. Impfkampagne im 18. und frühen 19. Jahrhundert, Bremen, 2014.}
\item{Heinz Spiess, Schutzimpfungen, Stuttgart, 1958.}
\item{Tilli Tansey, Pioneer of polio eradication, in: Medical History, Nature, Issue 520, 2015.}
\item{Hans U. P. Tolzin, Die Masern-Lüge! Was Sie unbedingt über Masern wissen sollten -- und was die Gesundheitsbehörden Ihnen verschweigen, Rottenburg, 2017.}
\item{Daniel Tröhler, Johann Heinrich Pestalozzi, Wien, 2008.}
\item{Sabine Falk und Alfred Stefan Weiss, "`Hier sind die Blattern."' Der Kampf von Staat und Kirche für die Durchsetzung der (Kinder-)Schutzpockenimpfung in Stadt und Land Salzburg (Ende des 18. Jahrhunderts bis ca. 1820), in Mitteilungen der Gesellschaft für Salzburger Landeskunde 131 (131), S. 163--186.}
\item{Boris Velimirovic, Impfgegner, in: Irmgard Oepen, Amardeo Sarma (Hg.), Parawissenschaften unter der Lupe, Münster, 1995, S. 43--47.}
\item{Hans-Ulrich Wehler, Modernisierungstheorie und Geschichte, Göttingen, 1975.}
\item{Christina Weiskopf, Abenteuer Impfung. Was Eltern über Kinderkrankheiten und Impfungen wissen sollen, Lappersdorf, 2007.}
\item{Alfred Stefan Weiss, Joseph Servatius von d'Outrepont (1776--1845). Zum 150. Todestag eines bedeutenden Salzburger und Würzburger Arztes, in: Salzburg Archiv 20 (1995), S. 169--184.}
\item{John Rowan Wilson, Polio! Die Geschichte eines Impfstoffes, Wien, 1963.}
\item{Johannes Wimmer, Gesundheit, Krankheit und Tod im Zeitalter der Aufklärung. Fallstudien aus den habsburgischen Erbländern, Wien, 1991.}
\item{Michael Winterhoff, Warum unsere Kinder Tyrannen werden oder: Die Abschaffung der Kindheit, Gütersloh, 2008.}
\item{Eberhard Wolff, Einschneidende Maßnahmen. Pockenschutzimpfung und traditionale Gesellschaft im Württemberg des frühen 19. Jahrhunderts, Stuttgart, 1998.}
\item{Eberhard Wolff, Medizinkritik der Impfgegner im Spannungsfeld zwischen Lebenswelt- und Wissenschaftsorientierung, in: Martin Dinges (Hg.), Medizinkritische Bewegungen im Deutschen Reich (ca. 1870 -- ca. 1933), Stuttgart, 1996, S. 79--108.}
\end{itemize}


\newpage

\section{Onlineartikel}

\begin{itemize}
\item{Argument: \url{http://www.ejka.org/de/content/wie-ist-eine-gute-argumentation-aufgebaut} 11.9.2017.}
\item{Argument: \url{http://www.duden.de/rechtschreibung/Argument} 11.9.2017.}
\item{Theodor Billroth: \url{http://geschichte.univie.ac.at/de/node/33917} 11.9.2017.}
\item{BMG Prävention: \url{http://www.bmg.gv.at/home/Schwerpunkte/Gesundheitsfoerderung_Praevention/} 11.9.2017.}
\item{BMG Impfen: \url{http://bmg.gv.at/home/Schwerpunkte/Gesundheitsfoerderung_Praevention/Impfen/} 11.9.2017.}
\item{Deutsches Sprichwort: \url{http://www.aphorismen.de/suche?f_thema=Gesundheit&seite=2} 11.9.2017.}
\item{Johann Christian Ehrmann: \url{https://www.deutsche-biographie.de/sfz23255.html#adbcontent} 11.9.2017.}

\item{Theodor Fritzsch, John Locke. Gedanken über Erziehung, Leipzig, 1920, Online in: \url{http://gutenberg.spiegel.de/buch/gedanken-uber-erziehung-6212/1} 12.5.2017.}
\item{Gesamte Rechtsvorschrift für das Epidemiegesetz: \url{https://www.ris.bka.gv.at/GeltendeFassung.wxe?Abfrage=Bundesnormen&Gesetzesnummer=10010265} 11.9.2017.}
\item{Gesundheitsprävention: \url{http://www.aamp.at/unsere-themen/praevention/definition-praevention/} 11.9.2017.}
\item{Geschichte der Pockenimpfung von 1713 bis 1977: \url{http://www.impf-alternative.de/2011/01/350/} 14.3.2017.}
\item{Geschichte der Statistik: \url{https://de.statista.com/statistik/lexikon/definition/154/statistik_fuer_anfaenger_geschichte_der_statistik/} 25.4.2017.}
\item{History of public health: \url{http://priory.com/history_of_medicine/public_health.htm} 11.9.2017.}
\item{Goldenes Brett vorm Kopf: \url{http://wien.orf.at/news/stories/2738165/} 11.9.2017.}
\item{HPV-Impfung: \url{http://www.netdoktor.at/gesundheit/impfung/hpv-impfung-5339} 11.9.2017.}
\item{HPV-Impfung: nüchterne Fakten statt hitziger Diskussionen: \url{http://www.medizin-transparent.at/hpv-impfung} 11.9.2017.}
\item{HPV-Impfstoff: Vier Frauen klagen Hersteller: \url{http://kurier.at/wissen/hpv-impfstoff-vier-frauen-klagen-hersteller/37.536.723} 11.9.2017.}
\item{Impfen nein danke: \url{http://www.impfen-nein-danke.de/} 11.9.2017.}
\item{Impfreaktion, Impfschaden und Impfnebenwirkung: \url{http://www.reisemed.at/impfungen/impfreaktionen-und-impfnebenwirkungen} 11.9.2017.}
\item{Impfschaden und Entschädigung: \url{https://www.gesundheit.gv.at/Portal.Node/ghp/public/content/Nebenwirkungen_von_Impfungen_LN.html} 11.9.2017.}
\item{Impfpflicht in Österreich (Onlineartikel vom 8.2.2017): \url{http://orf.at/stories/2378556/2378574/} 20.5.2017.}
\item{Impflicht in Italien (Onlineartikel vom 19.5.2017): \url{http://diepresse.com/home/ausland/aussenpolitik/5220924/Italien-fuehrt-Impfpflicht-fuer-Kinder-ein} 20.5.2017.}
\item{Edward Jenner, Enxyclopaedia Britannica: \url{http://www.britannica.com/\\biography/Edward-Jenner} 11.9.2017.}
\item{Edward Jenner: \url{http://www.bbc.co.uk/history/historic_figures/jenner_edward.shtml} 11.9.2017.}
\item{Hartmut Kaelble, Historischer Vergleich, Version: 1.0, in: Docupedia-Zeitgeschichte, 14.8.2012, \url{http://docupedia.de/zg/Historischer_Vergleich?oldid=125457} 11.9.2017.}
\item{Krankheit/Gesundheit in der Bibel: \url{https://www.bibelwissenschaft.de/de/wibilex/das-bibellexikon/lexikon/sachwort/anzeigen/details/krankheit-und-heilung-at/ch/b966be1d644e7c935682914461822921/} 11.9.2017.}
\item{Kinderimpfprogramm: \url{https://www.bmgf.gv.at/home/Service/Gesundheits\\leistungen/Kostenloses_nbsp_Kinder-Impfprogramm} 11.9.2017.}
\item{Herwig Kollaritsch, Maria Paulke-Korinek, Poliomyelitis, S. 25, in: Österreichische Ärztezeitung, Heft 22, Wien, November 2014, S. 24--33. Onlineausgabe: \url{http://www.aerztezeitung.at/fileadmin/PDF/2014_Verlinkungen/State_Polio.pdf} 11.9.2017.}
\item{Johann Loibner: \url{http://dr.loibner.net/} 11.9.2017.}
\item{Johann Loibner: Artikel Standpunkte: Höchstgericht kippt Berufsverbot für Impfkritiker, Springermedizin, 17.9.2013: \url{http://www.springermedizin.at/artikel/36633-standpunkte-hoechstgericht-kippt-berufsverbot-fuer-impfkritiker} 11.9.2017.}
\item{Keuchhusten: \url{http://www.reisemed.at/krankheiten/keuchhusten-pertussis} 11.9.2017.}
\item{Keuchhusten, Standard Online, 17.1.2014: \url{http://derstandard.at/1389857404641/Oesterreich-Keuchhusten-erlebt-Renaissance} 11.9.2017.}
\item{Kinderlähmung-Impfen oder nicht? in: Spiegel Online, Heft 17, 24.4.1957, S. 28--32: \url{http://www.spiegel.de/spiegel/print/d-41757287.html} 11.9.2017.}
\item{Kinderlähmung, in: Die Welt Digital, 24.10.2012: \url{http://www.welt.de/gesundheit/article110213993/Eine-Welt-ohne-Kinderlaehmung-ist-zum-Greifen-nah.html} 11.9.2017.}
\item{Gustav Kuschinsky, zitiert nach: \url{http://www.reisemed.at/impfungen/impfreak\\tionen-und-impfnebenwirkungen} 11,9,2017.}
\item{Lübecker Totentanz, Wiener Zeitung Online, 3.2.2012: \url{http://www.wienerzeitung.at/themen_channel/wissen/geschichte/432666_Luebecker-Totentanz.html} 11.9.2017.}
\item{Lübecker Impfunglück: \url{http://flexikon.doccheck.com/de/L\%C3\%BCbecker_Impfungl\%C3\%BCck} 11.9.2017.}
\item{Masernausbruch in den USA (Onlineartikel vom 10.5.2017): \url{https://www.newscientist.com/article/mg23431253-200-minnesota-measles-outbreak-follows-antivaccination-campaign/?utm_term=Autofeed&utm_campaign=Echobox&utm_medium=Social&cmpid=SOC} 20.5.2017.}
\item{Philipp Mayring, Qualitative Inhaltsanalyse, in: Forum: Qualitative Sozialforschung, Volume 1, No. 2, Art. 20, Juni 2000: \url{http://www.qualitative-research.net/index.php/fqs/article/view/1089/2383} 25.7.2017.}
\item{Thomas Mergel, Modernisierung, Pkt. 1, in: Europäische Geschichte Online (EGO), hg. vom Institut für Europäische Geschichte (IEG), Mainz 2011-04-27. \url{http://www.ieg-ego.eu/mergelt-2011-de} 11.9.2017.}
\item{Maria Montessori: \url{https://montessori.at/montessori/ms-paedagogik/} 11.9.2017.}
\item{Mumps: \url{http://www.reisemed.at/krankheiten/mumps} 11.9.2017.}
\item{Heinrich Oidtmann: \url{http://www.glasmalerei-oidtmann.de/chronik.html} 11.9.2017.}
\item{Oberste Sanitätsrat: \url{http://www.bmg.gv.at/home/Ministerium/Oberster_Sanitaetsrat/} 11.9.2017.}
\item{Pocken sind tot, Ebola ist auferstanden (Dr. Loibner): \url{http://www.aegis.at/wordpress/pocken-sind-tot-ebola-ist-auferstanden-dr-loibner/} 11.9.2017.}
\item{Poliomyelitis RKI-Ratgeber für Ärzte: \url{https://www.rki.de/DE/Content/Infekt/EpidBull/Merkblaetter/Ratgeber_Poliomyelitis.html} 20.7.2017.}
\item{Neue Impfung gegen Meningitis, Der Standard Online, 23.4.2014: \url{http://derstandard.at/1397521376319/Neue-Imfpung-gegen-Meningitis} 11.9.2017.}
\item{Österreich hat zweithöchste Masernrate in Europa, Die Presse Online, 25.5.2016: \url{http://diepresse.com/home/leben/gesundheit/4996225/Osterreich-hat-zweithochste-MasernRate-in-Europa} 11.9.2017.}
\item{Heinrich Pestalozzi: \url{http://www.heinrich-pestalozzi.de/} 11.9.2017.}
\item{Präventionsparadoxon: \url{http://www.leitbegriffe.bzga.de/bot_angebote_idx-161.html} 14.5.2017.}
\item{Preisausschreiben zum Beweis des Masernvirus: \url{https://web.archive.org/web/20120329214816\\/http://www.klein-klein-verlag.de/Viren-|-Erschienen-in-2011/24112011-das-masern-virus-100000-euro-belohnung.html} 29.1.2016.}
\item{Röteln-Embryopathie: \url{http://flexikon.doccheck.com/de/R\%C3\%B6telnembryopathie} 11.9.2017.}
\item{Serumtherapie: \url{http://www.spektrum.de/lexikon/biologie/serumkrankheit/61229} 11.9.2017.}
\item{Umstrittene Doku, Martin Both vom 2.4.2017: \url{https://www.merkur.de/bayern/umstrittene-doku-vaxxed-laeuft-in-muenchen-grosses-kino-fuer-impf\\gegner-8069609.html} 12.5.2017.}
\item{Vaccine Injury Compensation Programs: \url{http://www.historyofvaccines.org/content/articles/vaccine-injury-compensation-programs} 11.9.2017.}
\item{Vaccine Timeline: \url{http://www.immunize.org/timeline/} 11.9.2017.}
\item{Gerhard van Swieten: \url{http://austria-forum.org/af/Wissenssammlungen/Biographien/Swieten,_Gerard_van} 11.9.2017.}
\item{The Lancet, 1998 Feb 28, Heft 351(9103), S. 637-41, online: \url{http://www.thelancet.com/journals/lancet/article/PIIS0140-6736(97)11096-0/abstract} 25.7.2017.}
\item{UN-Kinderrechtskonvention: \url{https://www.kinderrechtskonvention.info/gesundheitssorge-3601/} 18.6.2017.}
\item{Verfassung der Weltgesundheitsorganisation vom 22.7.1946, Stand 8. Mai 2014, S. 1. in: \url{https://www.admin.ch/opc/de/classified-compilation/19460131/201405080000/0.810.1.pdf} 11.9.2017.}
\item{Weltgesundheitstag: \url{http://www.weltgesundheitstag.de/cms/index.asp?inst=wgt-who&snr=11138&t=2016\%A7\%A7Diabetes} 11.9.2017.}
\item{WHO-Ottawa-Charta: \url{http://www.euro.who.int/__data/assets/pdf_file/0006/129534/Ottawa_Charter_G.pdf} 11.9.2017.}
\item{WHO Model List of Essential Medicines: \url{http://www.who.int/medicines/publications/essentialmedicines/18th_EML.pdf} 11.9.2017.}
\item{Ursula Wiedermann-Schmitz, u. A., Reaktionen und Nebenwirkungen nach Impfungen. Erläuterungen und Definition in Ergänzung zum Österreichischen Impfplan, 2013, in: \url{http://bmg.gv.at/cms/home/attachments/1/5/5/CH1100/CMS1386342769315/impfungen-reaktionen_nebenwirkungen.pdf} 11.9.2017.}
\item{Jürgen Wilke, Zensur und Pressefreiheit, in: Europäischer Geschichte Online (EGO), Mainz, 2013: \url{http://ieg-ego.eu/de/threads/europaeische-medien/zensur-und-pressefreiheit-in-europa} 2.5.2017.}
\item{Zensur in Österreich: \url{https://austria-forum.org/af/Wissenssammlungen/Essays/Medien/Pre\%C3\%9Ffrechheit_und_Zensur} 2.5.2017.}
\item{Zedlers Universallexikon, Band 2: \url{https://www.zedler-lexikon.de/index.html?c=blaettern&seitenzahl=710&bandnummer=02&view=150&l=de} 11.9.2017.}
\end{itemize}


\section{Gedruckte Quellen}

\begin{itemize}
\item{John Baron, The Life of Edward Jenner, London, 1838.}
\item{Johann de Carro, Beobachtungen und Erfahrungen über die Impfung der Kuhpocken, Wien, 1802.}
\item{Johann Christian Ehrmann, Ueber den Kuhpockenschwindel bei Gelegenheit der abgenöthigten Vertheidigung gegen die Brutalimpfmeistere, den Herrn Dr. und Hofrath Sömmering und den Herren Dr. Lehr, Frankfurt a. M. 1801.}
\item{Reinhold Gerling, Blattern und Schutzpocken-Impfung. Öffentliche Anklage: Impfgegner c/a Gesundheitsamt. Kritische  Beleuchtung  und  Widerlegung  der  Irrthümer der  im  Kaiserlichen  Gesundheitsamt  bearbeiteten  Denkschrift  zur Beurtheilung  des  Nutzens  des  Impfgesetzes, Berlin, 1896.}
\item{Heinrich Friedrich Germann, Historisch-Kritische Studien über den jetzigen Stand der Impffrage, 2. Band, Leipzig, 1875.}
\item{Franz Seraph Giel, Die Schutzpocken=Impfung in Bayern, vom Anbeginn ihrer Entstehung und gesetzlichen Einführung  bis auf gegenwärtige Zeit. Dann mit besonderer Beobachtung derselben in auswärtigen Staaten, München, 1830.}
\item{A. K. Hesselbach (Hrg.), Bibliothek der deutschen Medicin und Chirugie, Würzburg, 1820.}
\item{Georg Friedrich Krauss, Die Schutzpockenimpfung in ihrer endlichen Entscheidung, als Angelegenheit des Staats, der Familien und des Einzelnen, Nürnberg, 1820.}
\item{Gregor Krämer, Predigt zur Verhütung der Blatternpest, gehalten am Feste des heiligen Joseph, Salzburg, 1802.}
\item{Johann Kumpfhofer, Predigt von der Pflicht der Eltern ihren Kindern die Kuhpocken einimpfen zu lassen, Linz, 1808.}
\item{Johann Valentin Müller, Beweis dass die Kuhpocken mit den natürlichen Kinderblattern in keiner Verbindung stehen, und also ihre Einimpfung kein untrügliches Verwahrungsmittel gegen die natürlichen Blattern sein könne, dem Publicum zur Beherzigung gewidmet, Frankfurt a. M., 1801.}
\item{Carl Georg Gottlob Nittinger, Gott und Abgott oder die Impfhexe, Stuttgart, 1863.}
\item{Carl Georg Gottlob Nittinger, Über die 50jährige Impfvergiftung des württembergischen Volkes, Stuttgart, 1852.}
\item{Carl Georg Gottlob Nittinger, Die Impfung ein Mißbrauch: Spiegel für die Schrift: Würdigung der großen Vortheile der Kuhpocken-Impfung für das Menschengeschlecht von Dr. Michael Reiter, Stuttgart, 1853.}
\item{Carl Georg Gottlob Nittinger, Die Impfregie mit Blut und Eisen, Stuttgart, 1868.}
\item{Carl Georg Gottlob Nittinger, Das falsche Dogma von der Impfung und seine Rückwirkung auf Wissenschaft und Staat, München, 1857.}
\item{Max von Niessen, Gibt es Naturpockenschutz durch die Kulturpockenverseuchung der Vakzination? Dresden, 1935.}
\item{Joseph d'Outrepont, Belehrung des Landvolkes über die Schutzblattern. Nebst einem kurzen Unterrichte über die Impfung derselben für die Wundärzte, Salzburg, 1803.}
\item{Gustav Paul, Der Nutzen der Schutzpocken-Impfung. Vortrag gehalten am 30.März 1901 in der 87. Vollversammlung des Vereins für Kindergärten und Kinderbewahranstalten in Österreich, Wien, 1901.}
\item{Gustav Paul, Die Entwicklung der Schutzpockenimpfung in Österreich, Wien, 1901.}
\item{Matthäus Priegl, Predigten zur Empfehlung der Blattern=Einimpfung, Krems, 1817.}
\item{Wilhelm Ressel, Das Impfgeschäft als starrstes Dogma der modernen orthodoxen Medizin. Richtigstellung falscher und gefährlicher zunftwissenschaftlicher Ueberlieferungen. Zugleich und hauptsächlich ein Weckruf an Deutschlands Zeitungs=Redakteure, Dreseden, 1910.}
\item{Gustav Adolf Schlechtendahl, Wahn oder Wirklichkeit? Vorurteil oder Wahrheit? Gedanken und Aktenstücke zur Frage der Schutzpocken-Impfung, Berlin, 1908.}
\item{Carl Schreiber, Band I: Gründe gegen die allgemeine Kuhpockenimpfung, 2. Auflage, o. A., 1834, Reprint Göttingen, 1998.}
\item{M. Schulz, Impfung, Impfgeschäft und Impftechnik. Ein kurzer Leitfaden für Studierende und Arzte, Berlin, 1888.}
\item{Robert Walker, Untersuchung der Pocken in medicinischer und politischer Rücksicht, nebst einer glücklichen Methode, diese Krankheit zu heilen, einer Erklärung der Ursache der Pockengruben, einem Mittel dieselben abzuwenden und einem Anhange über den gegenwärtigen Zustand der Pocken, Leipzig, 1791.}
\item{David Zimmer, Der goldene Schatz der Kinderwelt. Ein Nachschlagbüchlein zur naturgemäßen, schnellen und einfachen Behandlung der am meisten vorkommenden Kinderkrankheiten, Wamsdorf, 1923.}
\end{itemize}

\section{Anhang}

\subsection{Quellenverzeichnis der Argumente}
\begin{itemize}
\item{C. R. Aikin, Kurze Uebersicht der wichtigsten Erfahrungen über die Kuhpocken, Pesth, 1802.}
\item{Friedrich Becker, Impfen oder Nichtimpfen. Beitrag zur Lösung der grossen Tagesfrage über den Impfzwang und zur Behandlung der Blatternkrankheit, Berlin, 1872.}
\item{Jakob Bernheim-Karrer, Gesundheitspflege des Kindes, 2. Auflage, Zürich, 1922.}
\item{Max Brucker, Ilse Gutjahr, Biologischer Ratgeber für Mutter und Kind, 5. Auflage, Lahnstein, 1987.}
\item{Gerhard Buchwald, Impfen. Das Geschäft mit der Angst, 3. Auflage, Lahnstein, 1995.}
\item{Johann de Carro, Beobachtungen und Erfahrungen über die Impfung der Kuhpocken, Wien, 1802.}
\item{Ludwig Fejes, Die Entstehung, Verbreitung und Verhütung der Seuchen, mit Erfahrungen aus dem Felde, Berlin, Wien, 1917.}
\item{Jana Gärtner, Elternratgeber im Wandel der Zeit. Deskriptive Ratgeberanalyse am Beispiel der sogenannten Klassischen Kinderkrankheiten unter Berücksichtigung der Impfdebatte, Berlin, 2010.}
\item{Heinrich Friedrich Germann, Historisch-Kritische Studien über den jetzigen Stand der Impffrage, 2. Band, Leipzig, 1875.}
\item{Franz Seraph Giel, Die Schutzpocken=Impfung in Bayern, vom Anbeginn ihrer Entstehung und gesetzlichen Einführung  bis auf gegenwärtige Zeit. Dann mit besonderer Beobachtung derselben in auswärtigen Staaten, München, 1830.}
\item{Friedrich Graf, Die Impfentscheidung. Ansichten, Überlegungen und Informationen -- vor jeglicher Ausführung!, 5. Auflage, Ascheberg, 2013.}
\item{Wolfgang Goebel u. Michaela Glöckler, Kinder Sprechstunde. Ein medizinisch-pädagogischer Ratgeber, Stuttgart 2005.}
\item{Martin Hirte, Impfen, Pro \& Contra. Das Handbuch für eine individuelle \\Impfentscheidung, München, 2008.}
\item{H. M. Husson, Historische und medizinische Untersuchungen über die Kuhpockenkrankheit, Marburg, 1801.}
\item{Georg Friedrich Krauss, Die Schutzpockenimpfung in ihrer endlichen Entscheidung, als Angelegenheit des Staats, der Familien und des Einzelnen, Nürnberg, 1820.}
\item{Johann Kumpfhofer, Predigt von der Pflicht der Eltern ihren Kindern die Kuhpocken einimpfen zu lassen, Linz, 1808.}
\item{Martina Lenzen-Schulte, Impfungen. 99 verblüffende Tatsachen, Wackernheim, 2008.}
\item{Wilhelm Mandt, Practische Darstellung der wichtigsten ansteckenden Epidemieen und Epizootien in ihrer Bedeutung für die medicinische Polizei, Berlin, 1828.}
\item{Theodor Munch, Der große Bluff. Irrwege und Lügen der Alternativmedizin, Berlin, 2013.}
\item{Carl Georg Gottlob Nittinger, Gott und Abgott der Impfhexe, Stuttgart, 1863.}
\item{Carl Georg Gottlob Nittinger, Das falsche Dogma von der Impfung und seine Rückwirkung auf Wissenschaft und Staat, München, 1857.}
\item{Joseph d'Outrepont, Belehrung des Landvolkes über die Schutzblattern. Nebst einem kurzen Unterrichte über die Impfung derselben für die Wundärzte, Salzburg, 1803.}
\item{Heinrich Oidtmann, Die historische und statistische Misshandlung der Impf-Frage im Reichstage zu Berlin 1878, Wien, 1879.}
\item{Gustav Paul, der Nutzen der Schutzpocken-Impfung. Vortrag gehalten am 30. März 1901 in der 87. Vollversammlung des Vereins für Kindergärten und Kinderbewahranstalten in Österreich, Wien, 1901.}
\item{Wilhelm Ressel, Das Impfgeschäft als starrstes Dogma der modernen orthodoxen Medizin. Richtigstellung falscher und gefährlicher zunftwissenschaftlicher Ueberlieferungen. Zugleich und hauptsächlich ein Weckruf an Deutschlands Zeitungs=Redakteure, Dreseden, 1910.}
\item{Carl Schreiber, Band I: Gründe gegen die allgemeine Kuhpockenimpfung, 2. Auflage, o. A., 1834, Reprint Göttingen, 1998.}
\item{M. Schulz, Impfung, Impfgeschäft und Impftechnik. Ein kurzer Leitfaden für Studierende und Arzte, Berlin, 1888.}
\item{N. N. Schürz, Ueber Epidemie, Contagium und Vaccination, Prag, 1866.}
\item{Rolf Schwarz, Impfen -- eine verborgene Gefahr? Impftheorie und Infektionstheorie auf dem Prüfstand, München, 2012.}
\item{Heinz Spiess, Schutzimpfungen, Stuttgart, 1958.}
\item{Christina Weiskopf, Abenteuer Impfung. Was Eltern über Kinderkrankheiten und Impfungen wissen sollen, Lappersdorf, 2007.}
\item{Siegfried Wolffberg, Über die Impfung: historisch-statistische Mittheilung über die Pockenepidemien und Impfung nebst einer Theorie der Schutzimpfung; ein Vortrag, Berlin, 1884.}
\item{David Zimmer, Der goldene Schatz der Kinderwelt. Ein Nachschlagbüchlein zur naturgemäßen, schnellen und einfachen Behandlung der am meisten vorkommenden Kinderkrankheiten, Wamsdorf, 1923.}
\end{itemize}
\newpage
\subsection{Liste der ausgewerteten Argumente}
\end{document}
